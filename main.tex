% Garante modo numerico do abnTeX2 para sincronizar com o estilo bibliografico
\PassOptionsToPackage{num}{abntex2cite}
\documentclass[11pt]{relatorio_tcc_ads_ifba}
\usepackage{float}
\usepackage{amsmath}
\usepackage{xurl} % Permite quebra automatica de URLs longas
\usepackage{microtype} % Melhora a tipografia e reduz overflows
\begin{document}

% Aluno/autor do documento (obrigatório)
% e.g. \aluno{Luiz Felipe Torres Farias}
\aluno{Pedro Batista de Almeida Filho}

% Titulo do seu projeto (obrigatório)
% e.g. \titulo{Desenvolvimento de Software XPTO}
\titulo{Tô De Olho: Democratizando a Transparência do Senado Federal através de Dados Abertos}

% Data da sua defesa (obrigatório)
% e.g. \date{21 de janeiro de 2024}
\date{Fevereiro de 2026}

% Orientador (obrigatório)
% Opção: [f] para orientadora. O valor default é [m]
% e.g. \orientador[f]{Flavia Maristela}
% e.g. \orientador{Frederico Barboza}
\orientador{Pablo Vieira Florentino}

% Coorientador (opcional)
% Opção: [f] para coorientadora. O valor default é [m]. Caso não possua coorientador, comentar ou deletar essa linha.
% e.g. \orientador[f]{Flavia Maristela}
% e.g. \orientador{Frederico Barboza}
% \coorientador{NOME}
%\coorientador{}

% Pretextual (obrigatório)
% Comando responsável por imprimir o conteúdo pré-textual (capa e sumário)
\pretextual

% ---------------------------------------------------------------------
% INÍCIO DA PARTE ESCRITA 
% ---------------------------------------------------------------------
\section{Visão geral }

O \textit{Tô De Olho} é uma plataforma \textit{web} de transparência parlamentar focada no Senado Federal. Sua proposta é aproximar cidadãos dos dados legislativos oficiais, convertendo informação dispersa em conhecimento fiscalizável e de fácil compreensão. O projeto vai além dos dados abertos básicos, integrando fontes complexas como a Cota para o Exercício da Atividade Parlamentar dos Senadores (CEAPS) e as ``emendas PIX''. Ao combinar uma arquitetura de \textbf{monolito modular} em Go, ingestão via APIs oficiais (Senado e Portal da Transparência) e um \textit{front-end} em Next.js, a plataforma busca reduzir a assimetria de informação sobre os 81 senadores da República \cite{Gomes2010}.

A literatura evidencia que TICs ampliam possibilidades de participação, mas só geram valor quando articuladas a contextos de uso. Avelino et al. mapeiam iniciativas, reforçando que tecnologias exigem visualizações para o controle social \cite{avelino2021democracia}. Com um corpo legislativo menor e mais ``caro'' \textit{per capita} que a Câmara, o Senado carece de ferramentas que cruzem votações com a execução orçamentária. À luz desses estudos, o \textit{Tô De Olho} procura transformar transparência passiva em \textit{accountability} ativa \cite{pateman1970, gomes2019}.

% Recomendo manter cada seção ou subseção de texto em um arquivo separado e depois utilizados com o comando include, conforme o exemplo a seguir. Porém, se preferir, também pode deixar seu texto neste arquivo
\subsection{Objetivos}

\subsubsection{Objetivo Geral}
Desenvolver uma plataforma \textit{web} de transparência política que centralize, organize e simplifique o acesso aos dados públicos do Senado Federal, fomentando a fiscalização cidadã e o debate qualificado sobre a atuação dos 81 senadores, com ênfase no monitoramento de gastos e emendas parlamentares.

\subsubsection{Objetivos Específicos}
\begin{itemize}
    \item Implementar um \textbf{\textit{backend} em \textit{Go}} com arquitetura de monolito modular para ingestão de dados das APIs oficiais do Senado e do Portal da Transparência;
    \item Desenvolver rotinas ETL para consumir as APIs Legislativa, Administrativa e do Portal da Transparência, priorizando fontes estruturadas;
    \item Desenvolver algoritmos de \textit{Ranking} baseados no estudo da metodologia \textit{Legislative Effectiveness \textit{Score}} (LES) de Volden e Wiseman (2014), adaptando critérios de efetividade legislativa ao contexto brasileiro para avaliar senadores com base em presença em votações, produtividade legislativa, economia na cota parlamentar e participação em comissões;
    \item Construir uma interface \textit{front-end} responsiva utilizando \textit{Next.js}, permitindo a visualização intuitiva de perfis, despesas e \textit{scorecards} de fiscalização;
    \item Garantir a acessibilidade e a usabilidade em dispositivos móveis por meio da adoção de componentes \textit{shadcn/ui}, construídos sobre \textit{Radix UI Primitives} com conformidade nativa às diretrizes WCAG 2.1 nível AA, e abordagem \textit{mobile-first} considerando o perfil de acesso à internet da população brasileira.
\end{itemize}
%\input{src/sections/definicoes}

\subsection{Declaração do Problema}

O Senado Federal disponibiliza dados públicos por meio de APIs próprias, enquanto a Controladoria-Geral da União (CGU) mantém o Portal da Transparência com dados de emendas parlamentares. Contudo, essas fontes encontram-se fragmentadas em órgãos distintos: a API Legislativa do Senado concentra informações sobre matérias e votações; a API Administrativa do Senado reúne dados da CEAPS e remunerações de gabinete; e o Portal da Transparência da CGU hospeda os registros de emendas e transferências federais. Para construir uma visão completa de um único senador, o cidadão precisaria consultar três sistemas de dois órgãos diferentes, com interfaces, formatos e periodicidades de atualização distintos.

Essa fragmentação adquire contornos mais graves quando analisamos as ``Transferências Especiais'' --- popularmente conhecidas como ``emendas PIX''. Criada em 2019, essa modalidade dispensa convênio e transfere recursos federais diretamente a estados e municípios. Alencar \cite{alencar2024emendaspix} demonstra que, do total de R\$ 20,5 bilhões transferidos por essa via, apenas R\$ 933 milhões tiveram prestação de contas adequada --- menos de 5\%. Em 2020, primeiro ano de vigência, as transferências especiais representavam 6,4\% das emendas individuais; em 2023, esse percentual saltou para 32,4\%. A distribuição é ainda mais desigual: no mesmo estado, alguns municípios receberam mais de R\$ 4.500 \textit{per capita}, enquanto outros receberam menos de R\$ 1 --- sem qualquer justificativa pública dos parlamentares.

Além da barreira técnica imposta pela fragmentação dos dados, há uma barreira social igualmente relevante. Segundo o Indicador de Alfabetismo Funcional \cite{inaf2024}, 29\% da população brasileira entre 15 e 64 anos é funcionalmente analfabeta, o que limita severamente a capacidade de interpretar planilhas, gráficos e relatórios disponibilizados nos portais oficiais. Nesse contexto, a simples disponibilização de dados brutos não garante transparência efetiva: é necessária uma ferramenta que consolide as informações dispersas e as apresente de forma visual e acessível, permitindo ao cidadão comum avaliar qualitativamente seus representantes \cite{avelino2021democracia}.

\subsection{Proposta de Solução de Software}

Diante da fragmentação de dados descrita e da barreira de letramento que impede o cidadão comum de interpretar planilhas e relatórios oficiais, propõe-se o \textit{Tô De Olho}: uma plataforma \textit{web} de código aberto concebida para centralizar a fiscalização do Senado Federal. A solução integra três APIs oficiais distintas --- Legislativa do Senado, Administrativa do Senado e Portal da Transparência da CGU --- consolidando informações dispersas em uma interface única e acessível.

O sistema organiza os dados em três dimensões complementares do mandato parlamentar:

\begin{itemize}
    \item \textbf{Atividade Legislativa}: votações nominais, participação em comissões, proposições de autoria e relatorias;
    \item \textbf{Gestão de Recursos}: despesas detalhadas da Cota Parlamentar (CEAPS), incluindo identificação de fornecedores e categorias de gasto;
    \item \textbf{Articulação Orçamentária}: emendas parlamentares com destaque para Transferências Especiais (``emendas PIX''), permitindo rastrear o destino dos recursos.
\end{itemize}

O diferencial da plataforma reside em quatro pilares:

\begin{enumerate}
    \item \textbf{Ranking Metodologicamente Fundamentado}: inspirado no \textit{State Legislative Effectiveness Score} (SLES) de Volden e Wiseman \cite{volden2018legislative}, o algoritmo de avaliação pondera produtividade legislativa (35\%), presença em votações (25\%), economia na cota parlamentar (20\%) e participação em comissões (20\%). Os critérios e pesos são públicos, permitindo ao cidadão compreender --- e questionar --- a metodologia;
    
    \item \textbf{Visualização Orientada à Ação}: seguindo os princípios de retórica visual de Hullman \cite{hullman2011visualization}, cada dado absoluto é contextualizado com médias comparativas, reduzindo a possibilidade de interpretações manipuladas e estimulando conclusões informadas;
    
    \item \textbf{Acessibilidade como Requisito}: a interface segue as diretrizes WCAG 2.1 nível AA, garantindo navegação por leitores de tela, contraste adequado e operação via teclado --- essencial para atingir os 29\% de brasileiros funcionalmente analfabetos identificados pelo INAF \cite{inaf2024};
    
    \item \textbf{Consolidação Multi-Fonte}: ao integrar dados de três órgãos distintos em uma única consulta, a plataforma elimina a necessidade de o cidadão navegar por sistemas heterogêneos com formatos e interfaces incompatíveis.
\end{enumerate}

Em síntese, o \textit{Tô De Olho} atua como um ``auditor digital'', automatizando cruzamentos de dados que, manualmente, seriam inviáveis para o eleitor comum. O objetivo não é substituir a análise crítica do cidadão, mas fornecer-lhe ferramentas para exercê-la de forma qualificada.



\subsection{Trabalhos Relacionados}

Diversas iniciativas no Brasil e no mundo buscam promover a transparência política por meio da tecnologia. À luz da Escada de Participação de Arnstein \cite{arnstein1969ladder}, podemos classificar essas ferramentas conforme o grau de poder que conferem ao cidadão.

\subsubsection{Portais Oficiais (Nível Informação)}

\textbf{Portal da Transparência (CGU):} Lançado em novembro de 2004 pela Controladoria-Geral da União, o Portal da Transparência consolidou-se como a principal ferramenta oficial do governo federal para acesso a dados de gastos públicos, servidores e transferências \cite{portaltransparencia2024}. Em 2018, o portal passou por reformulação completa para tornar a navegação mais intuitiva, e em 2024, ao completar 20 anos, recebeu novas atualizações que reafirmaram seu papel central no controle social. Os dados disponíveis abrangem execução orçamentária detalhada por órgão, remuneração individualizada de servidores, pagamentos de programas sociais (Bolsa Família, Auxílio Gás, Pé-de-Meia), licitações, contratos e --- particularmente relevante para este trabalho --- emendas parlamentares, incluindo registros relacionados à ADPF 854 sobre transparência orçamentária.

O portal registra entre 1,3 e 1,5 milhão de usuários únicos mensais, com aproximadamente 14 a 19 milhões de visualizações de página, demonstrando alto engajamento da sociedade civil e órgãos de controle. Para desenvolvedores, oferece uma API REST com limites de 90 requisições por minuto em horário comercial e 300 requisições por minuto durante a madrugada. Contudo, algumas limitações persistem: a periodicidade de atualização varia significativamente entre conjuntos de dados --- enquanto despesas e emendas são atualizadas diariamente, dados de imóveis funcionais podem apresentar defasagem superior a seis meses. Além disso, a granularidade contábil (com termos técnicos do Siafi --- Sistema Integrado de Administração Financeira do Governo Federal) representa barreira para cidadãos sem conhecimento em contabilidade pública.

\textbf{Portal de Dados Abertos do Senado Federal:} Lançado em 2012 em conformidade com a Lei de Acesso à Informação \cite{lai2011}, o portal foi institucionalizado pelo Ato da Comissão Diretora n. 14 de 2013, que estabeleceu a Política de Dados Abertos do Senado \cite{dadosabertossenado2024}. O ecossistema divide-se em duas APIs principais: a \textbf{API Legislativa}, que oferece dados sobre matérias, votações nominais, senadores e atividades de comissões; e a \textbf{API Administrativa}, focada em transparência de gastos (CEAPS), gestão de pessoas, orçamento e contratos. Os formatos suportados incluem JSON, XML e CSV, com documentação técnica via Swagger UI. A API possui limite de 10 requisições por segundo para garantir estabilidade.

\textbf{Portal de Dados Abertos da Câmara dos Deputados:} O fornecimento de dados legislativos pela Câmara iniciou-se em 2006 através do sistema SIT Câmara (Web Services) \cite{dadosabertoscamara2024}. Com a Lei de Acesso à Informação em 2011, o portal foi rebatizado como ``Dados Abertos'', eliminando a obrigatoriedade de cadastro prévio. Em 2017, lançou-se a API RESTful v2, substituindo os antigos Web Services por arquitetura mais moderna. O portal oferece \textit{endpoints} para deputados (perfis biográficos, discursos, frentes parlamentares), proposições (texto integral, tramitação), votações (incluindo voto individual de cada parlamentar) e cotas parlamentares (CEAP). Os formatos incluem JSON e XML via API, além de CSV, XLSX e ODS para \textit{downloads} em massa. O portal é referência em transparência legislativa, alimentando projetos como a Operação Serenata de Amor, Radar Governamental e VotoBom.

Embora esses portais representem avanços significativos na transparência passiva, situam-se no degrau mais básico da Escada de Arnstein --- informação bruta sem mediação interpretativa. O cidadão comum, sem conhecimento técnico sobre APIs ou contabilidade pública, enfrenta barreiras substanciais para transformar dados dispersos em fiscalização efetiva.

\subsubsection{Ferramentas de Fiscalização da Câmara dos Deputados}

O ecossistema de transparência para a Câmara dos Deputados é mais desenvolvido que para o Senado, contando com diversas iniciativas consolidadas:

\textbf{Operação Serenata de Amor:} Projeto pioneiro de código aberto, lançado em 2016 via financiamento coletivo, que utiliza inteligência artificial para detectar irregularidades em gastos parlamentares \cite{albuquerque2018serenata}. Desenvolvido pela Open Knowledge Brasil, é composto por dois sistemas complementares:

\begin{itemize}
    \item \textbf{Rosie:} Algoritmo de \textit{machine learning} desenvolvido em Python que audita a Cota para Exercício da Atividade Parlamentar (CEAP). Opera através de cinco classificadores principais: (1) \textit{meal price outlier} --- identifica refeições com valores acima da média para o local; (2) \textit{irregular companies} --- detecta gastos em empresas com situação cadastral irregular na Receita Federal; (3) \textit{traveled speeds} --- cruza gastos para identificar deslocamentos fisicamente impossíveis; (4) \textit{monthly subquota limit} --- verifica excesso nos limites mensais por categoria; e (5) \textit{election expenses} --- identifica uso indevido da cota para financiar campanhas.
    \item \textbf{Jarbas:} Interface \textit{web} desenvolvida em Django que permite aos cidadãos navegar pelos casos suspeitos identificados pela Rosie, visualizar notas fiscais digitalizadas e formalizar denúncias.
\end{itemize}

Até 2018, o projeto identificou \textbf{8.276 reembolsos suspeitos} envolvendo 735 deputados, totalizando aproximadamente \textbf{R\$ 3,6 milhões} em potenciais irregularidades. Um mutirão inicial resultou em 629 denúncias formais ao Congresso. Atualmente, a equipe principal migrou o foco para o projeto ``Querido Diário'', que aplica técnicas similares a diários oficiais municipais.

\textbf{De Olho no Congresso:} Plataforma \textit{web} moderna focada em gastos de Deputados Federais \cite{deolhonocongresso2024}. Com mais de \textbf{55 mil visitantes} e \textbf{95 mil consultas} realizadas, a ferramenta oferece interface acessível que inclui:

\begin{itemize}
    \item \textbf{Rankings Múltiplos:} Top 50 deputados com maiores gastos, ranking de partidos por consumo da cota e ranking de empresas fornecedoras --- incluindo filtro específico para ``empresas com sanções'' administrativas;
    \item \textbf{Painel de Alertas:} Sistema de detecção automática de despesas atípicas, incluindo: valores significativamente acima da média geral, pagamentos idênticos repetidos ao mesmo fornecedor, notas fiscais emitidas em finais de semana, e intervalos menores que 3 dias entre pagamentos;
    \item \textbf{Histórico Completo:} Gastos anuais e mensais com filtros por fornecedor, categoria e período, além de detalhamento de benefícios (auxílio-moradia, imóvel funcional) e equipe de gabinete.
\end{itemize}

A plataforma ressalta que os alertas são ``indicativos que merecem investigação'', servindo como guia para auditoria cidadã. Limita-se, porém, à Câmara dos Deputados e não oferece métricas de desempenho legislativo.

\textbf{De Olho em Você:} Plataforma \textit{web} com foco em ``transparência que dá para entender'', abrangendo aproximadamente 549 parlamentares da Câmara dos Deputados \cite{deolhoemvoce2024}. A plataforma integra dados da API da Câmara e do Portal da Transparência, destacando-se pela cobertura das \textbf{Emendas PIX} (Transferências Especiais). Entre suas funcionalidades principais:

\begin{itemize}
    \item \textbf{Mapas de Distribuição:} Cada perfil de deputado apresenta visualização geoespacial do destino de suas emendas parlamentares, permitindo identificar concentração de recursos por município;
    \item \textbf{Ranking de Cidades:} Classificação por faixa populacional das cidades que mais receberam Transferências Especiais (ex: municípios de até 20 mil habitantes);
    \item \textbf{Comparador de Parlamentares:} Ferramenta que permite selecionar de 2 a 5 deputados para comparação lado a lado de gastos de cota, equipe de gabinete, emendas enviadas e fornecedores em comum;
    \item \textbf{Painel de Fornecedores:} Ranking das empresas que mais recebem recursos, identificando padrões de concentração de gastos.
\end{itemize}

Entretanto, os rankings do ``De Olho em Você'' baseiam-se em métricas agregadas diretas (quem mais gastou, quem mais enviou emendas), \textbf{sem metodologia explícita de efetividade legislativa}. Além disso, a plataforma não contempla o Senado Federal.

\subsubsection{Experiências Internacionais}

\textbf{TheyWorkForYou (Reino Unido):} Lançada em 2004 e operada pela organização sem fins lucrativos mySociety \cite{mysociety2024}, a plataforma monitora cinco parlamentos britânicos: Câmara dos Comuns, Câmara dos Lordes, Parlamento Escocês, Senedd (País de Gales) e Assembleia da Irlanda do Norte \cite{theyworkforyou2024}. Em 2023/24, registrou mais de \textbf{4,8 milhões de visitas}. Entre suas funcionalidades:

\begin{itemize}
    \item \textbf{Hansard Pesquisável:} Arquivo completo de todos os discursos e debates parlamentares;
    \item \textbf{Alertas por E-mail:} Notificações automáticas quando um parlamentar específico discursa ou quando uma palavra-chave é mencionada;
    \item \textbf{Busca por Código Postal:} Identificação imediata do representante local.
\end{itemize}

A plataforma consolidou-se como referência mundial em \textit{civic tech} parlamentar, inspirando iniciativas em diversos países.

\textbf{OpenSecrets (Estados Unidos):} Principal organização de pesquisa sobre dinheiro na política americana, resultante da fusão em 2021 entre o \textit{Center for Responsive Politics} (fundado em 1983 por dois ex-senadores) e o \textit{National Institute on Money in State Politics} \cite{opensecrets2024}. Vencedora de múltiplos \textit{Webby Awards}, a plataforma rastreia:

\begin{itemize}
    \item \textbf{Financiamento de Campanhas:} Contribuições individuais, PACs e Super PACs;
    \item \textbf{Lobbying:} Gastos de empresas e grupos de interesse para influenciar legislação;
    \item \textbf{Revolving Door:} Monitoramento de ex-congressistas que se tornaram lobistas;
    \item \textbf{Dark Money:} Análise de fundos de origem não divulgada que influenciam eleições.
\end{itemize}

É fonte primária para veículos como \textit{The New York Times} e \textit{The Washington Post}, oferecendo APIs e exportações de dados para pesquisadores acadêmicos.

\subsubsection{Lacuna Identificada e Diferencial do Tô De Olho}

A análise sistemática dos trabalhos relacionados evidencia um cenário paradoxal: enquanto a Câmara dos Deputados --- com 513 parlamentares --- dispõe de ao menos três plataformas consolidadas de fiscalização cidadã, o Senado Federal permanece como uma ``caixa preta'' digital. Essa lacuna não é trivial. Os 81 senadores exercem mandatos de oito anos, atuam como câmara revisora de toda legislação federal e detêm competências exclusivas de alto impacto: confirmação de ministros do STF, julgamento de presidentes da República e aprovação de dívidas externas. A ausência de ferramentas de monitoramento específicas representa, portanto, uma falha sistêmica no ecossistema de \textit{accountability} brasileiro.

Mais do que apenas replicar soluções existentes para o âmbito senatorial, o \textit{Tô De Olho} propõe-se a \textbf{sintetizar o melhor de cada iniciativa} analisada, superando limitações identificadas:

\begin{itemize}
    \item \textbf{Do ``De Olho em Você''}, incorporamos a \textbf{visualização geoespacial de Emendas PIX} --- permitindo que o cidadão identifique, em mapas interativos, quais municípios receberam recursos de cada senador --- e o \textbf{comparador de parlamentares}, que possibilita análise lado a lado de até cinco senadores em múltiplas dimensões;
    
    \item \textbf{Do ``De Olho no Congresso''}, adotamos o \textbf{painel de alertas automáticos} para despesas atípicas --- notas fiscais em finais de semana, valores acima da média, pagamentos repetidos em intervalos curtos --- e o \textbf{ranking de fornecedores}, incluindo cruzamento com empresas sob sanção administrativa;
    
    \item \textbf{Do ``Serenata de Amor''}, adotamos a filosofia de \textbf{código aberto} e documentação transparente.
\end{itemize}

O diferencial central do \textit{Tô De Olho}, entretanto, reside em uma contribuição original: a implementação de um \textbf{Índice de Efetividade Legislativa} adaptado ao contexto brasileiro. Enquanto as ferramentas existentes limitam-se a ordenar parlamentares por \textit{volume de gastos} --- métrica que penaliza a parcimônia --- ou \textit{quantidade de proposições} --- que ignora a qualidade e o impacto legislativo ---, propomos um modelo multidimensional inspirado no \textit{State Legislative Effectiveness Score} (SLES) de Volden e Wiseman \cite{volden2018legislative}.

Nosso índice pondera quatro dimensões objetivas, com pesos públicos e metodologia reprodutível:

\begin{enumerate}
    \item \textbf{Produtividade Legislativa (35\%)}: Avalia a capacidade de transformar proposições em leis, com multiplicadores por tipo (PEC: 3x, PLP: 2x) e estágio de tramitação alcançado;
    
    \item \textbf{Presença em Votações (25\%)}: Mensura o comparecimento efetivo às sessões deliberativas, descontando ausências justificadas por licença médica ou missão oficial;
    
    \item \textbf{Economia na Cota Parlamentar (20\%)}: Compara o uso individual da CEAPS com a mediana do Senado, premiando a eficiência no uso de recursos públicos;
    
    \item \textbf{Participação em Comissões (20\%)}: Pondera o engajamento em comissões permanentes e especiais, com bônus para cargos de liderança (presidente, relator).
\end{enumerate}

A exposição pública de critérios e pesos não é mera formalidade: representa um compromisso ético com a \textbf{transparência metodológica}. Diferente de rankings opacos, o cidadão poderá compreender --- e questionar --- os fundamentos da classificação, evitando que a plataforma seja percebida como veículo de viés político.

Em síntese, o \textit{Tô De Olho} posiciona-se como a \textbf{primeira plataforma integrada} de fiscalização cidadã voltada ao Senado Federal, combinando três vertentes:

\begin{enumerate}
    \item \textbf{Consolidação Multi-Fonte}: Integra dados de três APIs oficiais (Legislativa, Administrativa e Portal da Transparência) em interface única;
    \item \textbf{Inteligência de Dados}: Oferece alertas automáticos, rankings e visualizações que convertem dados brutos em informação acionável;
    \item \textbf{Rigor Metodológico}: Fundamenta-se em literatura acadêmica sobre efetividade legislativa, com metodologia aberta a auditoria pública.
\end{enumerate}

% Fundamentação Teórica - Conceitos base do projeto
\section{Fundamentação Teórica}

Esta seção apresenta os conceitos fundamentais que embasam o desenvolvimento do \textit{Tô De Olho}, abrangendo transparência pública, democracia digital, teoria de accountability e arquitetura de software.

\subsection{Transparência Pública e Dados Abertos}

A transparência governamental constitui pilar fundamental do Estado Democrático de Direito. No Brasil, a Lei de Acesso à Informação (LAI --- Lei nº 12.527/2011) estabelece que o acesso é a regra e o sigilo, a exceção, garantindo aos cidadãos o direito de solicitar e receber informações públicas sem necessidade de justificativa \cite{lai2011}. A LAI impõe aos órgãos públicos o dever de divulgação proativa de informações de interesse coletivo, incluindo dados sobre despesas, contratos e remunerações de servidores.

A literatura distingue duas modalidades complementares de transparência. A \textbf{transparência ativa} ocorre quando o Estado disponibiliza informações de forma proativa em portais e bases de dados, independentemente de solicitação --- exemplificada pelos Portais de Dados Abertos do Senado Federal e da Câmara dos Deputados. A \textbf{transparência passiva}, por sua vez, responde às solicitações dos cidadãos via canais específicos como o Sistema Eletrônico do Serviço de Informação ao Cidadão (e-SIC). Enquanto a primeira amplia o acesso massivo a dados estruturados, a segunda garante o direito individual à informação específica \cite{avelino2021democracia}.

O conceito de \textit{Open Government Data} (Dados Governamentais Abertos) vai além da simples disponibilização: preconiza que as informações públicas devem ser liberadas em formatos abertos, processáveis por máquina e livres de licenças restritivas. Tim Berners-Lee, inventor da \textit{World Wide Web}, propôs em 2010 uma escala de cinco estrelas para avaliar a qualidade dos dados abertos \cite{bernerslee2010linked}:

\begin{enumerate}
    \item \textbf{Uma estrela}: Dados disponíveis na web em qualquer formato, sob licença aberta (ex: PDF escaneado);
    \item \textbf{Duas estrelas}: Dados estruturados e legíveis por máquina (ex: planilha Excel);
    \item \textbf{Três estrelas}: Dados em formato aberto não-proprietário (ex: CSV ao invés de Excel);
    \item \textbf{Quatro estrelas}: Dados identificados por URIs, seguindo padrões W3C como RDF;
    \item \textbf{Cinco estrelas}: Dados linkados a outras fontes, formando uma rede de conhecimento interoperável (\textit{Linked Open Data}).
\end{enumerate}

Os portais brasileiros de dados abertos situam-se predominantemente entre duas e três estrelas: oferecem arquivos CSV e JSON processáveis por máquina, mas raramente implementam identificadores únicos (URIs) ou linkagem semântica entre bases de dados de diferentes órgãos. A fragmentação identificada na Declaração do Problema --- três APIs de dois órgãos distintos --- exemplifica essa limitação: embora os dados sejam tecnicamente ``abertos'', a ausência de interoperabilidade impõe barreiras significativas à consolidação e análise integrada \cite{avelino2021democracia}.

O Brasil integra a \textit{Open Government Partnership} (OGP) desde 2011, tendo desenvolvido seis planos de ação nacionais com participação da sociedade civil. Essas iniciativas resultaram em 130 reformas voltadas à melhoria da governança e ao fortalecimento da Lei de Acesso à Informação \cite{ogp2024brazil}. Contudo, como demonstram as pesquisas sobre dados abertos, a mera disponibilização de informações não garante transparência efetiva: é necessário que os dados alcancem o público, que este tenha capacidade de processá-los, e que existam mecanismos institucionais para responsabilização dos agentes públicos.

\subsection{Democracia Digital e Participação Cidadã}

O conceito de democracia digital refere-se ao emprego de tecnologias de informação e comunicação (TICs) para produzir ``mais democracia e melhores democracias'' \cite{Gomes2010}. Wilson Gomes identifica três fases históricas neste campo: a \textbf{teledemocracia} (anos 1970-90), marcada por experimentos com televisão interativa e enquetes eletrônicas; a \textbf{fase da internet} (1995-2005), caracterizada pelo debate sobre potenciais e limites da rede para a participação política; e a \textbf{autonomização contemporânea}, onde subtemas como governo aberto, \textit{smart cities} e parlamento digital desenvolvem-se de forma independente, com metodologias e agendas próprias \cite{gomes2019}.

A participação cidadã mediada por tecnologia pode assumir diferentes níveis de profundidade e poder real. Sherry Arnstein, em seu trabalho seminal de 1969, propõe a ``Escada da Participação Cidadã'' (\textit{Ladder of Citizen Participation}), uma tipologia de oito degraus que classifica o grau de poder efetivamente conferido aos cidadãos \cite{arnstein1969ladder}:

\begin{itemize}
    \item \textbf{Não-participação} (degraus 1-2): \textit{Manipulação} e \textit{Terapia} --- formas em que o objetivo é ``educar'' ou ``curar'' os participantes, não ouvi-los. Comitês consultivos sem poder deliberativo exemplificam essa categoria.
    
    \item \textbf{Participação simbólica} (degraus 3-5): \textit{Informação}, \textit{Consulta} e \textit{Pacificação} --- cidadãos podem ouvir e ser ouvidos, mas sem garantia de que suas vozes influenciem decisões. Audiências públicas e pesquisas de opinião situam-se neste nível.
    
    \item \textbf{Poder cidadão} (degraus 6-8): \textit{Parceria}, \textit{Delegação de Poder} e \textit{Controle Cidadão} --- redistribuição efetiva de poder decisório. Orçamentos participativos vinculantes e conselhos com poder de veto exemplificam esses degraus superiores.
\end{itemize}

Ferramentas de transparência como o \textit{Tô De Olho} situam-se primariamente no degrau da \textbf{informação}: proveem ao cidadão dados estruturados sobre a atuação parlamentar, condição necessária --- mas não suficiente --- para o exercício pleno da fiscalização. Como adverte Arnstein, ``participação sem redistribuição de poder é um processo vazio e frustrante para os desprovidos de poder'' \cite{arnstein1969ladder}. Reconhecendo essa limitação estrutural, o projeto busca ir além da mera disponibilização de dados: ao oferecer \textit{rankings}, comparativos e visualizações contextualizadas, a plataforma \textbf{empodera} o cidadão para uma fiscalização mais qualificada, fornecendo-lhe ferramentas para exercer pressão informada sobre seus representantes.

No contexto brasileiro, a brecha digital representa obstáculo adicional à democracia digital. Segundo o INAF, 29\% da população entre 15 e 64 anos é funcionalmente analfabeta \cite{inaf2024}, limitando severamente a capacidade de interpretar planilhas, gráficos e relatórios técnicos. Avelino et al. mapeiam iniciativas de governo aberto no âmbito federal, identificando avanços significativos na disponibilização de dados, porém alertando que tecnologias precisam ser mediadas por visualizações claras e linguagem acessível para efetivação do controle social \cite{avelino2021democracia}.

\subsection{Teoria Principal-Agente e Accountability}

A relação entre cidadãos e representantes eleitos pode ser analisada através da lente da \textbf{teoria econômica de agência}, originalmente desenvolvida para compreender relações contratuais em organizações. Neste modelo, os cidadãos (eleitores) atuam como \textbf{principais} que delegam autoridade a \textbf{agentes} (parlamentares e burocratas) para tomar decisões em seu nome. O problema fundamental surge da \textbf{assimetria informacional}: os agentes possuem mais informação sobre suas próprias ações, esforços e competências do que os principais que os monitoram \cite{moe1984new}.

Esta assimetria manifesta-se de duas formas principais. O \textbf{risco moral} (\textit{moral hazard}) ocorre quando os principais não conseguem observar plenamente as ações dos agentes após a delegação: um parlamentar pode ``enrolar'' em suas funções, ausentar-se de votações ou priorizar interesses particulares sem que o eleitorado perceba. A \textbf{seleção adversa} (\textit{adverse selection}) surge antes da delegação: candidatos podem exagerar suas qualificações ou ocultar incompetências durante campanhas eleitorais. Em ambos os casos, a divergência entre os interesses do principal e do agente gera ``custos de agência'' --- ineficiências, corrupção ou políticas distorcidas.

O conceito de \textbf{accountability} --- frequentemente traduzido como ``prestação de contas'' ou ``responsabilização'' --- emerge como mecanismo para mitigar esses problemas. A literatura distingue três vertentes complementares \cite{alencar2024emendaspix}:

\begin{itemize}
    \item \textbf{Accountability vertical}: Exercida pelos eleitores através do voto. Parlamentares que não atendem às expectativas podem ser punidos nas urnas. Contudo, eleições são instrumentos ``grosseiros'': ocorrem periodicamente (a cada 4-8 anos para senadores), envolvem múltiplas dimensões de avaliação simultâneas, e dependem de que o eleitor tenha informação suficiente para julgar o desempenho.
    
    \item \textbf{Accountability horizontal}: Exercida por instituições de controle como tribunais de contas, controladorias, Ministério Público e o próprio Poder Judiciário. Esses órgãos possuem competência técnica e acesso privilegiado a informações, mas enfrentam limitações de capacidade operacional diante do volume de atos a fiscalizar.
    
    \item \textbf{Accountability social}: Exercida por organizações da sociedade civil, imprensa investigativa, acadêmicos e cidadãos organizados. Esta modalidade complementa as anteriores ao ampliar a capacidade de monitoramento e pressionar por transparência.
\end{itemize}

A transparência é condição necessária --- mas não suficiente --- para a accountability. Pesquisas demonstram que para a divulgação de informações traduzir-se em responsabilização efetiva, três condições devem ser satisfeitas: (1) a informação deve efetivamente alcançar o público relevante; (2) o público deve ter capacidade de processá-la e reagir a ela; e (3) devem existir mecanismos institucionais que permitam consequências para os agentes. A simples disponibilização de dados brutos --- o que alguns autores chamam de ``governo nu'' (\textit{naked government}) --- pode até amplificar percepções negativas sem gerar accountability efetiva.

O \textit{Tô De Olho} posiciona-se como ferramenta de \textbf{accountability social}, reduzindo a assimetria informacional entre eleitores e senadores. Ao consolidar dados fragmentados, contextualizar valores com médias comparativas e oferecer rankings metodologicamente fundamentados, a plataforma amplia a capacidade do cidadão de monitorar seus representantes --- condição prévia para o exercício tanto da accountability vertical (voto informado) quanto da pressão por accountability horizontal (denúncias fundamentadas a órgãos de controle).

\subsection{Métricas de Efetividade Legislativa}

A avaliação quantitativa do desempenho parlamentar constitui tema relevante na ciência política contemporânea. Craig Volden e Alan E. Wiseman, co-diretores do \textit{Center for Effective Lawmaking}, desenvolveram o \textit{Legislative Effectiveness Score} (LES), uma métrica que mensura a capacidade de parlamentares em conduzir suas proposições através do processo legislativo \cite{volden2014lawmakers, volden2018legislative}.

A metodologia do LES fundamenta-se em cinco estágios do processo legislativo, cada qual representando um grau crescente de sucesso na agenda do parlamentar:

\begin{enumerate}
    \item \textbf{Introdução}: O projeto é formalmente apresentado;
    \item \textbf{Ação em comissão}: O projeto recebe parecer ou é debatido em comissão temática;
    \item \textbf{Votação em plenário (câmara de origem)}: O projeto é levado à votação na casa onde foi apresentado;
    \item \textbf{Aprovação na câmara de origem}: O projeto é aprovado e segue para a outra casa;
    \item \textbf{Conversão em lei}: O projeto completa a tramitação e é sancionado.
\end{enumerate}

Crucialmente, nem todos os projetos possuem igual peso na metodologia. Volden e Wiseman categorizam as proposições em três níveis de significância: \textbf{projetos comemorativos} (como denominação de logradouros), que recebem peso mínimo; \textbf{projetos substantivos}, que alteram políticas públicas de forma moderada; e \textbf{projetos substantivos e significativos}, que promovem mudanças estruturais relevantes. Esta ponderação evita que parlamentares ``inflem'' suas estatísticas com proposições triviais \cite{thelawmakers2024}.

Os autores identificaram fatores consistentemente correlacionados à maior efetividade legislativa: senioridade no mandato, ocupação de posições em comissões estratégicas (especialmente presidências e relatorias), pertencimento ao partido majoritário e experiência prévia em legislaturas estaduais. Interessantemente, pesquisas subsequentes demonstraram que, controlando demais variáveis, parlamentares mulheres tendem a ser mais efetivas que seus colegas homens \cite{volden2018legislative}.

Embora desenvolvida para o contexto norte-americano, a metodologia oferece um \textit{framework} adaptável para avaliar parlamentares brasileiros. No \textit{Tô De Olho}, o ``Score'' do senador inspira-se nesta abordagem, combinando quatro dimensões com pesos públicos:

\begin{itemize}
    \item \textbf{Produtividade Legislativa (35\%)}: Avalia proposições de autoria e relatorias, com multiplicadores por tipo (PEC: 3x, PLP: 2x) e estágio de tramitação alcançado;
    \item \textbf{Presença em Votações (25\%)}: Mensura comparecimento efetivo às sessões deliberativas;
    \item \textbf{Economia na Cota Parlamentar (20\%)}: Compara uso individual da CEAPS com a mediana do Senado;
    \item \textbf{Participação em Comissões (20\%)}: Pondera engajamento em comissões, com bônus para cargos de liderança.
\end{itemize}

A transparência metodológica --- expor publicamente os critérios e pesos utilizados --- é fundamental para que o \textit{ranking} seja percebido como ferramenta de informação, não de manipulação política. Diferente de classificações opacas, o cidadão poderá compreender --- e questionar --- os fundamentos da avaliação.

\subsection{Visualização de Dados e Retórica Visual}

A apresentação de dados ao cidadão não é neutra: escolhas de \textit{design} influenciam profundamente a interpretação das informações. Jessica Hullman e Nicholas Diakopoulos investigaram os ``efeitos de enquadramento'' (\textit{framing effects}) em visualizações narrativas, demonstrando que técnicas retóricas como seleção, omissão, ênfase e sequenciamento podem direcionar a leitura do público de forma consciente ou inconsciente \cite{hullman2011visualization}.

Os autores identificam quatro categorias de técnicas retóricas em visualizações de dados:

\begin{enumerate}
    \item \textbf{Proveniência}: Identificação da origem e credibilidade dos dados. Visualizações que ocultam fontes ou datas de atualização comprometem a verificabilidade. No \textit{Tô De Olho}, cada gráfico exibe a fonte oficial (API do Senado ou Portal da Transparência) e a data da última sincronização.
    
    \item \textbf{Mapeamento visual}: Como elementos gráficos representam variáveis numéricas. Escalas inconsistentes, truncamento de eixos ou escolhas de cores podem distorcer percepções. A plataforma adota escalas consistentes em gráficos comparativos e paletas de cores acessíveis.
    
    \item \textbf{Anotações linguísticas}: Textos, títulos e legendas que guiam a interpretação. Valores absolutos sem contexto podem induzir conclusões equivocadas (ex: ``Senador X gastou R\$ 100 mil'' parece muito sem saber que a média é R\$ 150 mil). O \textit{Tô De Olho} contextualiza valores com médias e percentis.
    
    \item \textbf{Interatividade}: Controles que permitem ao usuário explorar dados por conta própria reduzem a dependência de narrativas pré-construídas. A plataforma oferece filtros por partido, estado e período, permitindo análises personalizadas.
\end{enumerate}

A \textbf{literacia em visualização de dados} (\textit{Data Visualization Literacy} --- DVL) refere-se à capacidade de interpretar corretamente representações visuais de informações. Pesquisas demonstram que mesmo populações com alta escolaridade frequentemente cometem erros de interpretação em gráficos aparentemente simples. No contexto brasileiro, onde 29\% da população é funcionalmente analfabeta \cite{inaf2024}, o desafio é ainda maior: visualizações complexas podem excluir justamente os cidadãos mais vulneráveis à falta de transparência.

Para o \textit{Tô De Olho}, esses princípios orientam decisões de \textit{design}: priorizar visualizações simples e intuitivas; oferecer múltiplas formas de apresentação (gráficos, tabelas, textos explicativos); e testar a compreensibilidade com usuários de diferentes perfis. O objetivo é maximizar a transparência metodológica, evitando que a plataforma seja percebida como veículo de viés político.

\subsection{Civic Tech e Sociedade Civil}

O termo \textbf{civic tech} (tecnologia cívica) refere-se ao uso de tecnologias digitais para fortalecer a participação cidadã, a transparência governamental e a colaboração entre sociedade e Estado. Diferencia-se de \textit{GovTech} (tecnologia para eficiência governamental interna) por seu foco na interface com o cidadão e no empoderamento da sociedade civil.

No Brasil, o ecossistema de civic tech consolidou-se a partir de 2010, impulsionado pela Lei de Acesso à Informação (2011) e pelo crescimento de organizações especializadas. Entre as iniciativas mais relevantes:

\begin{itemize}
    \item \textbf{Transparência Brasil}: Fundada em 2000, é uma das principais organizações dedicadas à promoção da integridade e supervisão cívica no setor público. Desenvolve pesquisas, relatórios e ferramentas para monitoramento de políticas públicas \cite{transparenciabrasil2024}.
    
    \item \textbf{Open Knowledge Brasil}: Responsável pela Operação Serenata de Amor e pelo projeto ``Querido Diário'', que aplica técnicas de inteligência artificial para auditar diários oficiais municipais.
    
    \item \textbf{Fiquem Sabendo}: Organização sem fins lucrativos que utiliza civic tech de código aberto para expor gastos governamentais não divulgados, tendo revelado mais de 500 bilhões de reais em despesas não reportadas ao longo de 27 anos.
    
    \item \textbf{Associação Brasileira de Jornalismo Investigativo (Abraji)}: Emprega jornalismo de dados e ferramentas tecnológicas para investigar corrupção e má gestão de recursos públicos.
\end{itemize}

Apesar dos avanços, o ecossistema enfrenta desafios significativos. A \textbf{sustentabilidade financeira} é precária: muitas iniciativas dependem de financiamento coletivo, bolsas internacionais ou trabalho voluntário, limitando sua capacidade de operação contínua. A \textbf{brecha digital} restringe o alcance a populações com menor acesso à internet ou habilidades tecnológicas limitadas. E a \textbf{fragmentação de esforços} --- múltiplas ferramentas com escopos parcialmente sobrepostos --- dispersa recursos e dificulta o engajamento do cidadão comum.

O \textit{Tô De Olho} posiciona-se neste ecossistema com um diferencial claro: o foco exclusivo no \textbf{Senado Federal}, câmara legislativa até então carente de ferramentas específicas de fiscalização cidadã. Enquanto a Câmara dos Deputados conta com ao menos três plataformas consolidadas (Serenata de Amor, De Olho no Congresso, De Olho em Você), os 81 senadores --- que exercem mandatos de oito anos e detêm competências exclusivas de alto impacto --- permanecem em relativa ``sombra'' digital.

\subsection{Arquitetura de Software: Monolito Modular}

A arquitetura de \textbf{monolito modular} representa uma abordagem intermediária entre o monolito tradicional e os microsserviços \cite{dragoni2017microservices}. Enquanto o monolito tradicional tende a se tornar uma ``bola de lama'' (\textit{big ball of mud}) com o crescimento, e microsserviços introduzem complexidade operacional significativa (orquestração, rede, consistência eventual), o monolito modular organiza a aplicação em módulos bem definidos dentro de um único artefato de \textit{deploy}.

Esta escolha é particularmente adequada para equipes pequenas e projetos acadêmicos, oferecendo benefícios de organização e manutenibilidade sem a complexidade operacional de sistemas distribuídos. Os principais benefícios incluem: simplicidade de \textit{deploy} com um único contêiner em ambiente \textit{serverless} (Cloud Run); baixa latência entre módulos, já que a comunicação ocorre via chamadas de função em memória; e facilidade de evolução futura, permitindo eventual migração para microsserviços se a escala justificar.

\subsection{Engenharia de Dados: APIs e Processos ETL}

A estratégia de ingestão de dados do \textit{Tô De Olho} fundamenta-se no padrão ETL (\textit{Extract, Transform, Load}): extração dos dados brutos das fontes oficiais, transformação para normalização e enriquecimento, e carga no banco de dados da aplicação.

A abordagem adotada consome exclusivamente APIs RESTful oficiais: a API Legislativa do Senado (matérias, votações, comissões), a API Administrativa (CEAPS, remunerações de servidores) e a API do Portal da Transparência (emendas parlamentares). O processo combina duas estratégias complementares: \textbf{backfill} para carga histórica inicial abrangendo dados desde 2019, e \textbf{sincronização incremental} diária via tarefas agendadas (\textit{CronJobs}) que capturam atualizações recentes.

A escolha por APIs oficiais --- em detrimento de técnicas de \textit{web scraping} --- fundamenta-se em critérios de confiabilidade e manutenibilidade: APIs possuem contratos mais estáveis, formatos estruturados (JSON/XML) e documentação oficial, reduzindo a fragilidade do sistema a mudanças de layout em páginas web.

\subsection{Emendas PIX e Desafios de Transparência Orçamentária}

As Transferências Especiais, popularmente conhecidas como ``emendas PIX'', constituem modalidade de repasse de recursos federais a estados e municípios criada pela Emenda Constitucional nº 105/2019. Diferentemente de convênios tradicionais, que exigem plano de trabalho prévio e prestação de contas detalhada, as emendas PIX transferem recursos diretamente às contas dos entes federativos com discricionariedade ampla sobre sua aplicação.

A pesquisa de Alencar \cite{alencar2024emendaspix} revela deficiências graves de transparência fiscal nesta modalidade. Do total de R\$ 20,5 bilhões transferidos via emendas PIX até 2023, apenas R\$ 933 milhões --- menos de 5\% --- tiveram prestação de contas adequada. A evolução do instrumento é expressiva: em 2020, primeiro ano de vigência, as transferências especiais representavam 6,4\% das emendas individuais; em 2023, esse percentual saltou para 32,4\%.

A distribuição territorial revela disparidades extremas: no mesmo estado, alguns municípios receberam mais de R\$ 4.500 \textit{per capita} via emendas PIX, enquanto outros receberam menos de R\$ 1 --- sem qualquer justificativa pública dos parlamentares sobre os critérios de alocação. Esta opacidade compromete as três vertentes de \textit{accountability} discutidas anteriormente: a \textbf{vertical} (eleitores não conseguem avaliar escolhas de seus representantes), a \textbf{horizontal} (tribunais de contas enfrentam dificuldades de fiscalização) e a \textbf{social} (jornalistas e pesquisadores encontram dados fragmentados e incompletos).

O \textit{Tô De Olho} aborda esta lacuna ao integrar dados de emendas do Portal da Transparência com informações legislativas do Senado, permitindo que o cidadão visualize, para cada senador: o total de recursos destinados via transferências especiais, os municípios beneficiados e a evolução temporal dos repasses. Ao consolidar informações dispersas em interface única, a plataforma contribui para reduzir a opacidade que caracteriza esta modalidade orçamentária.


% Metodologia - Abordagem de desenvolvimento e arquitetura
\section{Metodologia}

O desenvolvimento do \textit{Tô De Olho} enquadra-se no paradigma da \textit{Design Science Research} (DSR), abordagem metodológica adequada para pesquisas que visam a construção e avaliação de artefatos tecnológicos destinados a resolver problemas organizacionais identificados \cite{hevner2004design}. O problema abordado --- a fragmentação de dados públicos sobre a atuação de senadores federais em múltiplas APIs governamentais não integradas --- demanda a construção de um artefato de software capaz de consolidar, processar e apresentar essas informações de forma acessível ao cidadão.

Esta seção detalha a abordagem de desenvolvimento iterativo adotada, as fontes de dados governamentais integradas, a arquitetura do sistema, a estratégia de ingestão via ETL, o algoritmo de ranking inspirado no \textit{Legislative Effectiveness Score} e a infraestrutura de implantação.

\subsection{Abordagem de Desenvolvimento}

A natureza do projeto --- integração de múltiplas APIs governamentais com estruturas de dados heterogêneas e documentação variável --- demandou uma abordagem de desenvolvimento capaz de acomodar descobertas incrementais e ajustes frequentes de escopo. Metodologias tradicionais de desenvolvimento em cascata, que pressupõem requisitos estáveis e bem definidos desde o início, mostraram-se inadequadas para este contexto de exploração de APIs públicas com comportamentos nem sempre previsíveis.

Optou-se, portanto, por uma abordagem \textbf{iterativa e incremental}, na qual o trabalho foi organizado em ciclos de desenvolvimento focados em entregas funcionais. Cada ciclo produzia um incremento utilizável do sistema, permitindo validação contínua das funcionalidades implementadas e ajustes baseados nos aprendizados obtidos durante a integração com cada API.

O desenvolvimento contou com apoio de ferramentas de inteligência artificial generativa como assistentes de codificação, seguindo a tendência contemporânea de \textit{AI-assisted software development} \cite{peng2023impact}. Tais ferramentas, baseadas em modelos de linguagem de grande escala (LLMs), foram empregadas para aceleração de tarefas operacionais como geração de código \textit{boilerplate}, refatoração e \textit{debugging}. O uso de assistentes de IA em desenvolvimento de software representa uma evolução natural das ferramentas de produtividade, análoga à adoção de IDEs com \textit{autocomplete} e analisadores estáticos de código.

É fundamental distinguir entre \textbf{assistência operacional} e \textbf{autoria intelectual}. As decisões estruturantes do projeto --- arquitetura do sistema, design do algoritmo de ranking, seleção de fontes de dados, modelagem do domínio e interpretação de resultados --- foram integralmente concebidas, avaliadas e validadas pelo desenvolvedor. A ferramenta de IA atuou como acelerador de implementação, não como substituto do julgamento técnico. A engenharia de contexto --- técnica de estruturação de informações em arquivos de configuração que definem padrões de codificação, stack tecnológico e regras de negócio --- foi aplicada para maximizar a aderência das sugestões aos padrões do projeto, mantendo consistência arquitetural ao longo do desenvolvimento.

A divisão do trabalho ocorreu em cinco fases principais:
\begin{enumerate}
    \item \textbf{Fundação}: Estruturação do projeto em Golang, implementação do cliente para a API Legislativa do Senado, criação das \textit{migrations} do banco de dados e configuração inicial do \textit{frontend} em Next.js;
    \item \textbf{Ingestão de Dados}: Implementação do cliente para a API Administrativa, configuração do \textit{scheduler} para tarefas agendadas, coleta de votações nominais e carga de dados históricos;
    \item \textbf{Ranking e API}: Desenvolvimento do serviço de cálculo de rankings, criação dos \textit{endpoints} REST para consumo pelo \textit{frontend}, configuração do cache Redis e implementação de testes automatizados;
    \item \textbf{Frontend}: Desenvolvimento do \textit{dashboard} principal, interface de ranking interativo e páginas de perfil dos senadores;
    \item \textbf{Emendas e Polimento}: Integração com o Portal da Transparência para dados de emendas parlamentares, visualizações de dados e preparação para \textit{deploy}.
\end{enumerate}

\subsection{Fontes de Dados}

O sistema integra três fontes de dados governamentais oficiais, fundamentadas no arcabouço legal brasileiro de transparência pública. A Lei de Acesso à Informação (Lei n. 12.527/2011) estabelece como diretriz a ``disponibilização de informações em formatos abertos, estruturados e legíveis por máquina'' \cite{lai2011}, princípio que as APIs governamentais operacionalizam. Conforme demonstrado pela Operação Serenata de Amor, tecnologias desenvolvidas sobre esses dados abertos podem gerar valor público ao facilitar o controle social do gasto parlamentar \cite{albuquerque2018serenata}.

A seleção dos \textit{endpoints} e campos consumidos seguiu três critérios: (i) \textbf{relevância para fiscalização cidadã} --- priorizando dados de gastos, votações e atuação legislativa; (ii) \textbf{disponibilidade e confiabilidade} --- selecionando fontes com documentação oficial e atualização regular; e (iii) \textbf{viabilidade técnica} --- considerando formatos estruturados (JSON/XML), autenticação simples e limites de requisição adequados.

\subsubsection{API Legislativa do Senado}

Documentada em \url{legis.senado.leg.br/dadosabertos}, esta API RESTful fornece dados do processo legislativo. A URL base para requisições é \url{https://legis.senado.leg.br/dadosabertos}. Os dados são retornados em formato JSON, com suporte a paginação e limite de 10 requisições por segundo.

\textbf{Endpoints de Senadores e Mandatos:}
\begin{itemize}
    \item \texttt{/dadosabertos/senador/lista/atual}: Lista de senadores em exercício, retornando código parlamentar, nome, foto, partido e UF;
    \item \texttt{/dadosabertos/senador/\{codigo\}}: Detalhes completos do senador, incluindo biografia, e-mail e telefone;
    \item \texttt{/dadosabertos/senador/\{codigo\}/mandatos}: Histórico de mandatos com legislatura, tipo e datas;
    \item \texttt{/dadosabertos/senador/\{codigo\}/licencas}: Licenças oficiais com motivo e período --- essencial para ajuste dos cálculos de presença.
\end{itemize}

\textbf{Endpoints de Votações:}
\begin{itemize}
    \item \texttt{/dadosabertos/votacao}: Votações nominais filtráveis por código do parlamentar, data e orientação de voto;
    \item \texttt{/dadosabertos/votacaoComissao/parlamentar/\{codigo\}}: Votações em comissões específicas;
    \item \texttt{/dadosabertos/plenario/votacao/orientacaoBancada/\{data\}}: Orientação partidária para cálculo de fidelidade.
\end{itemize}

\textbf{Endpoints de Atuação Legislativa:}
\begin{itemize}
    \item \texttt{/dadosabertos/processo?codigoParlamentarAutor=\{codigo\}}: Proposições de autoria do senador;
    \item \texttt{/dadosabertos/processo/relatoria?codigoParlamentar=\{codigo\}}: Relatorias designadas;
    \item \texttt{/dadosabertos/senador/\{codigo\}/comissoes}: Participação em comissões;
    \item \texttt{/dadosabertos/senador/\{codigo\}/discursos}: Discursos proferidos.
\end{itemize}

\subsubsection{API Administrativa do Senado}

Documentada em \url{adm.senado.gov.br/adm-dadosabertos/swagger-ui}, esta interface disponibiliza dados financeiros e administrativos. A URL base para requisições é \url{https://adm.senado.gov.br/adm-dadosabertos}.

\textbf{Cota Parlamentar (CEAPS):}
\begin{itemize}
    \item \texttt{/api/v1/senadores/despesas\_ceaps/\{ano\}}: Despesas da Cota para o Exercício da Atividade Parlamentar dos Senadores, retornando código do senador, tipo de despesa, fornecedor, CPF/CNPJ, valor reembolsado e mês.
\end{itemize}

\textbf{Estrutura e Benefícios:}
\begin{itemize}
    \item \texttt{/api/v1/senadores/auxilio-moradia}: Opção por auxílio-moradia;
    \item \texttt{/api/v1/senadores/escritorios}: Escritórios de apoio parlamentar.
\end{itemize}

\textbf{Servidores de Gabinete:}
\begin{itemize}
    \item \texttt{/api/v1/servidores/servidores}: Lista de servidores com filtros por lotação;
    \item \texttt{/api/v1/servidores/remuneracoes/\{ano\}/\{mes\}}: Folha de pagamento mensal, retornando remuneração básica e líquida;
    \item \texttt{/api/v1/servidores/lotacoes}: Mapeamento entre senador e gabinete.
\end{itemize}

\subsubsection{Portal da Transparência (CGU)}

A API do Portal da Transparência (\url{api.portaldatransparencia.gov.br}) fornece dados de emendas parlamentares e transferências federais. O acesso requer autenticação via chave de API no \textit{header} \texttt{chave-api-dados}.

\textbf{Endpoint Principal:}
\begin{itemize}
    \item \texttt{/api-de-dados/emendas}: Lista de emendas parlamentares com parâmetros de filtro: \texttt{ano} (ano fiscal), \texttt{nomeAutor} (nome do parlamentar), \texttt{tipoEmenda} (Individual, Bancada, Comissão, Relator ou \textbf{Transferência Especial}).
\end{itemize}

O filtro \texttt{tipoEmenda=Transferência Especial} permite identificar as chamadas ``emendas PIX'', modalidade de repasse que dispensa convênio. Os campos retornados incluem código da emenda, autor, localidade do gasto, função, subfunção e valores (empenhado, liquidado e pago).

\subsubsection{Limitações e Cobertura Temporal}

Cada fonte de dados apresenta limitações que impactam o escopo e a profundidade das análises possíveis:

\begin{itemize}
    \item \textbf{API Legislativa}: Votações nominais disponíveis apenas a partir de 2019 (legislatura 56); dados anteriores requerem consulta a arquivos históricos em formato XML. Limite de 10 requisições por segundo;
    \item \textbf{API Administrativa}: Despesas CEAPS disponíveis desde 2008 em CSV; API REST cobre apenas o ano corrente. Não há \textit{endpoint} para vincular servidores diretamente ao gabinete do senador --- essa associação requer cruzamento com a tabela de lotações;
    \item \textbf{Portal da Transparência}: Emendas disponíveis a partir de 2015. Busca por autor utiliza correspondência textual (nome), não código parlamentar, exigindo normalização de grafia. Limite de 300 requisições por minuto com autenticação via chave de API.
\end{itemize}

Essas restrições foram consideradas no desenho do sistema: dados históricos são carregados via arquivos CSV (backfill), enquanto a sincronização contínua utiliza as APIs REST com tratamento adequado de limites de requisição.

\subsection{Estratégia de Ingestão de Dados}

A ingestão de dados representa um dos maiores desafios técnicos do projeto: consolidar informações dispersas em três APIs governamentais distintas, cada uma com suas idiossincrasias, limites de requisição e formatos de resposta. A estratégia adotada segue o paradigma \textit{Extract, Transform, Load} (ETL) em uma abordagem híbrida que combina carga inicial massiva (\textit{backfill}) com atualização incremental contínua.

Conforme fundamentado teoricamente na Seção~\ref{sec:etl-fundamentacao}, o processo ETL tipicamente consome entre 60\% e 80\% do esforço de desenvolvimento em projetos de \textit{data warehousing} \cite{vassiliadis2009survey}. Essa proporção reflete a complexidade inerente à extração de dados de fontes heterogêneas e à necessidade de transformações para garantir consistência e qualidade. A presente seção detalha a implementação específica adotada no \textit{Tô De Olho}, conforme ilustrado na Tabela~\ref{tab:ingestao}.

\begin{table}[H]
\centering
\caption{Estratégias de ingestão de dados: Backfill vs. Sincronização Contínua}
\label{tab:ingestao}
\begin{tabular}{|l|p{5.5cm}|p{5.5cm}|}
\hline
\textbf{Característica} & \textbf{Backfill (Carga Inicial)} & \textbf{Sincronização Contínua} \\
\hline
\textbf{Objetivo} & Popular o banco com dados históricos & Manter dados atualizados \\
\hline
\textbf{Execução} & Uma única vez, no setup do sistema & Diariamente, via cron jobs \\
\hline
\textbf{Volume de dados} & Grande (anos de histórico) & Pequeno (delta diário) \\
\hline
\textbf{Tempo de execução} & Horas (processamento em lote) & Minutos (\textless 5 min) \\
\hline
\textbf{Fonte preferencial} & CSVs para download em massa & APIs REST incrementais \\
\hline
\textbf{Exemplo} & CEAPS 2019--2025 via CSV & Votações dos últimos 7 dias \\
\hline
\end{tabular}
\end{table}

\subsubsection{Backfill (Carga Histórica)}

A carga inicial (\textit{Backfill}) distingue-se fundamentalmente da sincronização diária pelo volume e pela estratégia de escrita. Conforme preceituado por Kimball e Ross \cite{kimball2013data}, operações de carga histórica devem privilegiar o \textit{throughput} (vazão) em detrimento da latência individual. Esta recomendação reflete décadas de experiência prática: tentar carregar milhões de registros através de inserções unitárias resulta em tempos de execução proibitivos e pressão desnecessária sobre as APIs de origem.

Para este projeto, optou-se pela estratégia de \textbf{Backfill em Lote Isolado}. Ao invés de reutilizar os \textit{endpoints} REST --- projetados para transações unitárias e limitados por \textit{rate limiting} rigoroso --- o sistema consome arquivos CSV consolidados disponibilizados pelas fontes primárias. Esta abordagem reduz o \textit{overhead} de rede e permite o uso de \texttt{COPY} ou \textit{bulk inserts} no banco de dados, operações que alcançam taxas de inserção na ordem de dezenas de milhares de registros por segundo, ordens de magnitude superiores a inserções linha a linha.

A distinção entre carga inicial e sincronização contínua não é meramente operacional, mas também arquitetural. O processo de \textit{backfill} executa em ambiente isolado, podendo utilizar conexões de banco com configurações agressivas (\textit{batch size} elevado, desativação temporária de índices não-essenciais) sem impactar a disponibilidade do sistema em produção. As fontes de dados para carga inicial incluem:
\begin{itemize}
    \item \textbf{CEAPS}: Arquivos CSV anuais (2008--2025) disponíveis via \textit{endpoint} de dados abertos estáticos do Senado, totalizando aproximadamente 500 mil registros de despesas;
    \item \textbf{Votações}: Iteração controlada na API Legislativa por ano/legislatura (2019--Presente), respeitando limites de 10 requisições por segundo;
    \item \textbf{Emendas}: \textit{Dump} consolidado do Portal da Transparência, com dados desde 2015.
\end{itemize}

\subsubsection{Sincronização Contínua (Change Data Capture)}

Uma vez que os dados históricos foram carregados, o desafio seguinte consiste em mantê-los atualizados de forma eficiente. Idealmente, as APIs governamentais ofereceriam mecanismos de \textit{push} --- como \textit{Webhooks} ou fluxos de eventos (\textit{Streaming}) --- permitindo que o sistema fosse notificado proativamente sobre mudanças. Entretanto, nenhuma das três fontes de dados integradas disponibiliza tais recursos, uma limitação comum em APIs governamentais brasileiras.

Diante dessa restrição, adotou-se o padrão de \textbf{Polling Periódico Inteligente}, no qual o sistema atua como agente ativo que consulta periodicamente as fontes em busca de alterações (\textit{Pull Model}). Embora menos eficiente que abordagens baseadas em eventos, o \textit{polling} oferece vantagens significativas para integração com sistemas legados: simplicidade de implementação, ausência de dependências de infraestrutura complexa (como \textit{message brokers}) e compatibilidade garantida com qualquer API que suporte requisições HTTP.

A periodicidade dos \textit{cron jobs} foi calibrada com base na \textbf{Janela de Consistência Eventual} aceitável para cada domínio de dado. O conceito de consistência eventual, originário de bancos de dados distribuídos, reconhece que em sistemas descentralizados, os dados podem estar temporariamente desatualizados desde que eventualmente convirjam para o estado correto. Para um portal de fiscalização cidadã, a precisão ao segundo não é necessária; o que importa é que os dados reflitam a realidade em um horizonte razoável:

\begin{itemize}
    \item \textbf{Alta Volatilidade (Diário)}: Status de senadores, presença e votações do dia anterior. A execução ocorre às 03:00 (horário de Brasília), período de baixa atividade nos servidores governamentais, capturando o fechamento completo do dia legislativo;
    \item \textbf{Média Volatilidade (Semanal)}: Remuneração de servidores e mapeamento de lotações. A folha de pagamento do Senado é atualizada mensalmente, mas a execução semanal (domingos às 04:00) garante margem de segurança para capturar eventuais correções;
    \item \textbf{Baixa Volatilidade (Mensal)}: Consolidação de emendas parlamentares. Os dados de execução orçamentária do Portal da Transparência são consolidados mensalmente, justificando a execução no dia 05 de cada mês, após o fechamento contábil público.
\end{itemize}

Esta estratégia de \textit{polling} respeita os princípios de ``Cidadania de API'' (\textit{API Good Citizenship}), concentrando requisições em horários de baixo tráfego para minimizar impacto nos servidores governamentais e maximizar a taxa de sucesso das chamadas.

\subsubsection{Idempotência e Consistência Eventual}

Em sistemas distribuídos sujeitos a falhas de rede, a garantia de entrega \textit{exactly-once} é virtualmente impossível de alcançar na prática \cite{hummer2013modeldriven}. Qualquer comunicação entre componentes pode falhar após o processamento mas antes da confirmação, deixando o cliente em estado de incerteza. Portanto, o sistema foi projetado para operar sob o modelo \textit{at-least-once}, no qual operações podem ser repetidas sem efeitos colaterais indesejados, apoiado pela propriedade de \textbf{Idempotência}.

Matematicamente, uma operação de ingestão $f(x)$ é idempotente se $f(f(x)) = f(x)$. Ou seja, aplicar a função múltiplas vezes produz o mesmo resultado que aplicá-la uma única vez. Esta propriedade, originária da álgebra abstrata, encontra aplicação direta em operações de banco de dados: uma inserção \texttt{INSERT OR UPDATE} (\textit{upsert}) com chave primária definida é idempotente, pois tentativas subsequentes simplesmente sobrescrevem o registro existente com valores idênticos.

A implementação de idempotência no \textit{Tô De Olho} utiliza chaves naturais compostas (\textit{Composite Natural Keys}) em vez de identificadores gerados. Esta escolha reflete uma decisão arquitetural deliberada: os sistemas de origem (APIs do Senado e Portal da Transparência) são a fonte de verdade para identificação de entidades. As operações de \textit{upsert} garantem:
\begin{itemize}
    \item \textbf{Despesas CEAPS}: Unicidade garantida pela tupla \texttt{(senador\_id, fornecedor\_cnpj, data\_emissao, valor\_centavos)}. Esta combinação identifica univocamente cada reembolso, considerando que o mesmo senador não pode receber dois reembolsos idênticos do mesmo fornecedor no mesmo dia;
    \item \textbf{Votações}: Unicidade via \texttt{(id\_sessao, id\_parlamentar)}. Cada senador registra no máximo um voto por sessão de votação;
    \item \textbf{Emendas}: Unicidade via \texttt{(codigo\_emenda, autor, ano)}. O código de emenda é único por autor e ano fiscal.
\end{itemize}

Essa abordagem elimina a necessidade de gerenciamento de estado complexo no \textit{pipeline} de ingestão: em caso de dúvida, falha de rede ou interrupção inesperada, o sistema pode simplesmente reprocessar o lote inteiro, garantindo convergência para o estado correto sem risco de duplicação de dados.

\subsubsection{Resiliência e Tratamento de Erros}

A integração com APIs governamentais apresenta desafios de estabilidade: diferentemente de serviços comerciais com \textit{SLAs} bem definidos, as APIs públicas brasileiras operam sem garantias formais de disponibilidade. Para lidar com falhas transientes, o sistema implementa mecanismos simples e eficazes:

\begin{itemize}
    \item \textbf{Retry com Backoff Exponencial}: Para erros de rede ou respostas 5xx, o sistema aguarda progressivamente mais tempo entre tentativas. A fórmula segue o padrão recomendado pela AWS \cite{aws2015backoff}: $t_{wait} = base \times 2^{tentativa}$, com $base = 1s$ e máximo de 3 tentativas. Esta abordagem evita sobrecarregar APIs temporariamente indisponíveis;
    
    \item \textbf{Timeout Configurável}: Cada requisição possui timeout de 30 segundos, evitando que conexões travadas bloqueiem o processo de ingestão;
    
    \item \textbf{Falha Graciosa}: Quando uma fonte de dados está indisponível, o job registra o erro e prossegue com as demais fontes. Jobs que falham são marcados para reexecução manual.
\end{itemize}

Para cenários de produção com maior volume de requisições, padrões como \textit{Circuit Breaker} \cite{nygard2018release} poderiam ser adicionados para isolar falhas em cascata. Entretanto, dado o escopo do MVP --- com execuções diárias em horário de baixo tráfego --- a estratégia de retry simples atende adequadamente às necessidades.

\subsubsection{Logs e Monitoramento}

O monitoramento do \textit{pipeline} de ingestão adota uma abordagem pragmática, adequada ao escopo do projeto:

\begin{itemize}
    \item \textbf{Logs Estruturados}: Todos os eventos de ingestão são emitidos em formato JSON, contendo: identificador do job, fonte de dados, duração, quantidade de registros processados e eventual mensagem de erro. Esta estruturação facilita a análise posterior e permite identificar rapidamente a causa de falhas;
    
    \item \textbf{Endpoint de Health}: O sistema expõe um \textit{endpoint} \texttt{/health} que retorna o status da última sincronização de cada fonte, permitindo verificar se os dados estão atualizados.
\end{itemize}

Para evolução futura, a literatura recomenda a adoção dos três pilares da observabilidade --- logs, métricas e traces \cite{sridharan2018observability}. Ferramentas como Prometheus (métricas) e Jaeger (tracing distribuído) poderiam ser integradas conforme a complexidade operacional aumentar.

\subsection{Arquitetura do Sistema}

A definição da arquitetura de software constitui uma das decisões mais consequentes no ciclo de desenvolvimento. Uma escolha inadequada impõe custos que se acumulam ao longo do tempo: dívida técnica, dificuldade de evolução e, em casos extremos, a necessidade de reescrever sistemas inteiros. Conforme fundamentado teoricamente na Seção~2.3.3, o padrão de \textbf{monolito modular} emergiu na última década como alternativa pragmática que busca equilibrar a simplicidade operacional dos monolitos tradicionais com a organização interna característica de microsserviços \cite{laigner2024modular}.

O \textit{Tô De Olho} adota esta arquitetura não por inércia, mas por escolha deliberada após avaliação de alternativas. As subseções seguintes detalham a estrutura organizacional, os padrões de comunicação e o fluxo de dados que materializam essa decisão.

\subsubsection{Justificativa Arquitetural}

A decisão pelo monolito modular considerou quatro critérios interrelacionados:

\begin{itemize}
    \item \textbf{Simplicidade Operacional}: Um único contêiner Docker simplifica a infraestrutura de implantação. Em plataformas \textit{serverless} como Google Cloud Run, essa característica traduz-se em escala automática (inclusive a zero) sem a complexidade de orquestração de múltiplos serviços --- vantagem particularmente relevante para projetos com orçamento limitado e equipe reduzida;
    
    \item \textbf{Latência Previsível}: A comunicação entre módulos ocorre via chamadas de função em memória, eliminando a variabilidade de latência inerente a chamadas de rede. Em microsserviços, cada interação adiciona \textit{overhead} de serialização, transmissão e desserialização --- custos que, embora individualmente pequenos, acumulam-se em operações que atravessam múltiplos serviços;
    
    \item \textbf{Consistência Transacional}: Operações que envolvem múltiplos módulos compartilham uma única transação de banco de dados, garantindo atomicidade sem a complexidade de \textit{sagas} ou \textit{two-phase commit}. Por exemplo, a atualização do ranking de um senador após nova ingestão de votações executa atomicamente: ou todos os dados são persistidos, ou nenhum;
    
    \item \textbf{Estratégia MonolithFirst}: Martin Fowler argumenta que ``quase todos os casos de sucesso de microsserviços começaram com um monolito que cresceu grande demais e foi decomposto'' \cite{fowler2015monolith}. Iniciar com fronteiras modulares bem definidas permite compreender os limites naturais do domínio antes de incorrer na complexidade operacional de sistemas distribuídos. A estrutura atual facilita extração futura de microsserviços apenas onde houver necessidade comprovada (por exemplo, se o módulo de ranking exigir escalabilidade independente).
\end{itemize}

\subsubsection{Organização em Módulos}

A estrutura interna segue os princípios de \textit{Bounded Contexts} do Domain-Driven Design \cite{evans2003ddd}, onde cada módulo representa um contexto de domínio com responsabilidades claramente delimitadas. O diretório \texttt{internal/} do backend Go organiza-se em cinco módulos principais:

\begin{itemize}
    \item \texttt{internal/senador}: Gerencia o ciclo de vida de dados cadastrais dos parlamentares --- nome, partido, UF, foto, biografia --- e seus mandatos históricos. Este módulo atua como ``fonte de verdade'' para identificação de senadores, fornecendo identificadores únicos que os demais módulos referenciam;
    
    \item \texttt{internal/ceaps}: Processa e totaliza despesas da Cota para o Exercício da Atividade Parlamentar dos Senadores. Responsável pela ingestão de arquivos CSV históricos e sincronização incremental via API Administrativa, este módulo implementa detecção de anomalias (valores negativos, datas futuras) e agregações por categoria de despesa;
    
    \item \texttt{internal/votacao}: Coleta e armazena votações nominais do plenário e comissões. Calcula métricas derivadas como percentual de presença (ajustado por licenças oficiais), fidelidade partidária e padrões de votação;
    
    \item \texttt{internal/emenda}: Integra dados de emendas parlamentares do Portal da Transparência, com tratamento especial para Transferências Especiais (``emendas PIX''). Normaliza a correspondência textual de autores --- desafio não trivial dado que o Portal usa nomes, não códigos parlamentares;
    
    \item \texttt{internal/ranking}: Orquestra o cálculo de \textit{scores} compostos conforme metodologia detalhada na Seção~3.6. Opera como consumidor dos demais módulos, agregando métricas de produtividade, presença, economia e participação em comissões.
\end{itemize}

Adicionalmente, módulos transversais fornecem funcionalidades compartilhadas: \texttt{internal/api} expõe \textit{endpoints} REST consumidos pelo \textit{frontend}; \texttt{internal/scheduler} gerencia tarefas agendadas de sincronização; e \texttt{internal/cache} abstrai operações de leitura e escrita no Redis.

\subsubsection{Padrões de Comunicação}

Os módulos comunicam-se exclusivamente através de \textbf{interfaces Go} (contratos), nunca acessando estruturas internas de outros módulos diretamente. Esta restrição, inspirada no princípio de inversão de dependência, permite substituir implementações sem afetar consumidores --- por exemplo, trocar o adaptador de cache de Redis para memória local em testes unitários.

Para operações de ingestão que envolvem centenas de requisições HTTP, o sistema emprega \textbf{pools de workers} com \textit{goroutines}. O modelo de concorrência de Go, baseado em CSP (\textit{Communicating Sequential Processes}), permite processamento paralelo com footprint de memória reduzido: cada \textit{goroutine} consome aproximadamente 2KB iniciais \cite{nanz2015comparative}, viabilizando milhares de operações simultâneas em hardware modesto.

O padrão adotado segue a estrutura de ``fan-out/fan-in'': uma \textit{goroutine} coordenadora distribui tarefas para \textit{N} workers (fan-out), que processam independentemente e enviam resultados para um canal agregador (fan-in). Esta abordagem maximiza utilização de CPU em operações \textit{I/O-bound} como chamadas HTTP, mantendo o código legível e testável.

\subsubsection{Fluxo de Dados}

O ciclo completo de dados no sistema segue quatro estágios bem definidos:

\begin{enumerate}
    \item \textbf{Ingestão}: Os módulos de domínio (\texttt{senador}, \texttt{ceaps}, \texttt{votacao}, \texttt{emenda}) extraem dados das APIs governamentais conforme estratégia detalhada na Seção~3.3. Cada módulo implementa seu próprio adaptador HTTP, respeitando as idiossincrasias de cada fonte (paginação, limites de requisição, formatos de resposta);
    
    \item \textbf{Transformação}: Dados brutos são normalizados, validados e enriquecidos antes da persistência. Esta fase inclui padronização de nomes (tratamento de acentos, maiúsculas/minúsculas), conversão de formatos de data e cálculo de métricas derivadas. O GORM, ORM utilizado, abstrai as operações de \textit{upsert} que garantem idempotência;
    
    \item \textbf{Agregação}: O módulo de \texttt{ranking} consulta os demais módulos para computar \textit{scores} compostos. Resultados são materializados no Redis como estruturas pré-computadas, evitando recálculos a cada requisição. A invalidação de cache ocorre após cada sincronização bem-sucedida;
    
    \item \textbf{Apresentação}: A API REST expõe os dados agregados ao \textit{frontend} Next.js. Endpoints seguem convenções RESTful, com respostas paginadas e suporte a filtros por partido, UF e período temporal.
\end{enumerate}

Esta separação em estágios facilita a depuração (cada estágio pode ser testado isoladamente), a evolução (novos módulos integram-se sem alterar existentes) e a observabilidade (logs estruturados identificam claramente em qual estágio ocorreu uma falha).

\subsection{Stack Tecnológico}

A escolha das tecnologias foi orientada por critérios de desempenho, manutenibilidade e adequação ao domínio do problema. Para cada camada do sistema, foram avaliadas alternativas antes da decisão final, conforme detalhado a seguir.

\subsubsection{Backend --- Golang}

A linguagem Go foi selecionada para o \textit{backend} após avaliação comparativa de três alternativas:

\begin{itemize}
    \item \textbf{Node.js}: Embora ofereça vasto ecossistema e familiaridade com JavaScript, o modelo de concorrência baseado em \textit{event loop} único apresenta limitações para operações CPU-bound. Além disso, a gestão de dependências via npm frequentemente resulta em árvores de módulos complexas;
    \item \textbf{Python (FastAPI/Django)}: Excelente para prototipagem rápida e possui bibliotecas maduras para análise de dados. Entretanto, o \textit{Global Interpreter Lock} (GIL) limita o paralelismo real, e o desempenho em operações I/O-bound é inferior ao de linguagens compiladas;
    \item \textbf{Go}: Binários compilados estaticamente, modelo de concorrência nativo e tipagem estática. O fator decisivo foi o modelo de \textbf{goroutines}: threads leves gerenciadas pelo runtime Go, consumindo aproximadamente 2KB de memória inicial --- em contraste com threads do sistema operacional que utilizam cerca de 1MB cada \cite{nanz2015comparative}.
\end{itemize}

Essa eficiência permite processar centenas de milhares de conexões simultâneas, característica essencial para a ingestão paralela de dados de três APIs distintas com limites de requisição distintos.

O framework \textbf{Gin} foi escolhido por seu roteamento HTTP baseado em \textit{radix tree}, reportando desempenho até 40 vezes superior a frameworks anteriores \cite{alfian2024gin}. Alternativas como \textit{Echo} e \textit{Fiber} foram consideradas, mas Gin apresenta comunidade mais ativa e documentação mais abrangente. O ORM \textbf{GORM} oferece mapeamento objeto-relacional com suporte a migrações automáticas e operações de \textit{upsert} para garantia de idempotência.

Para ambientes de produção, o sistema implementa \textbf{connection pooling} --- técnica que mantém um cache de conexões reutilizáveis, evitando o custo de estabelecer novas conexões a cada requisição. Estudos demonstram que esta abordagem pode reduzir a utilização de CPU do banco de dados em até 30\% e aumentar significativamente o \textit{throughput} em cenários de alta concorrência. A configuração define limites para conexões ociosas (\texttt{MaxIdleConns}), conexões abertas (\texttt{MaxOpenConns}) e tempo máximo de vida (\texttt{ConnMaxLifetime}), prevenindo acúmulo de conexões obsoletas. Adicionalmente, a opção \texttt{PrepareStmt} habilita cache de \textit{prepared statements}, otimizando queries repetidas ao reutilizar planos de execução.

\subsubsection{Banco de Dados --- PostgreSQL e Redis}

Para a camada de persistência, foram avaliadas três opções:

\begin{itemize}
    \item \textbf{MySQL}: Popular e bem documentado, mas com suporte limitado a tipos de dados avançados e funcionalidades analíticas;
    \item \textbf{MongoDB}: Modelo de documentos oferece flexibilidade de schema, porém a natureza relacional dos dados parlamentares (senador $\rightarrow$ mandatos $\rightarrow$ votações) favorece bancos relacionais;
    \item \textbf{PostgreSQL}: Robustez em consultas analíticas, suporte a índices compostos e CTEs (\textit{Common Table Expressions}), essenciais para agregações por senador, período e tipo de despesa.
\end{itemize}

\textbf{PostgreSQL} foi selecionado. Originado em 1986 na Universidade da Califórnia, Berkeley \cite{stonebraker1986postgres}, sua conformidade ACID garante integridade nas operações de ingestão. Funcionalidades como \texttt{JSONB} permitem armazenar respostas brutas das APIs para auditoria, combinando benefícios de bancos relacionais e documentais.

\textbf{Redis} atua como camada de cache para rankings pré-computados e totalizadores de gastos. Por armazenar dados inteiramente em memória RAM, Redis alcança latências típicas entre 100 e 500 microssegundos \cite{redis2024}, eliminando a necessidade de recalcular métricas a cada requisição. Alternativas como Memcached foram descartadas pela ausência de estruturas de dados avançadas (sorted sets, hashes) que Redis oferece nativamente.

\subsubsection{Frontend --- Next.js 15}

Para a interface web, foram consideradas:

\begin{itemize}
    \item \textbf{React puro (SPA)}: Flexibilidade máxima, porém com SEO comprometido e necessidade de configuração manual de roteamento, SSR e otimizações;
    \item \textbf{Vue.js/Nuxt}: Curva de aprendizado suave, mas ecossistema menor para visualização de dados;
    \item \textbf{Next.js}: Framework React com SSR/SSG nativos, roteamento baseado em sistema de arquivos e otimizações automáticas de imagem e fonte.
\end{itemize}

A escolha de \textbf{Next.js} justifica-se pela necessidade de \textbf{SEO} e performance inicial \cite{salim2024nextjs}. Plataformas de fiscalização cidadã dependem de indexação por mecanismos de busca para alcançar seu público-alvo. Next.js resolve o problema de indexação com \textit{Server-Side Rendering} (SSR) e \textit{Static Site Generation} (SSG), entregando HTML pré-renderizado aos \textit{crawlers}.

A biblioteca \textbf{Recharts} foi selecionada para visualização de dados por sua integração nativa com React \cite{recharts2024}. Alternativas como D3.js oferecem maior controle, mas exigem implementação manual de componentes; Chart.js tem API imperativa menos compatível com o paradigma declarativo do React. \textbf{Tailwind CSS} permite estilização eficiente com classes utilitárias; a partir da versão 4, o modo \textit{Just-In-Time} (JIT) gera apenas o CSS utilizado \cite{tailwindcss2024}.

\subsection{Modelo de Dados}

O modelo relacional foi projetado para garantir integridade referencial e eficiência em consultas analíticas. A Figura~\ref{fig:er-diagram} apresenta o diagrama entidade-relacionamento do sistema, incluindo as sete entidades principais e seus relacionamentos.

\begin{figure}[H]
\centering
\includegraphics[width=0.95\textwidth]{docs/diagrama_er_modelo.png}
\caption{Diagrama entidade-relacionamento do sistema Tô De Olho}
\label{fig:er-diagram}
\par\small{Fonte: Autoria Própria}
\end{figure}

As entidades principais e seus relacionamentos são:

\begin{itemize}
    \item \textbf{Senador}: Entidade central com código parlamentar único (\texttt{codigo\_parlamentar}), nome, partido, UF e URL da foto. O código parlamentar, fornecido pela API Legislativa, atua como chave natural para integração entre sistemas;
    \item \textbf{Mandato}: Relacionamento 1:N com Senador, registrando legislatura, tipo (titular/suplente) e período. Um senador pode ter múltiplos mandatos em diferentes legislaturas;
    \item \textbf{Votação}: Implementado como tabela associativa entre Senador e Sessão, armazenando voto individual (Sim, Não, Abstenção, Obstrução) e orientação partidária. A combinação \texttt{(sessao\_id, senador\_id)} forma chave única;
    \item \textbf{Despesa CEAPS}: Relacionamento 1:N com Senador, contendo ano, mês, tipo de despesa, fornecedor, CPF/CNPJ e valor. A chave composta \texttt{(senador\_id, fornecedor\_cnpj, data\_emissao, valor\_centavos)} garante idempotência na ingestão;
    \item \textbf{Servidor de Gabinete}: Relacionamento 1:N com Senador, registrando nome, cargo, vínculo funcional e remuneração bruta/líquida;
    \item \textbf{Emenda}: Relacionamento 1:N com Senador, contendo ano, tipo (Individual, Bancada, Transferência Especial), localidade do gasto e valores (empenhado, liquidado e pago);
    \item \textbf{Comissão Membro}: Relacionamento 1:N com Senador, registrando participação em comissões permanentes e temporárias. Armazena sigla e nome da comissão, cargo exercido (Titular, Suplente, Presidente, Vice-Presidente) e período de atuação. Esta entidade é essencial para o cálculo do critério ``Participação em Comissões'' (20\%) do algoritmo de ranking.
\end{itemize}

\textbf{Estratégia de Indexação:} Índices compostos foram criados nas colunas de consulta frequente para otimizar operações de agregação:
\begin{itemize}
    \item \texttt{idx\_despesa\_senador\_ano}: \texttt{(senador\_id, ano)} para totalização de gastos por período;
    \item \texttt{idx\_votacao\_sessao\_senador}: \texttt{(sessao\_id, senador\_id)} para consultas de presença;
    \item \texttt{idx\_emenda\_senador\_tipo}: \texttt{(senador\_id, tipo, ano)} para filtros por modalidade de emenda;
    \item \texttt{idx\_comissao\_senador}: \texttt{(senador\_id, comissao\_sigla)} para consultas de participação em comissões.
\end{itemize}

Esta modelagem permite consultas analíticas eficientes, como ranking de senadores por economia na cota, distribuição geográfica de emendas ou pontuação por participação em comissões estratégicas, sem necessidade de joins custosos em tempo de execução.

\subsection{Algoritmo de Ranking}

A avaliação objetiva do desempenho parlamentar constitui elemento central para a participação cidadã informada. Conforme a escada de participação de Arnstein \cite{arnstein1969ladder}, o acesso a informações claras e comparáveis é pré-requisito para que cidadãos avancem de níveis meramente consultivos para formas efetivas de controle social. Entretanto, a construção de rankings requer transparência metodológica: critérios e pesos devem ser explícitos para que o público possa avaliar criticamente os resultados apresentados \cite{hullman2011visualization}.

O cálculo do \textit{score} de cada senador é inspirado no \textit{Legislative Effectiveness Score} (LES), metodologia desenvolvida por Volden e Wiseman para avaliar a efetividade legislativa de parlamentares americanos \cite{volden2018legislative}. O LES define efetividade legislativa como ``a capacidade comprovada de avançar itens da agenda de um parlamentar através do processo legislativo até sua transformação em lei'' \cite{volden2014lawmakers}. A metodologia original utiliza uma matriz de 5 estágios de tramitação por 3 tipos de proposição, resultando em 15 indicadores ponderados.

Para a adaptação brasileira, manteve-se a filosofia de valorizar o avanço de proposições através do processo legislativo, incorporando critérios adicionais relevantes ao contexto de fiscalização cidadã. O índice proposto compõe-se de quatro dimensões objetivamente mensuráveis: \textbf{produtividade legislativa} (capacidade de aprovar proposições), \textbf{presença em votações} (compromisso com o mandato), \textbf{economia na cota parlamentar} (responsabilidade fiscal) e \textbf{participação em comissões} (trabalho técnico especializado). A adaptação considerou as especificidades do Senado Federal e a disponibilidade de dados via API.

\subsubsection{Critérios e Pesos}

O índice compõe-se de quatro critérios objetivos, cujos pesos refletem sua importância relativa para a avaliação de desempenho parlamentar:

\begin{table}[H]
\centering
\caption{Critérios do algoritmo de ranking}
\label{tab:criterios}
\begin{tabular}{|l|c|l|}
\hline
\textbf{Critério} & \textbf{Peso} & \textbf{Justificativa} \\
\hline
Produtividade Legislativa & 35\% & Core do LES: capacidade de aprovar proposições \\
Presença em Votações & 25\% & Compromisso efetivo com o mandato \\
Economia na Cota (CEAPS) & 20\% & Responsabilidade fiscal \\
Participação em Comissões & 20\% & Trabalho técnico especializado \\
\hline
\end{tabular}
\end{table}

\subsubsection{Produtividade Legislativa}

Este critério avalia o avanço de proposições de autoria do senador através do processo legislativo, atribuindo pontuação crescente por estágio alcançado:

\begin{table}[H]
\centering
\caption{Pontuação por estágio de tramitação}
\label{tab:estagios}
\begin{tabular}{|l|c|c|c|}
\hline
\textbf{Estágio} & \textbf{Pontos} & \textbf{Mult. PEC} & \textbf{Mult. PLP} \\
\hline
Apresentado & 1 & $\times$3.0 & $\times$2.0 \\
Em Comissão & 2 & $\times$3.0 & $\times$2.0 \\
Aprovado em Comissão & 4 & $\times$3.0 & $\times$2.0 \\
Aprovado em Plenário & 8 & $\times$3.0 & $\times$2.0 \\
Transformado em Lei & 16 & $\times$3.0 & $\times$2.0 \\
\hline
\end{tabular}
\end{table}

Os multiplicadores refletem a maior complexidade de Propostas de Emenda Constitucional (PEC), que exigem quórum qualificado de 3/5 em duas votações, e Projetos de Lei Complementar (PLP), que requerem maioria absoluta. Projetos de Lei Ordinária (PL) utilizam o multiplicador base ($\times$1.0). Requerimentos e moções, por seu menor impacto legislativo, recebem multiplicador reduzido ($\times$0.5).

\textbf{Bônus de Relatoria}: Senadores que atuam como relatores recebem pontuação adicional: +4 pontos para relatoria de PEC, +2 pontos para PLP ou PL, e +1 ponto para relatoria em comissão permanente.

\subsubsection{Presença em Votações}

Calculada como a razão entre votações participadas e votações disponíveis:

\begin{equation}
\text{Presença} = \frac{\text{Votações Participadas}}{\text{Votações Disponíveis}} \times 100
\end{equation}

As votações disponíveis excluem períodos de licença oficial (médica ou para cargo executivo), garantindo que afastamentos justificados não penalizem o senador. Obstrução \textbf{não} conta como presença.

\subsubsection{Economia na Cota Parlamentar}

Calculada como a proporção não utilizada do teto da CEAPS:

\begin{equation}
\text{Economia} = \left(1 - \frac{\text{Gasto Senador}}{\text{Teto CEAPS}}\right) \times 100
\end{equation}

O teto da CEAPS varia por UF (R\$ 26.000 para DF até R\$ 44.000+ para estados da região Norte). Senadores com gasto acima de 120\% do teto recebem pontuação zero neste critério.

\subsubsection{Participação em Comissões}

Pontuação atribuída conforme o nível de engajamento:

\begin{itemize}
    \item Membro titular de comissão permanente: +2 pontos por comissão;
    \item Membro suplente: +1 ponto por comissão;
    \item Vice-presidente de comissão: +3 pontos;
    \item Presidente de comissão: +5 pontos;
    \item Membro de comissão temporária ou CPI: +1 ponto.
\end{itemize}

Comissões estratégicas recebem multiplicador adicional: CAE (Assuntos Econômicos) e CCJ (Constituição e Justiça) aplicam $\times$1.5, enquanto CAS (Assuntos Sociais) e CI (Infraestrutura) aplicam $\times$1.2.

\subsubsection{Fórmula Final}

Cada métrica é normalizada de 0 a 100 antes da ponderação. O \textit{score} final é calculado por:

\begin{equation}
Score = (\text{Produtividade} \times 0.35) + (\text{Presença} \times 0.25) + (\text{Economia} \times 0.20) + (\text{Comissões} \times 0.20)
\end{equation}

\subsubsection{Tratamento de Casos Especiais}

\begin{itemize}
    \item \textbf{Senadores novos (suplentes)}: Período mínimo de 30 dias para entrar no ranking, com \textit{badge} ``Novo'' por 6 meses;
    \item \textbf{Licença curta (\textless 30 dias)}: Mantido no ranking com ajuste nos denominadores;
    \item \textbf{Licença longa (\textgreater 30 dias)}: Exibido com \textit{badge} ``Licenciado'' e \textit{score} congelado;
    \item \textbf{Dados indisponíveis}: Critério afetado recebe peso zero; demais critérios são reponderados proporcionalmente.
\end{itemize}

\subsection{Infraestrutura e Implantação}

Todos os componentes são containerizados com Docker, utilizando \textit{multi-stage builds} para otimização das imagens. A implantação ocorre via Google Cloud Run, que oferece escala automática (inclusive a zero) e \textit{deploy} simplificado.

\subsubsection{Containerização e CI/CD}

O \textit{pipeline} de CI/CD, implementado com GitHub Actions, automatiza as etapas:
\begin{enumerate}
    \item \textbf{Build}: Compilação dos binários Go e verificação de erros;
    \item \textbf{Test}: Execução de testes unitários e de integração com \textit{testcontainers};
    \item \textbf{Publish}: Construção da imagem Docker e envio para o Google Container Registry;
    \item \textbf{Deploy}: Atualização automática do serviço no Cloud Run.
\end{enumerate}

\subsubsection{Graceful Shutdown}

Em ambientes containerizados, o Cloud Run envia o sinal \texttt{SIGTERM} antes de encerrar um container --- seja por escalonamento, atualização de versão ou inatividade. Falhar em tratar esse sinal adequadamente pode causar perda de requisições em andamento, corrupção de dados parcialmente escritos e experiência degradada para o usuário final.

O sistema implementa o padrão de \textbf{graceful shutdown}: ao receber \texttt{SIGTERM}, a aplicação para de aceitar novas requisições, aguarda a conclusão das requisições em andamento (com timeout configurável de 30 segundos), fecha conexões com banco de dados e Redis de forma ordenada, e só então encerra o processo. Esta abordagem garante que nenhuma operação de ingestão ou consulta seja interrompida abruptamente.

\subsubsection{Otimização de Cold Start}

Plataformas \textit{serverless} como Cloud Run podem escalar a zero instâncias quando não há tráfego, economizando custos. Entretanto, a primeira requisição após um período de inatividade incorre em latência adicional conhecida como \textit{cold start}. Para mitigar este efeito, o sistema adota \textbf{inicialização preguiçosa} (\textit{lazy initialization}): conexões com banco de dados e Redis são estabelecidas apenas no primeiro uso, não durante a inicialização do container. Esta técnica reduz o tempo de \textit{cold start} ao adiar operações custosas até que sejam efetivamente necessárias.



\section{Requisitos}

A elicitação de requisitos seguiu as diretrizes da norma ISO/IEC/IEEE 29148 \cite{ieee29148}, que estabelece boas práticas para especificação de requisitos em projetos de software. Os requisitos foram organizados em duas categorias: funcionais (RF), que descrevem as funcionalidades do sistema, e não-funcionais (RNF), que definem atributos de qualidade.

Para a categorização dos requisitos não-funcionais, adotou-se o modelo de qualidade da ISO/IEC 25010 \cite{iso25010}, que define oito características de qualidade de software: funcionalidade, eficiência de desempenho, compatibilidade, usabilidade, confiabilidade, segurança, manutenibilidade e portabilidade. Essa estrutura permite uma cobertura sistemática dos atributos de qualidade esperados para a plataforma.

\subsection{Requisitos Funcionais}

Conforme destacado na análise de trabalhos relacionados (Seção 1.6), o \textit{Tô De Olho} sintetiza funcionalidades de plataformas consolidadas como \textit{De Olho em Você} e \textit{De Olho no Congresso}, adaptando-as ao contexto do Senado Federal. Os requisitos funcionais foram organizados em sete módulos:

\textbf{Módulo de Senadores:}
\begin{itemize}
    \item \textbf{[RF01]} O sistema deve apresentar a lista atualizada dos 81 senadores com foto, partido e estado.
    \item \textbf{[RF02]} O sistema deve permitir a busca de senadores por nome, sigla partidária ou UF.
    \item \textbf{[RF03]} O sistema deve exibir o perfil completo do senador com abas organizadas (Visão Geral, Gastos, Gabinete, Votações, Emendas).
\end{itemize}

\textbf{Módulo de Transparência Financeira (CEAPS):}
\begin{itemize}
    \item \textbf{[RF04]} O sistema deve importar os lançamentos da Cota Parlamentar (CEAPS) através das APIs de Dados Abertos do Senado.
    \item \textbf{[RF05]} O sistema deve permitir visualizar o gasto acumulado por tipo de despesa (passagens, correios, consultorias, combustível).
    \item \textbf{[RF06]} O sistema deve exibir os fornecedores que mais receberam recursos de um determinado senador.
    \item \textbf{[RF07]} O sistema deve gerar alertas automáticos para despesas atípicas: valores acima da média, pagamentos em finais de semana, intervalos menores que 3 dias entre pagamentos ao mesmo fornecedor.
\end{itemize}

\textbf{Módulo de Emendas e Orçamento:}
\begin{itemize}
    \item \textbf{[RF08]} O sistema deve integrar com o Portal da Transparência para buscar emendas de autoria do senador.
    \item \textbf{[RF09]} O sistema deve destacar valores destinados via ``Transferências Especiais'' (emendas PIX).
    \item \textbf{[RF10]} O sistema deve exibir mapas interativos de distribuição de emendas por município, permitindo identificar concentração geográfica de recursos.
\end{itemize}

\textbf{Módulo de Atividade Legislativa:}
\begin{itemize}
    \item \textbf{[RF11]} O sistema deve listar as votações nominais recentes e o voto de cada senador (Sim/Não/Abstenção).
    \item \textbf{[RF12]} O sistema deve exibir a participação do senador em comissões permanentes e especiais, incluindo cargo ocupado (titular, suplente, presidente, relator).
    \item \textbf{[RF13]} O sistema deve listar proposições de autoria do senador, com indicação do tipo (PEC, PLP, PL) e estágio de tramitação.
    \item \textbf{[RF14]} O sistema deve exibir discursos proferidos pelo senador em plenário, com data e tema.
    \item \textbf{[RF15]} O sistema deve exibir a agenda pública de reuniões de comissões, permitindo visualizar pautas futuras.
    \item \textbf{[RF16]} O sistema deve apresentar links para os perfis oficiais do senador em redes sociais (Twitter/X, Instagram, Facebook).
\end{itemize}

\textbf{Módulo de Gabinete:}
\begin{itemize}
    \item \textbf{[RF17]} O sistema deve exibir a lista de servidores do gabinete do senador, incluindo cargo e vínculo.
    \item \textbf{[RF18]} O sistema deve apresentar a folha de pagamento do gabinete, com remuneração mensal por servidor.
\end{itemize}

\textbf{Módulo de Comparação e Análise:}
\begin{itemize}
    \item \textbf{[RF19]} O sistema deve permitir comparar de 2 a 5 senadores lado a lado em múltiplas dimensões: despesas, emendas, votações e fornecedores em comum.
    \item \textbf{[RF20]} O sistema deve exibir ranking de fornecedores com cruzamento de sanções administrativas, identificando empresas com situação cadastral irregular.
    \item \textbf{[RF21]} O sistema deve mostrar indicadores de confiança: data da última sincronização, completude dos dados e fonte de cada informação.
\end{itemize}

\textbf{Módulo de Ranking e Score:}
\begin{itemize}
    \item \textbf{[RF22]} O sistema deve calcular e exibir o \textit{Score} de efetividade legislativa de cada senador, baseado no algoritmo de ranking do projeto.
    \item \textbf{[RF23]} O sistema deve apresentar gráfico radar com as quatro dimensões do \textit{Score}: Produtividade Legislativa, Presença em Votações, Economia na Cota e Participação em Comissões.
    \item \textbf{[RF24]} O sistema deve permitir ordenar e filtrar senadores por cada critério individual do ranking.
\end{itemize}

\subsection{Requisitos Não-Funcionais}

\textbf{Desempenho:}
\begin{itemize}
    \item \textbf{[RNF01]} O sistema deve responder a requisições de consulta em até 2 segundos sob condições normais de uso.
    \item \textbf{[RNF02]} A arquitetura deve suportar escalabilidade horizontal para lidar com picos de acesso em períodos eleitorais.
\end{itemize}

\textbf{Usabilidade e Acessibilidade:}
\begin{itemize}
    \item \textbf{[RNF03]} O sistema deve ser acessível via navegadores \textit{web} em dispositivos \textit{desktop} e \textit{mobile}.
    \item \textbf{[RNF04]} A interface deve seguir o padrão \textit{mobile-first} para garantir boa experiência em dispositivos móveis.
    \item \textbf{[RNF05]} O sistema deve seguir as diretrizes de acessibilidade WCAG 2.1 nível AA.
\end{itemize}

\textbf{Confiabilidade:}
\begin{itemize}
    \item \textbf{[RNF06]} Os dados devem ser sincronizados diariamente com as APIs oficiais do Senado Federal e Portal da Transparência.
    \item \textbf{[RNF07]} O sistema deve manter disponibilidade mínima de 99\% durante o período eleitoral.
\end{itemize}

\textbf{Segurança e Privacidade:}
\begin{itemize}
    \item \textbf{[RNF08]} As comunicações devem ser criptografadas utilizando HTTPS/TLS.
    \item \textbf{[RNF09]} O sistema deve estar em conformidade com a LGPD (Lei Geral de Proteção de Dados).
\end{itemize}

\textbf{Manutenibilidade:}
\begin{itemize}
    \item \textbf{[RNF10]} A arquitetura modular deve permitir atualizações isoladas de cada módulo sem impacto em outros componentes.
    \item \textbf{[RNF11]} O código deve seguir padrões de desenvolvimento (\textit{linting}, formatação) e estar documentado.
    \item \textbf{[RNF12]} O sistema deve possuir \textit{pipelines} de CI/CD para integração e deploy contínuos.
\end{itemize}

\section{Design}

O design de sistemas de informação voltados à fiscalização cidadã deve conciliar requisitos aparentemente conflitantes: robustez suficiente para processar grandes volumes de dados governamentais, simplicidade operacional compatível com equipes reduzidas, e acessibilidade que permita ao cidadão comum compreender informações complexas. Conforme o paradigma de Design Science Research adotado neste trabalho \cite{hevner2004design}, as decisões de design devem ser orientadas tanto pela utilidade prática do artefato quanto pelo rigor teórico que fundamenta cada escolha.

Esta seção organiza as decisões técnicas em quatro dimensões interdependentes: a arquitetura de software que estrutura o sistema, o stack tecnológico que a implementa, a estratégia de ingestão que alimenta a base de dados, e o modelo relacional que organiza as informações. Cada decisão é apresentada com suas alternativas consideradas e a justificativa para a escolha final, seguindo a tradição de documentação arquitetural proposta por Dragoni et al. \cite{dragoni2017microservices}.

\subsection{Arquitetura do Sistema}

A escolha arquitetural representa uma das decisões mais consequentes em projetos de software, com implicações que se estendem por todo o ciclo de vida do sistema. Em aplicações que integram múltiplas APIs externas --- como é o caso do \textit{Tô De Olho}, que consome dados de três órgãos governamentais distintos --- a arquitetura deve equilibrar a flexibilidade para evolução futura com a simplicidade operacional exigida pelo contexto acadêmico.

Após análise de alternativas, optou-se pelo padrão de \textbf{monolito modular}. Esta abordagem, sistematizada em revisão de literatura recente \cite{laigner2024modular}, combina a organização interna característica de microsserviços --- com fronteiras claras entre domínios de negócio --- com a simplicidade de implantação dos monolitos tradicionais. A decisão fundamentou-se em quatro critérios:

\begin{itemize}
    \item \textbf{Simplicidade Operacional}: Um único contêiner Docker elimina a complexidade de orquestração distribuída. Em plataformas serverless como Google Cloud Run, essa característica traduz-se em escala automática (inclusive a zero) e custos proporcionais ao uso efetivo;
    
    \item \textbf{Latência Previsível}: A comunicação entre módulos ocorre via chamadas de função em memória, eliminando a variabilidade de latência inerente a redes --- fator crítico para operações que agregam dados de múltiplos domínios;
    
    \item \textbf{Consistência Transacional}: Operações que envolvem múltiplos módulos compartilham uma única transação de banco de dados, garantindo atomicidade sem os patterns complexos (sagas, two-phase commit) exigidos por sistemas distribuídos;
    
    \item \textbf{Estratégia MonolithFirst}: Martin Fowler argumenta que ``quase todos os casos de sucesso de microsserviços começaram com um monolito que cresceu grande demais e foi decomposto'' \cite{fowler2015monolith}. A estrutura modular atual permite extração futura de serviços independentes apenas onde houver necessidade comprovada.
\end{itemize}

\subsubsection{Organização em Módulos}

A estrutura interna segue os princípios de \textit{Bounded Contexts} do Domain-Driven Design \cite{evans2003ddd}, onde cada módulo representa um contexto de domínio com responsabilidades claramente delimitadas. O diretório \texttt{internal/} organiza-se em cinco módulos principais, cada um encapsulando um aspecto da fiscalização parlamentar:

\begin{itemize}
    \item \texttt{internal/senador}: Gerencia dados cadastrais e mandatos históricos, atuando como ``fonte de verdade'' para identificação de parlamentares;
    \item \texttt{internal/ceaps}: Processa despesas da Cota Parlamentar, implementando detecção de anomalias e agregações por categoria;
    \item \texttt{internal/votacao}: Coleta votações nominais e calcula métricas como presença e fidelidade partidária;
    \item \texttt{internal/emenda}: Integra dados do Portal da Transparência, com tratamento especial para ``emendas PIX'';
    \item \texttt{internal/ranking}: Orquestra o cálculo de scores conforme metodologia definida na Seção~3.3.
\end{itemize}

Módulos transversais complementam a estrutura: \texttt{internal/api} expõe endpoints REST, \texttt{internal/scheduler} gerencia tarefas agendadas, e \texttt{internal/cache} abstrai operações no Redis. Os módulos comunicam-se exclusivamente através de interfaces Go, respeitando o princípio de inversão de dependência.

\subsubsection{Visão de Implantação}

A Figura~\ref{fig:implantacao} apresenta a arquitetura de implantação do sistema, ilustrando o fluxo de dados desde as fontes governamentais até o usuário final. O backend executa no Google Cloud Run, consumindo dados armazenados em PostgreSQL (Cloud SQL) e Redis (Memorystore). O frontend, desenvolvido em Next.js, é hospedado na Vercel para otimização de distribuição global.

\begin{figure}[!htb]
    \centering
    \includegraphics[width=0.7\textwidth]{docs/diagrama_implantacao.png}
    \caption{Arquitetura de implantação do sistema Tô De Olho}
    \label{fig:implantacao}
    \par\small{Fonte: Autoria Própria}
\end{figure}

\subsection{Stack Tecnológico}

A seleção de tecnologias foi orientada por três critérios: desempenho em cenários de alta concorrência (característicos de pipelines de ingestão), manutenibilidade a longo prazo (considerando a natureza open-source do projeto), e adequação ao domínio (priorizando ferramentas com suporte robusto a operações analíticas).

\subsubsection{Backend em Go}

A linguagem Go foi selecionada após avaliação comparativa com Node.js e Python. O fator decisivo foi o modelo de \textbf{goroutines}: threads leves gerenciadas pelo runtime, consumindo aproximadamente 2KB de memória inicial --- em contraste com cerca de 1MB por thread de sistema operacional \cite{nanz2015comparative}. Essa eficiência viabiliza o processamento paralelo de centenas de requisições HTTP durante a ingestão.

O framework \textbf{Gin} foi escolhido por seu roteamento baseado em radix tree, com desempenho até 40 vezes superior a alternativas \cite{alfian2024gin}. O ORM \textbf{GORM} oferece migrações automáticas e operações de upsert essenciais para garantia de idempotência. Para ambientes de produção, o sistema implementa connection pooling com limites configuráveis de conexões ociosas e máximas.

\subsubsection{Camada de Persistência}

\textbf{PostgreSQL} foi selecionado por sua robustez em consultas analíticas e suporte a CTEs (\textit{Common Table Expressions}), essenciais para agregações complexas. Originado em 1986 na UC Berkeley \cite{stonebraker1986postgres}, o banco garante conformidade ACID crítica para operações de ingestão concorrente.

\textbf{Redis} atua como cache para rankings pré-computados, alcançando latências de 100--500 microssegundos por operação \cite{redis2024}. Esta camada evita recálculos custosos a cada requisição, materializando resultados após cada ciclo de sincronização.

\subsubsection{Frontend em Next.js}

A escolha de \textbf{Next.js 15} justifica-se pela necessidade crítica de indexação por mecanismos de busca \cite{salim2024nextjs}. Diferentemente de SPAs que renderizam conteúdo apenas via JavaScript, Next.js oferece Server-Side Rendering (SSR) e Static Site Generation (SSG), entregando HTML pré-renderizado aos crawlers --- requisito fundamental para plataformas de fiscalização que dependem de descoberta orgânica.

A biblioteca \textbf{Recharts} foi selecionada para visualização de dados por sua integração declarativa com React. \textbf{Tailwind CSS 4} permite desenvolvimento ágil com classes utilitárias e geração otimizada de CSS em produção \cite{tailwindcss2024}.

\subsection{Estratégia de Ingestão de Dados}

A consolidação de dados dispersos em três APIs governamentais representa um dos maiores desafios técnicos do projeto. O processo segue o paradigma \textit{Extract, Transform, Load} (ETL), que conforme Vassiliadis \cite{vassiliadis2009survey} tipicamente consome entre 60\% e 80\% do esforço total em projetos de data warehousing.

A estratégia adotada combina duas abordagens complementares: \textbf{backfill} para carga histórica inicial e \textbf{sincronização contínua} para manutenção incremental. Conforme Kimball e Ross \cite{kimball2013data}, operações de carga histórica devem privilegiar throughput sobre latência, utilizando arquivos CSV consolidados e bulk inserts. As fontes históricas incluem CEAPS (2008--2025, ~500 mil registros), votações (2019--presente) e emendas (2015--presente).

A sincronização contínua opera via polling periódico, calibrado pela volatilidade de cada domínio: votações diárias às 03:00 BRT (baixo tráfego nas APIs), remunerações semanais aos domingos, e emendas mensalmente após fechamento contábil.

Para garantir consistência, o sistema opera sob modelo \textit{at-least-once} com \textbf{idempotência} via chaves naturais compostas \cite{hummer2013modeldriven}. Falhas transientes são tratadas com retry exponencial ($t_{wait} = base \times 2^{tentativa}$) conforme recomendação AWS \cite{aws2015backoff}, timeout de 30 segundos e falha graciosa que permite continuidade da sincronização mesmo com fontes temporariamente indisponíveis.

\subsection{Modelo de Dados}

O modelo relacional constitui o alicerce sobre o qual todas as funcionalidades são construídas. O projeto priorizou dois objetivos: garantir integridade referencial durante ingestão concorrente, e otimizar consultas analíticas frequentes como agregações de gastos por senador, período e categoria.

A Figura~\ref{fig:er-diagram} apresenta o diagrama entidade-relacionamento.

\begin{figure}[!htb]
\centering
\includegraphics[width=0.7\textwidth]{docs/diagrama_er_modelo.png}
\caption{Diagrama entidade-relacionamento do sistema Tô De Olho}
\label{fig:er-diagram}
\par\small{Fonte: Autoria Própria}
\end{figure}

As entidades organizam-se em torno de \textbf{Senador}, a entidade central identificada por código parlamentar único. Relacionamentos 1:N conectam o senador a: \textbf{Mandato} (legislatura, tipo, período), \textbf{Despesa CEAPS} (com chave composta para idempotência), \textbf{Votação} (tabela associativa com sessão), \textbf{Servidor de Gabinete} (nome, cargo, remuneração), \textbf{Emenda} (incluindo Transferências Especiais) e \textbf{Comissão Membro} (participação com cargo e período).

Para otimizar consultas frequentes, foram criados os seguintes índices compostos:

\begin{itemize}
    \item \texttt{idx\_despesa\_senador\_ano} --- totalização de gastos por período;
    \item \texttt{idx\_votacao\_sessao\_senador} --- cálculo de presença em votações;
    \item \texttt{idx\_emenda\_senador\_tipo} --- filtros por modalidade de emenda;
    \item \texttt{idx\_comissao\_senador} --- agregação de participação em comissões.
\end{itemize}

\section{Testes de Software}

\subsection{Projeto de Testes}
A estratégia de qualidade combina testes em diferentes níveis, aproveitando o ferramental nativo de Go:

\begin{itemize}
    \item \textbf{Testes Unitários}: Validam regras de negócio isoladas, como o cálculo do ``Score'' e parsers de CSV. Implementados com o pacote padrão \texttt{testing}, utilizando \textit{Table-Driven Tests} para cobrir múltiplos cenários.
    \item \textbf{Testes de Integração}: Validam a comunicação entre componentes e banco de dados. Utiliza-se \texttt{testcontainers-go} para subir instâncias efêmeras do PostgreSQL e Redis.
    \item \textbf{Testes de Contrato}: Asseguram que as respostas dos serviços externos continuam respeitando os formatos esperados.
\end{itemize}

\section{Implantação}

A infraestrutura foi projetada seguindo os princípios da metodologia \textit{Twelve-Factor App} \cite{wiggins2017twelvefactor}, que prioriza portabilidade, escalabilidade e manutenibilidade.

\subsection{Containerização com Docker}

Todos os componentes são containerizados utilizando \textbf{\textit{multi-stage builds}}, técnica recomendada pelas melhores práticas do Google Cloud \cite{googlecontainers2024}. O processo separa o ambiente de compilação (imagem \texttt{golang:1.21-alpine}) do ambiente de execução (imagem mínima \texttt{gcr.io/distroless/static-debian12}), resultando em imagens de 15--25MB com superfície de ataque reduzida.

\subsection{Pipeline de CI/CD}

O pipeline foi implementado com \textbf{GitHub Actions}. Estudos empíricos demonstram que a adoção de GitHub Actions reduz significativamente o tempo de feedback e aumenta a frequência de deploys \cite{kinsman2021software}. O pipeline automatiza quatro etapas:

\begin{enumerate}
    \item \textbf{Build}: Compilação com flags de otimização (\texttt{-ldflags="-w -s"}) e verificação estática via \texttt{go vet};
    \item \textbf{Test}: Testes unitários e de integração com cobertura mínima de 80\%;
    \item \textbf{Publish}: Construção da imagem Docker e envio para o Google Container Registry;
    \item \textbf{Deploy}: Atualização automática do serviço no Cloud Run via \textit{rolling update}.
\end{enumerate}

\subsection{Plataforma Google Cloud Run}

A implantação ocorre via \textbf{Google Cloud Run} \cite{cloudrun2024}, plataforma serverless para containers. Características que justificaram a escolha: escala automática (de zero a centenas de containers), deploy simplificado (único comando \texttt{gcloud run deploy}) e integração nativa com Cloud SQL e Memorystore.

\subsection{Graceful Shutdown}

O Cloud Run envia o sinal \texttt{SIGTERM} antes de encerrar um container. Falhar em tratar este sinal pode causar perda de requisições e corrupção de transações \cite{nygard2018release}. O sistema implementa uma sequência ordenada: (1) parada de aceitação de novas conexões, (2) drenagem de requisições em andamento com timeout de 30 segundos, (3) fechamento ordenado de conexões com PostgreSQL e Redis, (4) encerramento do processo.

\subsection{Mitigação de Cold Start}

Plataformas serverless podem escalar a zero, mas a primeira requisição após inatividade incorre em latência adicional (\textit{cold start}) \cite{silva2020serverless}. O sistema adota: imagens mínimas (distroless), inicialização preguiçosa de conexões (\textit{lazy initialization}), binários estáticos Go e health checks leves (\texttt{/health} retorna imediatamente).

\section{Manual do Usuário Simplificado}

A plataforma é pública e não requer cadastro para consulta, maximizando a transparência.

\begin{enumerate}
    \item \textbf{Acesso}: Navegue para \url{https://todeolho.org}.
    \item \textbf{Ranking}: Na página inicial, visualize os ``Top 3'' senadores em Economia e Presença.
    \item \textbf{Busca}: Utilize a barra superior para digitar o nome de um senador.
    \item \textbf{Detalhes}: Ao clicar em um senador, navegue pelas abas ``Gastos'' (para ver notas fiscais detalhadas) e ``Emendas'' (para ver destino das verbas).
    \item \textbf{Fiscalização}: Use o botão ``Compartilhar'' para enviar ficha do senador nas redes sociais.
\end{enumerate}

\section{Considerações Finais}

Este trabalho apresentou o \textit{Tô De Olho}, uma plataforma \textit{web} que centraliza e democratiza o acesso aos dados do Senado Federal. A arquitetura de \textbf{monolito modular} em Go, combinada com a ingestão de dados via APIs oficiais, mostrou-se adequada para consolidar informações dispersas em três fontes distintas. O \textit{front-end} em Next.js oferece ao cidadão uma interface acessível para fiscalizar despesas da CEAPS, acompanhar votações nominais e avaliar o desempenho dos 81 senadores por meio de um ranking com metodologia transparente.

Uma limitação relevante deste trabalho é a ausência do módulo de fórum para debate cívico, inicialmente planejado mas não implementado devido a restrições de prazo. Conforme aponta Costa \cite{duarte2019}, a gestão pública digital efetiva requer não apenas transparência, mas também canais de participação direta. O \textit{Tô De Olho} avança significativamente na dimensão da \textit{accountability}, porém ainda não contempla espaços deliberativos.

Para trabalhos futuros, sugere-se: (1) a implementação do Fórum de Cidadania para debate qualificado; (2) a expansão do escopo para incluir a Câmara dos Deputados, tornando a plataforma bicameral; e (3) a aplicação de técnicas de aprendizado de máquina para detecção de padrões anômalos em despesas parlamentares, seguindo o exemplo da Operação Serenata de Amor. Essas evoluções fortaleceriam o controle social e aproximariam a ferramenta dos degraus superiores da Escada de Arnstein. Por fim, ressalta-se o potencial impacto da ferramenta no pleito de 2026. Ao fornecer um histórico estruturado da atuação parlamentar, o \textit{Tô De Olho} pode qualificar o debate público e auxiliar eleitores em suas decisões de voto.


\section*{Agradecimentos}
Agradeço ao meu orientador, Prof. Pablo Vieira Florentino, pela orientação e apoio durante o desenvolvimento deste trabalho. Aos colegas de curso que contribuíram com discussões e sugestões. À minha família pelo apoio incondicional. Por fim, às comunidades de código aberto que mantiveram as ferramentas e documentações utilizadas neste projeto.

\bibliography{referencias}

\newpage
\section*{Apêndice A --- Glossário, Siglas e Abreviações}

\begin{description}
    \item[ACID] \textit{Atomicity, Consistency, Isolation, Durability} --- propriedades que garantem integridade em transações de banco de dados.
    \item[ADPF] Arguição de Descumprimento de Preceito Fundamental --- ação judicial perante o STF para questionar violações à Constituição.
    \item[API] \textit{Application Programming Interface} --- interface que permite comunicação entre sistemas.
    \item[CEAP] Cota para o Exercício da Atividade Parlamentar --- verba destinada aos Deputados Federais (equivalente à CEAPS para senadores).
    \item[CEAPS] Cota para o Exercício da Atividade Parlamentar dos Senadores --- verba destinada ao custeio de despesas relacionadas ao exercício do mandato parlamentar.
    \item[CGU] Controladoria-Geral da União --- órgão responsável pelo controle interno do Poder Executivo Federal e pela transparência pública.
    \item[CI/CD] \textit{Continuous Integration / Continuous Deployment} --- práticas de integração e implantação contínuas de software.
    \item[CSR] \textit{Client-Side Rendering} --- renderização de páginas no navegador do usuário.
    \item[CSV] \textit{Comma-Separated Values} --- formato de arquivo de texto para dados tabulares.
    \item[DVL] \textit{Data Visualization Literacy} --- literacia em visualização de dados; capacidade de interpretar representações visuais de informações.
    \item[ETL] \textit{Extract, Transform, Load} --- processo de extração, transformação e carga de dados.
    \item[FCP] \textit{First Contentful Paint} --- métrica de performance web que mede o tempo até o primeiro conteúdo visível.
    \item[GCP] \textit{Google Cloud Platform} --- plataforma de serviços em nuvem do Google.
    \item[INAF] Indicador de Alfabetismo Funcional --- pesquisa que avalia níveis de alfabetização no Brasil.
    \item[JIT] \textit{Just-In-Time} --- compilação sob demanda; no contexto de CSS, geração de estilos apenas quando utilizados.
    \item[JSON] \textit{JavaScript Object Notation} --- formato leve de intercâmbio de dados.
    \item[LAI] Lei de Acesso à Informação (Lei nº 12.527/2011) --- legislação que garante o direito de acesso a informações públicas.
    \item[LCP] \textit{Largest Contentful Paint} --- métrica de performance web que mede o tempo até o maior elemento visível.
    \item[LES] \textit{Legislative Effectiveness Score} --- metodologia para avaliação de efetividade legislativa desenvolvida por Volden e Wiseman.
    \item[LGPD] Lei Geral de Proteção de Dados (Lei nº 13.709/2018) --- legislação brasileira sobre privacidade e proteção de dados pessoais.
    \item[MVP] \textit{Minimum Viable Product} --- produto mínimo viável, versão inicial com funcionalidades essenciais.
    \item[OGP] \textit{Open Government Partnership} --- Parceria para Governo Aberto; iniciativa multilateral para promoção da transparência.
    \item[ORM] \textit{Object-Relational Mapping} --- técnica de mapeamento objeto-relacional para persistência de dados.
    \item[PAC] \textit{Political Action Committee} --- comitê de ação política nos EUA para arrecadação de fundos eleitorais.
    \item[PEC] Proposta de Emenda à Constituição --- visa alterar a Constituição; exige quórum de 3/5 em duas votações.
    \item[PIX] Sistema de pagamentos instantâneos do Banco Central; no contexto parlamentar, refere-se às ``Emendas PIX'' (Transferências Especiais) --- modalidade de repasse que dispensa convênio.
    \item[PLP] Projeto de Lei Complementar --- proposição que regulamenta matérias específicas previstas na Constituição; exige maioria absoluta para aprovação.
    \item[RDF] \textit{Resource Description Framework} --- modelo de dados para representação de informações na web semântica.
    \item[REST] \textit{Representational State Transfer} --- estilo arquitetural para APIs \textit{web}.
    \item[SEO] \textit{Search Engine Optimization} --- otimização para motores de busca.
    \item[SIAFI] Sistema Integrado de Administração Financeira do Governo Federal --- sistema de contabilidade pública do governo brasileiro.
    \item[SLES] \textit{State Legislative Effectiveness Score} --- versão estadual do LES, adaptada para legislaturas subnacionais.
    \item[SSG] \textit{Static Site Generation} --- geração estática de páginas web em tempo de build.
    \item[SSR] \textit{Server-Side Rendering} --- renderização de páginas no servidor.
    \item[STF] Supremo Tribunal Federal --- órgão máximo do Poder Judiciário brasileiro.
    \item[TIC] Tecnologias da Informação e Comunicação.
    \item[TLS] \textit{Transport Layer Security} --- protocolo de segurança para comunicações criptografadas na internet.
    \item[TTL] \textit{Time To Live} --- tempo de vida de dados em cache.
    \item[UF] Unidade Federativa --- estado brasileiro.
    \item[URI] \textit{Uniform Resource Identifier} --- identificador único de recursos na web.
    \item[W3C] \textit{World Wide Web Consortium} --- organização internacional que desenvolve padrões para a web.
    \item[WCAG] \textit{Web Content Accessibility Guidelines} --- diretrizes de acessibilidade para conteúdo \textit{web}.
    \item[XML] \textit{eXtensible Markup Language} --- linguagem de marcação para estruturação de dados.
\end{description}


\end{document}
