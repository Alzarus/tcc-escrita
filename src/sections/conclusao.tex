\section{Considerações Finais}

O projeto \textit{Tô De Olho} atingiu seu objetivo principal de desenvolver uma plataforma \textit{web} capaz de centralizar e democratizar o acesso aos dados do Senado Federal. A arquitetura de \textit{microservices} em Golang, aliada a uma estratégia híbrida de ingestão de dados (API do Senado, Crawlers e Portal da Transparência), provou-se robusta para lidar com a complexidade e a dispersão das informações. A interface desenvolvida em Next.js priorizou a usabilidade e a acessibilidade, oferecendo ao cidadão ferramentas intuitivas para fiscalizar despesas da CEAPS, acompanhar votações e conhecer o perfil de seus 81 representantes.

Entretanto, é importante destacar uma limitação no escopo final da implementação. Inicialmente, planejou-se um módulo de fórum para debate cívico validado. Devido a restrições de tempo hábil para desenvolvimento e testes de segurança rigorosos, este recurso não foi incluído na versão atual. Conforme aponta Costa, a gestão pública digital apoia-se em pilares que incluem não apenas a transparência e a prestação de contas, mas também a participação direta \cite{duarte2019}. Embora o \textit{Tô De Olho} avance significativamente na transparência ativa e na \textit{accountability}, a ausência do fórum limita, neste momento, a dimensão da participação deliberativa direta na plataforma.

Como trabalhos futuros, sugere-se a implementação deste espaço de discussão (Fórum de Cidadania), bem como a expansão do escopo para incluir dados da Câmara dos Deputados (tornando o sistema bicameral) e a aplicação de técnicas de aprendizado de máquina para identificar padrões anômalos de gastos na cota parlamentar, fortalecendo ainda mais o controle social.
