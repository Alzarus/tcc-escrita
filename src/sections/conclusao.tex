\section{Considerações Finais}

Este trabalho apresentou o \textit{Tô De Olho}, uma plataforma \textit{web} que centraliza e democratiza o acesso aos dados do Senado Federal. A arquitetura de \textbf{monolito modular} em Go, combinada com a ingestão de dados via APIs oficiais, mostrou-se adequada para consolidar informações dispersas em três fontes distintas. O \textit{front-end} em Next.js oferece ao cidadão uma interface acessível para fiscalizar despesas da CEAPS, acompanhar votações nominais e avaliar o desempenho dos 81 senadores por meio de um ranking com metodologia transparente.

Uma limitação relevante deste trabalho é a ausência do módulo de fórum para debate cívico, inicialmente planejado mas não implementado devido a restrições de prazo. Conforme aponta Costa \cite{duarte2019}, a gestão pública digital efetiva requer não apenas transparência, mas também canais de participação direta. O \textit{Tô De Olho} avança significativamente na dimensão da \textit{accountability}, porém ainda não contempla espaços deliberativos.

Para trabalhos futuros, sugere-se: (1) a implementação do Fórum de Cidadania para debate qualificado; (2) a expansão do escopo para incluir a Câmara dos Deputados, tornando a plataforma bicameral; e (3) a aplicação de técnicas de aprendizado de máquina para detecção de padrões anômalos em despesas parlamentares, seguindo o exemplo da Operação Serenata de Amor. Essas evoluções fortaleceriam o controle social e aproximariam a ferramenta dos degraus superiores da Escada de Arnstein. Por fim, ressalta-se o potencial impacto da ferramenta no pleito de 2026. Ao fornecer um histórico estruturado da atuação parlamentar, o \textit{Tô De Olho} pode qualificar o debate público e auxiliar eleitores em suas decisões de voto.
