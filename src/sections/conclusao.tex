\section{Considerações Finais}

Este trabalho apresentou o \textit{Tô De Olho}, uma plataforma \textit{web} que centraliza e democratiza o acesso aos dados do Senado Federal. A arquitetura de \textbf{monolito modular} em \textit{Go}, combinada com a ingestão de dados via APIs oficiais, mostrou-se adequada para consolidar informações dispersas em três fontes distintas. O \textit{front-end} em \textit{Next.js} oferece ao cidadão uma interface acessível para fiscalizar despesas da CEAPS, acompanhar votações nominais e avaliar o desempenho dos 81 senadores por meio de um \textit{ranking} com metodologia transparente.


Uma limitação relevante deste trabalho é a ausência do módulo de fórum para debate cívico, inicialmente planejado mas não implementado devido a restrições de prazo. Conforme aponta Costa (\citeyear{duarte2019}), a gestão pública digital efetiva requer não apenas transparência, mas também canais de participação direta. O \textit{Tô De Olho} avança significativamente na dimensão da prestação de contas, porém ainda não contempla espaços deliberativos.

A principal barreira encontrada durante o desenvolvimento, contudo, transcendeu os desafios técnicos de ingestão de dados. A tradução da complexidade do Regimento Interno do Senado em regras de negócio determinísticas revelou-se um exercício de contínuo compromisso. Simplificar a atuação parlamentar --- intrinsecamente política e negociada --- em métricas quantitativas impôs reducionismos necessários ao modelo de dados, mas que devem ser interpretados com cautela. A experiência reforçou que ferramentas de tecnologia cívica são meios, e não fins; elas iluminam dados, mas a interpretação do contexto permanece uma atribuição insubstituível do cidadão.

Um aspecto transversal ao desenvolvimento merece registro: o uso sistemático de assistentes de inteligência artificial como ferramenta de produtividade. O projeto foi construído com o auxílio do \textit{Antigravity}, alimentado pelos modelos \textit{Claude Opus 4.5} e \textit{Gemini 3 Pro}, empregados primariamente em tarefas de implementação e depuração. A eficácia dessa abordagem, contudo, não foi automática: dependeu de investimento deliberado em engenharia de \textit{prompt} --- arquivos de contexto mantidos pelo autor com restrições arquiteturais, boas práticas e decisões de projeto funcionaram como ``memória persistente'' para os modelos. Além disso, foram empregadas estratégias sistemáticas de \textit{prompting}, como encadeamento de raciocínio (\textit{chain of thought}), exemplos dirigidos (\textit{few-shot}), questionamento maiêutico (\textit{maieutic prompting}) e árvore de pensamentos (\textit{tree of thoughts}).

Esse fluxo de trabalho possibilitou que as aproximadamente 14.300 linhas de código que constituem a plataforma --- abrangendo nove módulos de \textit{back-end}, sete rotas de \textit{front-end}, \textit{pipelines} ETL para três APIs externas, testes de contrato e infraestrutura de implantação --- fossem produzidas em 22 dias de desenvolvimento. A estimativa para o mesmo escopo sem assistência de IA, considerando a complexidade do domínio legislativo, a engenharia reversa de APIs com formatos heterogêneos e os requisitos de acessibilidade, situa-se entre sete e nove meses para um desenvolvedor solo em regime acadêmico --- uma redução de prazo de aproximadamente 70\%.

O processo revelou, entretanto, que a IA funcionou como amplificadora de capacidade técnica, não como substituta do julgamento de engenharia. As decisões de arquitetura, modelagem de domínio e estratégia de dados permaneceram integralmente sob responsabilidade do autor. Alucinações dos modelos --- código sintaticamente válido mas semanticamente incorreto --- exigiram vigilância constante e intervenções manuais, reafirmando a necessidade de domínio conceitual sólido por parte do desenvolvedor para direcionar e validar as saídas geradas.

Para trabalhos futuros, almeja-se a implementação das funcionalidades mapeadas nos Requisitos mas não contempladas no MVP, a saber:
\begin{enumerate}
    \item \textbf{Módulo de Gabinete (RF17, RF18)}: Visualização da lista de servidores comissionados e folha de pagamento detalhada;
    \item \textbf{Módulo de Fornecedores Suplementar (RF06, RF07, RF20)}: Classificação de recebedores, alertas de despesas atípicas e cruzamento com sanções administrativas;
    \item \textbf{Atividade Legislativa Expandida (RF14-RF16)}: Inclusão de discursos em plenário, agenda de comissões e integração com redes sociais oficiais;
    \item \textbf{Normalização Temporal}: Implementação de uma normalização por ``meses de mandato ativo'', permitindo comparações mais justas entre legislaturas completas.
\end{enumerate}

Além disso, sugere-se: (1) a expansão do escopo para incluir a Câmara dos Deputados, tornando a plataforma bicameral; e (2) a aplicação de técnicas de aprendizado de máquina para detecção de padrões anômalos. Essas evoluções fortaleceriam o controle social e aproximariam a ferramenta dos degraus superiores da Escada de Arnstein. Por fim, ressalta-se o potencial impacto da ferramenta no pleito de 2026. Ao fornecer um histórico estruturado da atuação parlamentar, o \textit{Tô De Olho} pode qualificar o debate público e auxiliar eleitores em suas decisões de voto.
