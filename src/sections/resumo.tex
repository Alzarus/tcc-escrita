\clearpage
\thispagestyle{empty}
\begin{center}
    \section*{Resumo}
\end{center}

O presente trabalho propõe o desenvolvimento do \textit{Tô De Olho}, uma plataforma \textit{web} de transparência parlamentar focada no Senado Federal. O Senado, apesar de sua relevância legislativa, carece de ferramentas de fiscalização cidadã integradas que facilitem o controle social, contrastando com o ecossistema consolidado para a Câmara dos Deputados. A solução adota uma arquitetura de monolito modular em linguagem \textit{Go}, com interface implementada em \textit{Next.js}, e integra dados de três APIs governamentais (Senado Legislativo, Senado Administrativo e Portal da Transparência) para consolidar informações sobre atividade legislativa, execução da Cota para o Exercício da Atividade Parlamentar (CEAPS) e alocação de emendas parlamentares. Como diferencial metodológico, apresenta-se um Índice de Efetividade Legislativa adaptado ao contexto brasileiro, que pondera produtividade, presença em plenário, economia de recursos e participação em comissões. Os resultados demonstram a viabilidade técnica da unificação de fontes de dados heterogêneas e o potencial da plataforma em reduzir a assimetria informacional, transformando transparência passiva em prestação de contas ativa e qualificando o debate público para o cenário eleitoral de 2026. A plataforma adota premissas de acessibilidade (WCAG) e design \textit{mobile-first} para democratizar o acesso conforme o perfil da população brasileira.

\vspace{1cm}

\noindent
\textbf{Palavras-chaves}: Transparência Pública. Senado Federal. Dados Abertos. Accountability. Engenharia de Software.
