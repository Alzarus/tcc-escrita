\section{Design}

O design de sistemas de informação voltados à fiscalização cidadã deve conciliar requisitos aparentemente conflitantes: robustez suficiente para processar grandes volumes de dados governamentais, simplicidade operacional compatível com equipes reduzidas, e acessibilidade que permita ao cidadão comum compreender informações complexas. Conforme o paradigma de Design Science Research adotado neste trabalho \cite{hevner2004design}, as decisões de design devem ser orientadas tanto pela utilidade prática do artefato quanto pelo rigor teórico que fundamenta cada escolha.

Esta seção organiza as decisões técnicas em quatro dimensões interdependentes: a arquitetura de software que estrutura o sistema, o stack tecnológico que a implementa, a estratégia de ingestão que alimenta a base de dados, e o modelo relacional que organiza as informações. Cada decisão é apresentada com suas alternativas consideradas e a justificativa para a escolha final, seguindo a tradição de documentação arquitetural proposta por Dragoni et al. \cite{dragoni2017microservices}.

\subsection{Arquitetura do Sistema}

A escolha arquitetural representa uma das decisões mais consequentes em projetos de software, com implicações que se estendem por todo o ciclo de vida do sistema. Em aplicações que integram múltiplas APIs externas --- como é o caso do \textit{Tô De Olho}, que consome dados de três órgãos governamentais distintos --- a arquitetura deve equilibrar a flexibilidade para evolução futura com a simplicidade operacional exigida pelo contexto acadêmico.

Após análise de alternativas, optou-se pelo padrão de \textbf{monolito modular}. Esta abordagem, sistematizada em revisão de literatura recente \cite{laigner2024modular}, combina a organização interna característica de microsserviços --- com fronteiras claras entre domínios de negócio --- com a simplicidade de implantação dos monolitos tradicionais. A decisão fundamentou-se em quatro critérios:

\begin{itemize}
    \item \textbf{Simplicidade Operacional}: Um único contêiner Docker elimina a complexidade de orquestração distribuída. Em plataformas serverless como Google Cloud Run, essa característica traduz-se em escala automática (inclusive a zero) e custos proporcionais ao uso efetivo;
    
    \item \textbf{Latência Previsível}: A comunicação entre módulos ocorre via chamadas de função em memória, eliminando a variabilidade de latência inerente a redes --- fator crítico para operações que agregam dados de múltiplos domínios;
    
    \item \textbf{Consistência Transacional}: Operações que envolvem múltiplos módulos compartilham uma única transação de banco de dados, garantindo atomicidade sem os patterns complexos (sagas, two-phase commit) exigidos por sistemas distribuídos;
    
    \item \textbf{Estratégia MonolithFirst}: Martin Fowler argumenta que ``quase todos os casos de sucesso de microsserviços começaram com um monolito que cresceu grande demais e foi decomposto'' \cite{fowler2015monolith}. A estrutura modular atual permite extração futura de serviços independentes apenas onde houver necessidade comprovada.
\end{itemize}

\subsubsection{Organização em Módulos}

A estrutura interna segue os princípios de \textit{Bounded Contexts} do Domain-Driven Design \cite{evans2003ddd}, onde cada módulo representa um contexto de domínio com responsabilidades claramente delimitadas. O diretório \texttt{internal/} organiza-se em cinco módulos principais, cada um encapsulando um aspecto da fiscalização parlamentar:

\begin{itemize}
    \item \texttt{internal/senador}: Gerencia dados cadastrais e mandatos históricos, atuando como ``fonte de verdade'' para identificação de parlamentares;
    \item \texttt{internal/ceaps}: Processa despesas da Cota Parlamentar, implementando detecção de anomalias e agregações por categoria;
    \item \texttt{internal/votacao}: Coleta votações nominais e calcula métricas como presença e fidelidade partidária;
    \item \texttt{internal/emenda}: Integra dados do Portal da Transparência, com tratamento especial para ``emendas PIX'';
    \item \texttt{internal/ranking}: Orquestra o cálculo de scores conforme metodologia definida na Seção~3.3.
\end{itemize}

Módulos transversais complementam a estrutura: \texttt{internal/api} expõe endpoints REST, \texttt{internal/scheduler} gerencia tarefas agendadas, e \texttt{internal/cache} abstrai operações no Redis. Os módulos comunicam-se exclusivamente através de interfaces Go, respeitando o princípio de inversão de dependência.

\subsubsection{Visão de Implantação}

A Figura~\ref{fig:implantacao} apresenta a arquitetura de implantação do sistema, ilustrando o fluxo de dados desde as fontes governamentais até o usuário final. O backend executa no Google Cloud Run, consumindo dados armazenados em PostgreSQL (Cloud SQL) e Redis (Memorystore). O frontend, desenvolvido em Next.js, é hospedado na Vercel para otimização de distribuição global.

\begin{figure}[!htb]
    \centering
    \includegraphics[width=0.7\textwidth]{docs/diagrama_implantacao.png}
    \caption{Arquitetura de implantação do sistema Tô De Olho}
    \label{fig:implantacao}
    \par\small{Fonte: Autoria Própria}
\end{figure}

\subsection{Stack Tecnológico}

A seleção de tecnologias foi orientada por três critérios: desempenho em cenários de alta concorrência (característicos de pipelines de ingestão), manutenibilidade a longo prazo (considerando a natureza open-source do projeto), e adequação ao domínio (priorizando ferramentas com suporte robusto a operações analíticas).

\subsubsection{Backend em Go}

A linguagem Go foi selecionada após avaliação comparativa com Node.js e Python. O fator decisivo foi o modelo de \textbf{goroutines}: threads leves gerenciadas pelo runtime, consumindo aproximadamente 2KB de memória inicial --- em contraste com cerca de 1MB por thread de sistema operacional \cite{nanz2015comparative}. Essa eficiência viabiliza o processamento paralelo de centenas de requisições HTTP durante a ingestão.

O framework \textbf{Gin} foi escolhido por seu roteamento baseado em radix tree, com desempenho até 40 vezes superior a alternativas \cite{alfian2024gin}. O ORM \textbf{GORM} oferece migrações automáticas e operações de upsert essenciais para garantia de idempotência. Para ambientes de produção, o sistema implementa connection pooling com limites configuráveis de conexões ociosas e máximas.

\subsubsection{Camada de Persistência}

\textbf{PostgreSQL} foi selecionado por sua robustez em consultas analíticas e suporte a CTEs (\textit{Common Table Expressions}), essenciais para agregações complexas. Originado em 1986 na UC Berkeley \cite{stonebraker1986postgres}, o banco garante conformidade ACID crítica para operações de ingestão concorrente.

\textbf{Redis} atua como cache para rankings pré-computados, alcançando latências de 100--500 microssegundos por operação \cite{redis2024}. Esta camada evita recálculos custosos a cada requisição, materializando resultados após cada ciclo de sincronização.

\subsubsection{Frontend em Next.js}

A escolha de \textbf{Next.js 15} justifica-se pela necessidade crítica de indexação por mecanismos de busca \cite{salim2024nextjs}. Diferentemente de SPAs que renderizam conteúdo apenas via JavaScript, Next.js oferece Server-Side Rendering (SSR) e Static Site Generation (SSG), entregando HTML pré-renderizado aos crawlers --- requisito fundamental para plataformas de fiscalização que dependem de descoberta orgânica.

A biblioteca \textbf{Recharts} foi selecionada para visualização de dados por sua integração declarativa com React. \textbf{Tailwind CSS 4} permite desenvolvimento ágil com classes utilitárias e geração otimizada de CSS em produção \cite{tailwindcss2024}.

\subsection{Estratégia de Ingestão de Dados}

A consolidação de dados dispersos em três APIs governamentais representa um dos maiores desafios técnicos do projeto. O processo segue o paradigma \textit{Extract, Transform, Load} (ETL), que conforme Vassiliadis \cite{vassiliadis2009survey} tipicamente consome entre 60\% e 80\% do esforço total em projetos de data warehousing.

A estratégia adotada combina duas abordagens complementares: \textbf{backfill} para carga histórica inicial e \textbf{sincronização contínua} para manutenção incremental. Conforme Kimball e Ross \cite{kimball2013data}, operações de carga histórica devem privilegiar throughput sobre latência, utilizando arquivos CSV consolidados e bulk inserts. As fontes históricas incluem CEAPS (2008--2025, ~500 mil registros), votações (2019--presente) e emendas (2015--presente).

A sincronização contínua opera via polling periódico, calibrado pela volatilidade de cada domínio: votações diárias às 03:00 BRT (baixo tráfego nas APIs), remunerações semanais aos domingos, e emendas mensalmente após fechamento contábil.

Para garantir consistência, o sistema opera sob modelo \textit{at-least-once} com \textbf{idempotência} via chaves naturais compostas \cite{hummer2013modeldriven}. Falhas transientes são tratadas com retry exponencial ($t_{wait} = base \times 2^{tentativa}$) conforme recomendação AWS \cite{aws2015backoff}, timeout de 30 segundos e falha graciosa que permite continuidade da sincronização mesmo com fontes temporariamente indisponíveis.

\subsection{Modelo de Dados}

O modelo relacional constitui o alicerce sobre o qual todas as funcionalidades são construídas. O projeto priorizou dois objetivos: garantir integridade referencial durante ingestão concorrente, e otimizar consultas analíticas frequentes como agregações de gastos por senador, período e categoria.

A Figura~\ref{fig:er-diagram} apresenta o diagrama entidade-relacionamento.

\begin{figure}[!htb]
\centering
\includegraphics[width=0.7\textwidth]{docs/diagrama_er_modelo.png}
\caption{Diagrama entidade-relacionamento do sistema Tô De Olho}
\label{fig:er-diagram}
\par\small{Fonte: Autoria Própria}
\end{figure}

As entidades organizam-se em torno de \textbf{Senador}, a entidade central identificada por código parlamentar único. Relacionamentos 1:N conectam o senador a: \textbf{Mandato} (legislatura, tipo, período), \textbf{Despesa CEAPS} (com chave composta para idempotência), \textbf{Votação} (tabela associativa com sessão), \textbf{Servidor de Gabinete} (nome, cargo, remuneração), \textbf{Emenda} (incluindo Transferências Especiais) e \textbf{Comissão Membro} (participação com cargo e período).

Para otimizar consultas frequentes, foram criados os seguintes índices compostos:

\begin{itemize}
    \item \texttt{idx\_despesa\_senador\_ano} --- totalização de gastos por período;
    \item \texttt{idx\_votacao\_sessao\_senador} --- cálculo de presença em votações;
    \item \texttt{idx\_emenda\_senador\_tipo} --- filtros por modalidade de emenda;
    \item \texttt{idx\_comissao\_senador} --- agregação de participação em comissões.
\end{itemize}
