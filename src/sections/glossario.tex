\section*{Apêndice A --- Glossário, Siglas e Abreviações}

\begin{description}[leftmargin=3.5cm, labelwidth=3cm, labelsep=0.5cm, itemsep=2pt, parsep=0pt, topsep=0pt]
    \item[ACID] \textit{Atomicity, Consistency, Isolation, Durability} --- propriedades que garantem integridade em transações de banco de dados.
    \item[ADPF] Arguição de Descumprimento de Preceito Fundamental --- ação judicial perante o STF para questionar violações à Constituição.
    \item[API] \textit{Application Programming Interface} --- interface que permite comunicação entre sistemas.
    \item[CAE] Comissão de Assuntos Econômicos --- comissão permanente do Senado Federal.
    \item[CCJ] Comissão de Constituição, Justiça e Cidadania --- comissão permanente do Senado Federal.
    \item[CEAP] Cota para o Exercício da Atividade Parlamentar --- verba destinada aos Deputados Federais (equivalente à CEAPS para senadores).
    \item[CEAPS] Cota para o Exercício da Atividade Parlamentar dos Senadores --- verba destinada ao custeio de despesas relacionadas ao exercício do mandato parlamentar.
    \item[CGU] Controladoria-Geral da União --- órgão responsável pelo controle interno do Poder Executivo Federal e pela transparência pública.
    \item[CI/CD] \textit{Continuous Integration / Continuous Deployment} --- práticas de integração e implantação contínuas de software.
    \item[CSS] \textit{Cascading Style Sheets} --- linguagem de folhas de estilo utilizada para definir a apresentação de documentos \textit{web}.
    \item[CSR] \textit{Client-Side Rendering} --- renderização de páginas no navegador do usuário.
    \item[CSV] \textit{Comma-Separated Values} --- formato de arquivo de texto para dados tabulares.
    \item[DSR] \textit{Design Science Research} --- abordagem metodológica de pesquisa focada na criação e avaliação de artefatos.
    \item[DVL] \textit{Data Visualization Literacy} --- literacia em visualização de dados; capacidade de interpretar representações visuais de informações.
    \item[ETL] \textit{Extract, Transform, Load} --- processo de extração, transformação e carga de dados.
    \item[FCP] \textit{First Contentful Paint} --- métrica de performance web que mede o tempo até o primeiro conteúdo visível.
    \item[GCP] \textit{Google Cloud Platform} --- plataforma de serviços em nuvem do Google.
    \item[HTML] \textit{HyperText Markup Language} --- linguagem de marcação padrão para documentos desenhados para serem exibidos em um navegador.
    \item[HTTPS] \textit{HyperText Transfer Protocol Secure} --- extensão segura do protocolo HTTP para comunicação em redes de computadores.
    \item[IDE] \textit{Integrated Development Environment} --- ambiente de desenvolvimento integrado; software que auxilia programadores no desenvolvimento de software.
    \item[INAF] Indicador de Alfabetismo Funcional --- pesquisa que avalia níveis de alfabetização no Brasil.
    \item[JIT] \textit{Just-In-Time} --- compilação sob demanda; no contexto de CSS, geração de estilos apenas quando utilizados.
    \item[JSON] \textit{JavaScript Object Notation} --- formato leve de intercâmbio de dados.
    \item[LAI] Lei de Acesso à Informação (Lei nº 12.527/2011) --- legislação que garante o direito de acesso a informações públicas.
    \item[LCP] \textit{Largest Contentful Paint} --- métrica de performance web que mede o tempo até o maior elemento visível.
    \item[LES] \textit{Legislative Effectiveness \textit{Score}} --- metodologia para avaliação de efetividade legislativa desenvolvida por Volden e Wiseman.
    \item[LGPD] Lei Geral de Proteção de Dados (Lei nº 13.709/2018) --- legislação brasileira sobre privacidade e proteção de dados pessoais.
    \item[LLM] \textit{Large Language Model} --- modelo de linguagem grande; tipo de inteligência artificial treinado em vastas quantidades de texto.
    \item[MVP] \textit{Minimum Viable Product} --- produto mínimo viável, versão inicial com funcionalidades essenciais.
    \item[OGP] \textit{Open Government Partnership} --- Parceria para Governo Aberto; iniciativa multilateral para promoção da transparência.
    \item[ORM] \textit{Object-Relational Mapping} --- técnica de mapeamento objeto-relacional para persistência de dados.
    \item[PAC] \textit{Political Action Committee} --- comitê de ação política nos EUA para arrecadação de fundos eleitorais.
    \item[PEC] Proposta de Emenda à Constituição --- visa alterar a Constituição; exige quórum de 3/5 em duas votações.
    \item[PIX] Sistema de pagamentos instantâneos do Banco Central; no contexto parlamentar, refere-se às ``Emendas PIX'' (Transferências Especiais).
    \item[PL] Projeto de Lei --- proposição destinada a dispor sobre matéria de competência normativa do Poder Legislativo.
    \item[PLP] Projeto de Lei Complementar --- proposição que regulamenta matérias específicas previstas na Constituição.
    \item[RDF] \textit{Resource Description \textit{Framework}} --- modelo de dados para representação de informações na web semântica.
    \item[REST] \textit{Representational State Transfer} --- estilo arquitetural para APIs \textit{web}.
    \item[SEO] \textit{Search Engine Optimization} --- otimização para motores de busca.
    \item[SHA] \textit{Secure Hash Algorithm} --- família de funções de hash criptográficas.
    \item[SIAFI] Sistema Integrado de Administração Financeira do Governo Federal --- sistema de contabilidade pública do governo brasileiro.
    \item[SLES] \textit{State Legislative Effectiveness \textit{Score}} --- versão estadual do LES, adaptada para legislaturas subnacionais.
    \item[SQL] \textit{Structured Query Language} --- linguagem de consulta estruturada; padrão para gerenciamento de bancos de dados relacionais.
    \item[SSG] \textit{Static Site Generation} --- geração estática de páginas web em tempo de \textit{build}.
    \item[SSR] \textit{Server-Side Rendering} --- renderização de páginas no servidor.
    \item[STF] Supremo Tribunal Federal --- órgão máximo do Poder Judiciário brasileiro.
    \item[TIC] Tecnologias da Informação e Comunicação.
    \item[TLS] \textit{Transport Layer Security} --- protocolo de segurança para comunicações criptografadas na internet.
    \item[TTL] \textit{Time To Live} --- tempo de vida de dados em cache.
    \item[UF] Unidade Federativa --- estado brasileiro.
    \item[URI] \textit{Uniform Resource Identifier} --- identificador único de recursos na web.
    \item[URL] \textit{Uniform Resource Locator} --- endereço de um recurso disponível em uma rede (como a Web).
    \item[W3C] \textit{World Wide Web Consortium} --- organização internacional que desenvolve padrões para a web.
    \item[WCAG] \textit{Web Content Accessibility Guidelines} --- diretrizes de acessibilidade para conteúdo \textit{web}.
    \item[XML] \textit{eXtensible Markup Language} --- linguagem de marcação para estruturação de dados.
\end{description}
