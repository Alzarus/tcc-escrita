\section{Requisitos}

A elicitação de requisitos seguiu as diretrizes da norma ISO/IEC/IEEE 29148 \cite{ieee29148}, que estabelece boas práticas para especificação de requisitos em projetos de software. Os requisitos foram organizados em duas categorias: funcionais (RF), que descrevem as funcionalidades do sistema, e não-funcionais (RNF), que definem atributos de qualidade.

Para a categorização dos requisitos não-funcionais, adotou-se o modelo de qualidade da ISO/IEC 25010 \cite{iso25010}, que define oito características de qualidade de software: funcionalidade, eficiência de desempenho, compatibilidade, usabilidade, confiabilidade, segurança, manutenibilidade e portabilidade. Essa estrutura permite uma cobertura sistemática dos atributos de qualidade esperados para a plataforma.

\subsection{Requisitos Funcionais}

Conforme destacado na análise de trabalhos relacionados (Seção 1.6), o \textit{Tô De Olho} sintetiza funcionalidades de plataformas consolidadas como \textit{De Olho em Você} e \textit{De Olho no Congresso}, adaptando-as ao contexto do Senado Federal. Os requisitos funcionais foram organizados em sete módulos:

\textbf{Módulo de Senadores:}
\begin{itemize}
    \item \textbf{[RF01]} O sistema deve apresentar a lista atualizada dos 81 senadores com foto, partido e estado.
    \item \textbf{[RF02]} O sistema deve permitir a busca de senadores por nome, sigla partidária ou UF.
    \item \textbf{[RF03]} O sistema deve exibir o perfil completo do senador com abas organizadas (Visão Geral, Gastos, Gabinete, Votações, Emendas).
\end{itemize}

\textbf{Módulo de Transparência Financeira (CEAPS):}
\begin{itemize}
    \item \textbf{[RF04]} O sistema deve importar os lançamentos da Cota Parlamentar (CEAPS) através das APIs de Dados Abertos do Senado.
    \item \textbf{[RF05]} O sistema deve permitir visualizar o gasto acumulado por tipo de despesa (passagens, correios, consultorias, combustível).
    \item \textbf{[RF06]} O sistema deve exibir os fornecedores que mais receberam recursos de um determinado senador.
    \item \textbf{[RF07]} O sistema deve gerar alertas automáticos para despesas atípicas: valores acima da média, pagamentos em finais de semana, intervalos menores que 3 dias entre pagamentos ao mesmo fornecedor.
\end{itemize}

\textbf{Módulo de Emendas e Orçamento:}
\begin{itemize}
    \item \textbf{[RF08]} O sistema deve integrar com o Portal da Transparência para buscar emendas de autoria do senador.
    \item \textbf{[RF09]} O sistema deve destacar valores destinados via ``Transferências Especiais'' (emendas PIX).
    \item \textbf{[RF10]} O sistema deve exibir mapas interativos de distribuição de emendas por município, permitindo identificar concentração geográfica de recursos.
\end{itemize}

\textbf{Módulo de Atividade Legislativa:}
\begin{itemize}
    \item \textbf{[RF11]} O sistema deve listar as votações nominais recentes e o voto de cada senador (Sim/Não/Abstenção).
    \item \textbf{[RF12]} O sistema deve exibir a participação do senador em comissões permanentes e especiais, incluindo cargo ocupado (titular, suplente, presidente, relator).
    \item \textbf{[RF13]} O sistema deve listar proposições de autoria do senador, com indicação do tipo (PEC, PLP, PL) e estágio de tramitação.
    \item \textbf{[RF14]} O sistema deve exibir discursos proferidos pelo senador em plenário, com data e tema.
    \item \textbf{[RF15]} O sistema deve exibir a agenda pública de reuniões de comissões, permitindo visualizar pautas futuras.
    \item \textbf{[RF16]} O sistema deve apresentar links para os perfis oficiais do senador em redes sociais (Twitter/X, Instagram, Facebook).
\end{itemize}

\textbf{Módulo de Gabinete:}
\begin{itemize}
    \item \textbf{[RF17]} O sistema deve exibir a lista de servidores do gabinete do senador, incluindo cargo e vínculo.
    \item \textbf{[RF18]} O sistema deve apresentar a folha de pagamento do gabinete, com remuneração mensal por servidor.
\end{itemize}

\textbf{Módulo de Comparação e Análise:}
\begin{itemize}
    \item \textbf{[RF19]} O sistema deve permitir comparar de 2 a 5 senadores lado a lado em múltiplas dimensões: despesas, emendas, votações e fornecedores em comum.
    \item \textbf{[RF20]} O sistema deve exibir ranking de fornecedores com cruzamento de sanções administrativas, identificando empresas com situação cadastral irregular.
    \item \textbf{[RF21]} O sistema deve mostrar indicadores de confiança: data da última sincronização, completude dos dados e fonte de cada informação.
\end{itemize}

\textbf{Módulo de Ranking e Score:}
\begin{itemize}
    \item \textbf{[RF22]} O sistema deve calcular e exibir o \textit{Score} de efetividade legislativa de cada senador, baseado no algoritmo de ranking do projeto.
    \item \textbf{[RF23]} O sistema deve apresentar gráfico radar com as quatro dimensões do \textit{Score}: Produtividade Legislativa, Presença em Votações, Economia na Cota e Participação em Comissões.
    \item \textbf{[RF24]} O sistema deve permitir ordenar e filtrar senadores por cada critério individual do ranking.
\end{itemize}

\subsection{Requisitos Não-Funcionais}

\textbf{Desempenho:}
\begin{itemize}
    \item \textbf{[RNF01]} O sistema deve responder a requisições de consulta em até 2 segundos sob condições normais de uso.
    \item \textbf{[RNF02]} A arquitetura deve suportar escalabilidade horizontal para lidar com picos de acesso em períodos eleitorais.
\end{itemize}

\textbf{Usabilidade e Acessibilidade:}
\begin{itemize}
    \item \textbf{[RNF03]} O sistema deve ser acessível via navegadores \textit{web} em dispositivos \textit{desktop} e \textit{mobile}.
    \item \textbf{[RNF04]} A interface deve seguir o padrão \textit{mobile-first} para garantir boa experiência em dispositivos móveis.
    \item \textbf{[RNF05]} O sistema deve seguir as diretrizes de acessibilidade WCAG 2.1 nível AA.
\end{itemize}

\textbf{Confiabilidade:}
\begin{itemize}
    \item \textbf{[RNF06]} Os dados devem ser sincronizados diariamente com as APIs oficiais do Senado Federal e Portal da Transparência.
    \item \textbf{[RNF07]} O sistema deve manter disponibilidade mínima de 99\% durante o período eleitoral.
\end{itemize}

\textbf{Segurança e Privacidade:}
\begin{itemize}
    \item \textbf{[RNF08]} As comunicações devem ser criptografadas utilizando HTTPS/TLS.
    \item \textbf{[RNF09]} O sistema deve estar em conformidade com a LGPD (Lei Geral de Proteção de Dados).
\end{itemize}

\textbf{Manutenibilidade:}
\begin{itemize}
    \item \textbf{[RNF10]} A arquitetura modular deve permitir atualizações isoladas de cada módulo sem impacto em outros componentes.
    \item \textbf{[RNF11]} O código deve seguir padrões de desenvolvimento (\textit{linting}, formatação) e estar documentado.
    \item \textbf{[RNF12]} O sistema deve possuir \textit{pipelines} de CI/CD para integração e deploy contínuos.
\end{itemize}
