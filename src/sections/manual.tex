\section{Manual do Usuário: A Jornada do Cidadão}
\label{sec:manual}

Para demonstrar a efetividade do \textit{Tô De Olho} como ferramenta de controle social, esta seção apresenta um **Cenário de Uso** prático. A narrativa ilustra o fluxo natural de um cidadão eleitor que deseja auditar a atuação dos senadores do seu estado, simulando a descoberta de informações na plataforma.

\subsection{Passo 1: O Primeiro Contato e a Transparência}

Ao acessar \url{https://todeolho.org}, o usuário é recepcionado pela \textbf{Página Inicial} (Figura \ref{fig:dashboard}). Imediatamente, visualiza-se o pódio com os ``Top 3'' senadores mais bem avaliados pelo algoritmo. Para compreender a origem das notas, aciona-se o botão ``Entenda o Cálculo''.

O sistema abre o modal de \textbf{Metodologia} (Figura \ref{fig:metodologia}), explicando de forma clara os pesos atribuídos a cada critério: Produtividade (35\%), Presença (25\%), Economia (20\%) e Comissões (20\%). Satisfeito com a transparência dos critérios, o usuário prossegue.

\begin{figure}[H]
    \centering
    \caption{Página Inicial: \textit{Dashboard} e Destaques}
    \includegraphics[width=1.0\textwidth]{src/images/dashboard.png}
    \label{fig:dashboard}
    \small{Fonte: O Autor (2026).}
\end{figure}

\begin{figure}[H]
    \centering
    \caption{Modal de Metodologia: Transparência dos Critérios}
    \includegraphics[width=1.0\textwidth]{src/images/modal_metodologia.png}
    \label{fig:metodologia}
    \small{Fonte: O Autor (2026).}
\end{figure}

\subsection{Passo 2: Explorando o \textit{Ranking} Completo}

Desejando ver como estão os senadores do seu estado, o cidadão clica em ``Ver \textit{Ranking} Completo'' e é direcionado para a listagem geral (Figura \ref{fig:ranking_completo}). Utilizando os filtros laterais, seleciona-se o estado (UF) e ordena-se a lista pelo critério de ``Economia de Verba''. A interface responde instantaneamente, permitindo identificar rapidamente quem são os representantes mais econômicos.

\begin{figure}[H]
    \centering
    \caption{\textit{Ranking} Completo com Filtros Ativos}
    \includegraphics[width=1.0\textwidth]{src/images/ranking_completo.png}
    \label{fig:ranking_completo}
    \small{Fonte: O Autor (2026).}
\end{figure}

\subsection{Passo 3: A Lupa no Parlamentar}

Interessado por um nome específico na lista, o usuário seleciona o perfil do senador para uma auditoria detalhada. A \textbf{Ficha Parlamentar} se abre (Figura \ref{fig:ficha_senador}), exibindo o ``\textit{Score} de Efetividade'' em um gráfico radar. Nota-se que o senador tem alta pontuação em ``Presença'', mas baixa em ``Proposições''.

Para investigar a baixa produtividade, navega-se pelas abas:

\begin{itemize}
    \item \textbf{Votações (Figura \ref{fig:votacoes})}: Analisa-se o histórico de votos em plenário, verificando se o parlamentar seguiu a orientação partidária ou votou de acordo com convicções pessoais em pautas polêmicas.
    \item \textbf{Despesas (Figura \ref{fig:despesas})}: No gráfico de gastos, percebe-se um pico anormal em dezembro. Ao clicar na barra correspondente, o sistema exibe a lista de notas fiscais, revelando gastos elevados com divulgação de atividade parlamentar.
    \item \textbf{Emendas (Figura \ref{fig:mapa_emendas})}: O mapa interativo mostra que a maior parte dos recursos indicados pelo senador foi para apenas dois municípios, sugerindo uma concentração em colégios eleitorais específicos.
    \item \textbf{Comissões (Figura \ref{fig:comissoes})}: Por fim, verifica-se que o senador é suplente em apenas uma comissão, confirmando a baixa atividade legislativa.
\end{itemize}

\begin{figure}[H]
    \centering
    \caption{Ficha Parlamentar: Visão Geral e Radar}
    \includegraphics[width=1.0\textwidth]{src/images/ficha_parlamentar.png}
    \label{fig:ficha_senador}
    \small{Fonte: O Autor (2026).}
\end{figure}

\begin{figure}[H]
    \centering
    \caption{Histórico de Votações Nominais}
    \includegraphics[width=1.0\textwidth]{src/images/ficha_votacoes_detalhe.png}
    \label{fig:votacoes}
    \small{Fonte: O Autor (2026).}
\end{figure}

\begin{figure}[H]
    \centering
    \caption{Detalhamento de Gastos (CEAPS)}
    \includegraphics[width=1.0\textwidth]{src/images/ficha_ceaps_detalhe.png}
    \label{fig:despesas}
    \small{Fonte: O Autor (2026).}
\end{figure}

\begin{figure}[H]
    \centering
    \caption{Mapa de Distribuição de Emendas}
    \includegraphics[width=1.0\textwidth]{src/images/mapa_emendas_detalhe.png}
    \label{fig:mapa_emendas}
    \small{Fonte: O Autor (2026).}
\end{figure}

\begin{figure}[H]
    \centering
    \caption{Participação em Comissões}
    \includegraphics[width=1.0\textwidth]{src/images/ficha_comissoes_detalhe.png}
    \label{fig:comissoes}
    \small{Fonte: O Autor (2026).}
\end{figure}

\subsection{Passo 4: A Decisão Comparativa}

Ainda em dúvida entre dois candidatos, utiliza-se a ferramenta de comparação. Na página de \textit{ranking} ou no perfil, marca-se a caixa de seleção ``Comparar'' para o Senador A e o Senador B. O \textbf{Dock de Comparação} surge no rodapé da tela (Figura \ref{fig:dock}), confirmando a seleção.

Ao clicar em ``Comparar Agora'', o usuário é levado à tela de Comparação Lado-a-Lado (Figura \ref{fig:comparacao_tela}), onde visualiza os gráficos dos dois parlamentares sobrepostos. A comparação revela que, embora o Senador A seja mais econômico, o Senador B possui uma produção legislativa significativamente superior e presença constante em comissões importantes. Munido de dados concretos e comparáveis, encerra-se a jornada com informações suficientes para uma decisão de voto fundamentada.

\begin{figure}[H]
    \centering
    \caption{Dock de Seleção para Comparação}
    \includegraphics[width=1.0\textwidth]{src/images/dock_comparacao.png}
    \label{fig:dock}
    \small{Fonte: O Autor (2026).}
\end{figure}

\begin{figure}[H]
    \centering
    \caption{Tela de Comparação Lado-a-Lado (Resultado)}
    \includegraphics[width=1.0\textwidth]{src/images/comparacao_tela.png}
    \label{fig:comparacao_tela}
    \small{Fonte: O Autor (2026).}
\end{figure}
% Seção 8.3 removida
