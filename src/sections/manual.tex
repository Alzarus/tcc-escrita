\section{Manual do Usuário Simplificado}
\label{sec:manual}

O \textit{Tô De Olho} foi concebido sob a premissa de máxima acessibilidade, removendo barreiras de entrada para o cidadão comum. Diferentemente de sistemas governamentais tradicionais, a plataforma não exige cadastro, \textit{login} ou conhecimentos técnicos prévios. Esta seção descreve os fluxos principais de interação, desenhados para permitir uma fiscalização intuitiva e eficiente.

\subsection{Acesso e Visão Geral}

Ao acessar o endereço \url{https://todeolho.org}, o usuário é recepcionado pela \textbf{Página Inicial}, que atua como um painel de síntese ("\textit{dashboard}"). O destaque visual é dado aos rankings dinâmicos, apresentando os "Top 3" senadores em critérios positivos, como maior economia na Cota Parlamentar e maior assiduidade em votações (Figura \ref{fig:dashboard}).

\begin{figure}[H]
    \centering
    \caption{Página Inicial com Rankings e Destaques}
    \includegraphics[width=1.0\textwidth]{src/images/placeholder_dashboard.png} % Substituir por print real
    \label{fig:dashboard}
    \small{Fonte: O Autor (2026).}
\end{figure}

\begin{enumerate}
    \item \textbf{Barra de Busca Global}: Localizada no topo de todas as páginas, permite localizar senadores por nome, apelido parlamentar ou estado (UF). A busca possui recurso de autocompletar, facilitando a navegação.
    \item \textbf{Mapa Interativo}: Visualização geográfica que permite filtrar senadores clicando diretamente nos estados do mapa do Brasil (Figura \ref{fig:busca_mapa}).
\end{enumerate}

\begin{figure}[H]
    \centering
    \caption{Recursos de Busca e Navegação por Mapa}
    \includegraphics[width=1.0\textwidth]{src/images/placeholder_mapa.png} % Substituir por print real
    \label{fig:busca_mapa}
    \small{Fonte: O Autor (2026).}
\end{figure}

\subsection{Análise do Perfil Parlamentar}

Ao selecionar um senador, o cidadão tem acesso à \textbf{Ficha Parlamentar}, estruturada em abas temáticas para evitar sobrecarga cognitiva:

\begin{itemize}
    \item \textbf{Visão Geral}: Apresenta o "Score de Efetividade" (gráfico radar) e dados biográficos essenciais.
    \item \textbf{Análise de Gastos}: Gráficos de evolução temporal das despesas da CEAPS. É possível "mergulhar" nos dados (\textit{drill-down}) clicando nas barras do gráfico para visualizar a lista detalhada de notas fiscais daquele mês.
    \item \textbf{Emendas e Recursos}: Exibe o destino das emendas parlamentares. O destaque é o mapa de "Emendas PIX", onde cada ponto representa um município beneficiado, permitindo ao usuário verificar se recursos foram enviados para sua cidade sem finalidade definida.
    \item \textbf{Votações}: Histórico de posicionamentos do senador (Sim/Não/Abstenção/Ausência) em votações nominais no plenário.
\end{itemize}

A Figura \ref{fig:ficha_senador} ilustra a consolidação dessas informações na interface do usuário.

\begin{figure}[H]
    \centering
    \caption{Ficha Parlamentar: Visão Geral e Gráficos}
    \includegraphics[width=1.0\textwidth]{src/images/placeholder_perfil.png} % Substituir por print real
    \label{fig:ficha_senador}
    \small{Fonte: O Autor (2026).}
\end{figure}

\subsection{Ferramentas de Compartilhamento}

Reconhecendo o papel das redes sociais na disseminação política, todas as visualizações (gráficos e rankings) possuem botões de compartilhamento direto para WhatsApp, Twitter/X e Facebook. Ao clicar, o sistema gera automaticamente uma imagem sintetizada ("\textit{card}") com as estatísticas do senador, facilitando a viralização de informações factuais e a fiscalização coletiva.
