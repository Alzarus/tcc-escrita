\clearpage
\thispagestyle{empty}
\begin{center}
    \section*{Abstract}
\end{center}

This work proposes the development of \textit{Tô De Olho}, a web platform for parliamentary transparency focused on the Federal Senate. The Senate, despite its legislative importance, lacks integrated citizen oversight tools to facilitate social control, in contrast to the consolidated ecosystem for the Chamber of Deputies. The solution adopts a modular monolith architecture in Go, with an interface implemented in Next.js, and integrates data from three government APIs (Legislative Senate, Administrative Senate, and Transparency Portal) to consolidate information on legislative activity, execution of the Quota for the Exercise of Parliamentary Activity (CEAPS), and parliamentary amendments allocation. As a methodological differentiator, a Legislative Effectiveness Index adapted to the Brazilian context is presented, weighting productivity, floor attendance, resource economy, and participation in committees. The results demonstrate the technical feasibility of unifying heterogeneous data sources and the platform's potential to reduce informational asymmetry, transforming passive transparency into active accountability and qualifying public debate for the 2026 electoral scenario. The platform adopts accessibility premises (WCAG) and mobile-first design to democratize access according to the Brazilian population profile.

\vspace{1cm}

\noindent
\textbf{Keywords}: Public Transparency. Federal Senate. Open Data. Accountability. Software Engineering.
