\section{Testes de Software}
\label{sec:testes}

A garantia da qualidade em sistemas de transparência pública assume papel crítico, dado que falhas na apresentação de dados podem comprometer a confiança do cidadão nas instituições monitoradas. A estratégia de validação do \textit{Tô De Olho} fundamentou-se no modelo da Pirâmide de Testes \cite{myers2011art}, priorizando uma base sólida de testes automatizados de baixo nível para garantir \textit{feedback} rápido durante o ciclo de desenvolvimento.

A implementação seguiu as diretrizes de \textit{Continuous Testing}, integrando as rotinas de verificação ao \textit{pipeline} de entrega contínua. Para isso, utilizou-se o ferramental nativo da linguagem Go, complementado por bibliotecas específicas para cenários de integração.

\subsection{Testes Unitários}

Os testes unitários constituem a primeira linha de defesa contra regressões, focando na validação isolada de componentes de lógica de negócio e utilitários. A escolha da linguagem Go favoreceu esta prática através de seu suporte nativo no pacote \texttt{testing}, que elimina a necessidade de \textit{frameworks} externos complexos.

Adotou-se o padrão de \textit{Table-Driven Tests} (Testes Orientados a Tabelas), idiomático em Go. Esta técnica permite testar múltiplos cenários (casos de sucesso, erros de borda, entradas inválidas) reutilizando a mesma lógica de asserção, o que aumenta a legibilidade e facilita a manutenção.

\begin{itemize}
    \item \textbf{Cálculo de Scores}: Validação dos algoritmos de pontuação dos senadores, garantindo que os pesos da metodologia SLES (Seção \ref{sec:metodologia}) sejam aplicados corretamente sobre os dados brutos.
    \item \textbf{Parsers de Dados}: Verificação das rotinas de tratamento de dados provenientes das APIs do Senado, assegurando a correta conversão de formatos (XML, JSON) e tipos numéricos monetários.
\end{itemize}

A cobertura de código (\textit{code coverage}) foi monitorada com a ferramenta nativa \texttt{go test -cover}, estabelecendo-se uma meta mínima de 80\% para pacotes críticos do domínio.

\subsection{Testes de Integração}

Diferentemente dos testes unitários, que utilizam \textit{mocks} para isolar dependências, os testes de integração validam a interação entre os módulos do sistema e componentes de infraestrutura real, como o banco de dados PostgreSQL e o cache Redis.

Para viabilizar esses testes de forma isolada e reproduzível, utilizou-se a biblioteca \texttt{testcontainers-go}. Esta ferramenta permite instanciar contêineres Docker descartáveis durante a execução da suíte de testes.

\begin{enumerate}
    \item \textbf{Persistência de Dados}: Testes que verificam se as operações de escrita e leitura no PostgreSQL respeitam as restrições de integridade referencial e unicidade, essenciais para a idempotência da ingestão.
    \item \textbf{Mecanismos de Cache}: Validação das estratégias de invalidação e expiração de chaves no Redis, garantindo que o usuário final não visualize dados obsoletos.
\end{enumerate}

\subsection{Testes de Contrato}

Considerando que o \textit{Tô De Olho} depende integralmente da estabilidade de APIs de terceiros (Senado e Portal da Transparência), implementaram-se testes de contrato para detectar mudanças não anunciadas nos formatos de resposta (\textit{breaking changes}).

Estes testes realizam requisições controladas às APIs externas e validam se a estrutura do JSON/XML retornado corresponde aos esquemas esperados pelos \textit{structs} da aplicação. Falhas nesta camada disparam alertas para a equipe de desenvolvimento, sinalizando a necessidade de adaptação nos adaptadores de integração antes que o erro impacte o ambiente de produção.
