\subsection{Objetivos}

\subsubsection{Objetivo Geral}
Desenvolver uma plataforma \textit{web} de transparência política que centralize, organize e simplifique o acesso aos dados públicos do Senado Federal, fomentando a fiscalização cidadã e o debate qualificado sobre a atuação dos 81 senadores, com ênfase no monitoramento de gastos e emendas parlamentares.

\subsubsection{Objetivos Específicos}
\begin{itemize}
    \item Implementar um \textbf{\textit{backend} em \textit{Go}} com arquitetura de monolito modular para ingestão de dados das APIs oficiais do Senado e do Portal da Transparência;
    \item Desenvolver rotinas ETL para consumir as APIs Legislativa, Administrativa e do Portal da Transparência, priorizando fontes estruturadas;
    \item Desenvolver algoritmos de \textit{Ranking} baseados no estudo da metodologia \textit{Legislative Effectiveness \textit{Score}} (LES) de Volden e Wiseman (2014), adaptando critérios de efetividade legislativa ao contexto brasileiro para avaliar senadores com base em presença em votações, produtividade legislativa, economia na cota parlamentar e participação em comissões;
    \item Construir uma interface \textit{front-end} responsiva utilizando \textit{Next.js}, permitindo a visualização intuitiva de perfis, despesas e \textit{scorecards} de fiscalização;
    \item Garantir a acessibilidade e a usabilidade em dispositivos móveis por meio da adoção de componentes \textit{shadcn/ui}, construídos sobre \textit{Radix UI Primitives} com conformidade nativa às diretrizes WCAG 2.1 nível AA, e abordagem \textit{mobile-first} considerando o perfil de acesso à internet da população brasileira.
\end{itemize}