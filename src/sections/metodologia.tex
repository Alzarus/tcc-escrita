\section{Metodologia}
\label{sec:metodologia}

O desenvolvimento do \textit{Tô De Olho} enquadra-se no paradigma da \textit{Design Science Research} (DSR), abordagem metodológica adequada para pesquisas que visam a construção e avaliação de artefatos tecnológicos destinados a resolver problemas organizacionais identificados \cite{hevner2004design}. O problema abordado --- a fragmentação de dados públicos sobre a atuação de senadores federais em múltiplas APIs governamentais não integradas --- demanda a construção de um artefato de software capaz de consolidar, processar e apresentar essas informações de forma acessível ao cidadão.

Esta seção detalha a abordagem de desenvolvimento iterativo adotada, as fontes de dados governamentais integradas e a metodologia de avaliação de desempenho parlamentar inspirada no \textit{Legislative Effectiveness \textit{Score}}.

\subsection{Abordagem de Desenvolvimento}

A natureza do projeto --- integração de múltiplas APIs governamentais com estruturas de dados heterogêneas e documentação variável --- demandou uma abordagem de desenvolvimento capaz de acomodar descobertas incrementais e ajustes frequentes de escopo. Metodologias tradicionais de desenvolvimento em cascata, que pressupõem requisitos estáveis e bem definidos desde o início, mostraram-se inadequadas para este contexto de exploração de APIs públicas com comportamentos nem sempre previsíveis.

Optou-se, portanto, por uma abordagem \textbf{iterativa e incremental}, na qual o trabalho foi organizado em ciclos de desenvolvimento focados em entregas funcionais. Cada ciclo produzia um incremento utilizável do sistema, permitindo validação contínua das funcionalidades implementadas e ajustes baseados nos aprendizados obtidos durante a integração com cada API.

O desenvolvimento contou com apoio de ferramentas de inteligência artificial generativa como assistentes de codificação, seguindo a tendência contemporânea de desenvolvimento de software assistido por IA (\textit{AI-assisted software development}) \cite{peng2023impact}. Tais ferramentas, baseadas em modelos de linguagem de grande escala (LLMs), foram empregadas para aceleração de tarefas operacionais como geração de código padrão (\textit{boilerplate}), refatoração e depuração (\textit{debugging}). O uso de assistentes de IA em desenvolvimento de software representa uma evolução natural das ferramentas de produtividade, análoga à adoção de IDEs com preenchimento automático (\textit{autocomplete}) e analisadores estáticos de código.

É fundamental distinguir entre \textbf{assistência operacional} e \textbf{autoria intelectual}. As decisões estruturantes do projeto --- arquitetura do sistema, design do algoritmo de \textit{ranking}, seleção de fontes de dados, modelagem do domínio e interpretação de resultados --- foram integralmente concebidas, avaliadas e validadas pelo desenvolvedor. A ferramenta de IA atuou como acelerador de implementação, não como substituto do julgamento técnico.

A divisão do trabalho ocorreu em cinco fases principais:
\begin{enumerate}
    \item \textbf{Fundação}: Estruturação do projeto em \textit{Go}, implementação do cliente para a API Legislativa do Senado, criação das migrações (\textit{migrations}) do banco de dados e configuração inicial do \textit{front-end} em \textit{Next.js} utilizando \textit{Bun} como ambiente de execução (\textit{runtime});
    \item \textbf{Ingestão de Dados}: Implementação do cliente para a API Administrativa, configuração do agendador (\textit{scheduler}) para tarefas agendadas, coleta de votações nominais e carga de dados históricos;
    \item \textbf{Classificação e API}: Desenvolvimento do serviço de cálculo de pontuações, criação dos pontos de acesso (\textit{endpoints}) REST para consumo pelo \textit{front-end}, configuração do cache \textit{Redis} e implementação de testes automatizados;
    \item \textbf{\textit{Front-end}}: Desenvolvimento do painel principal (\textit{dashboard}), interface de classificação interativa e páginas de perfil dos senadores;
    \item \textbf{Emendas e Polimento}: Integração com o Portal da Transparência para dados de emendas parlamentares, visualizações de dados e preparação para implantação.
\end{enumerate}

\subsection{Fontes de Dados}

O sistema integra três fontes de dados oficiais, fundamentadas no arcabouço legal brasileiro de transparência pública. A Lei de Acesso à Informação (Lei n. 12.527/2011) estabelece como diretriz a ``disponibilização de informações em formatos abertos, estruturados e legíveis por máquina'' \cite{lai2011}, princípio que as APIs governamentais operacionalizam. Conforme demonstrado pela Operação Serenata de Amor, tecnologias desenvolvidas sobre esses dados abertos podem gerar valor público ao facilitar o controle social do gasto parlamentar \cite{albuquerque2018serenata}.

A seleção dos \textit{endpoints} seguiu três critérios: (i) \textbf{relevância} --- priorizando dados de gastos, votações e atuação legislativa; (ii) \textbf{confiabilidade} --- selecionando fontes com documentação oficial; e (iii) \textbf{viabilidade técnica} --- considerando formatos estruturados (JSON/XML) e limites de requisição adequados.

\subsubsection{API Legislativa do Senado}

Documentada em \url{legis.senado.leg.br/dadosabertos}, esta API RESTful fornece dados do processo legislativo. A URL base para requisições é \url{https://legis.senado.leg.br/dadosabertos}. Os dados são retornados em formato JSON, com suporte a paginação e limite de 10 requisições por segundo. Os principais recursos utilizados incluem: lista de senadores em exercício, detalhes biográficos, histórico de mandatos e licenças, votações nominais em plenário e comissões, orientação partidária, proposições de autoria, relatorias, participação em comissões e discursos.

\subsubsection{API Administrativa do Senado}

Documentada em \url{adm.senado.gov.br/adm-dadosabertos/swagger-ui}, esta interface disponibiliza dados financeiros e administrativos. Os recursos utilizados incluem: despesas da Cota Parlamentar (CEAPS) com detalhamento por fornecedor e tipo, opção por auxílio-moradia, escritórios de apoio, lista de servidores de gabinete, remunerações mensais e mapeamento de lotações.

\subsubsection{Portal da Transparência (CGU)}

A API do Portal da Transparência (\url{api.portaldatransparencia.gov.br}) fornece dados de emendas parlamentares e transferências federais. O acesso requer autenticação via chave de API no cabeçalho (\textit{header}) \texttt{chave-api-dados}. O filtro \texttt{tipoEmenda=Transferência Especial} permite identificar as ``emendas PIX'', modalidade de repasse que dispensa convênio.

Para a carga histórica, utiliza-se o CSV consolidado de emendas disponibilizado pelo Portal. A identificação do autor depende de normalização textual de \texttt{nomeAutor} para casar com o cadastro de senadores, garantindo ingestão consistente mesmo com variações de grafia.

\subsubsection{Limitações e Cobertura Temporal}

Cada fonte de dados apresenta limitações que impactam o escopo das análises:

\begin{itemize}
    \item \textbf{API Legislativa}: Votações nominais disponíveis a partir de 2023 (legislatura 57). Limite de 10 requisições por segundo;
    \item \textbf{API Administrativa}: Despesas CEAPS disponíveis desde 2008 em CSV; API REST cobre apenas o ano corrente. Não há \textit{endpoint} para vincular servidores diretamente ao gabinete;
    \item \textbf{Portal da Transparência}: Emendas disponíveis a partir de 2015. Busca por autor utiliza nome textual, exigindo normalização de grafia. Limite de 300 requisições por minuto.
\end{itemize}

\subsection{Metodologia de Avaliação de Desempenho}

A avaliação objetiva do desempenho parlamentar é elemento central para a participação cidadã. Conforme a escada de participação de Arnstein (\citeyear{arnstein1969ladder}), o acesso a informações claras é pré-requisito para que cidadãos avancem de níveis consultivos para controle social efetivo. A construção de índices de classificação (\textit{rankings}) requer transparência metodológica \cite{hullman2011visualization}. Ressalta-se que o índice proposto não pretende mensurar ``qualidade moral'' ou ``mérito político'' subjetivo dos parlamentares, mas sim oferecer um instrumento comparativo fundamentado em critérios observáveis e quantificáveis de produtividade e gestão fiscal.

O cálculo da pontuação (\textit{score}) de cada senador é inspirado no \textit{Legislative Effectiveness Score} (LES), metodologia desenvolvida por Volden e Wiseman (\citeyear{volden2014lawmakers}) para avaliar a efetividade legislativa de parlamentares americanos. O LES define efetividade legislativa como ``a capacidade comprovada de avançar itens da agenda de um parlamentar através do processo legislativo até sua transformação em lei''.

Para a adaptação brasileira, manteve-se a filosofia de valorizar o avanço de proposições através do processo legislativo, incorporando critérios adicionais relevantes ao contexto de fiscalização cidadã. O índice proposto compõe-se de quatro dimensões objetivamente mensuráveis:

\begin{table}[H]
\centering
\caption{Critérios do algoritmo de \textit{ranking}}
\label{tab:criterios}
\begin{tabular}{|l|c|l|}
\hline
\textbf{Critério} & \textbf{Peso} & \textbf{Justificativa} \\
\hline
Produtividade Legislativa & 35\% & Núcleo (\textit{Core}) do LES: capacidade de aprovar proposições \\
Presença em Votações & 25\% & Compromisso efetivo com o mandato \\
Economia na Cota (CEAPS) & 20\% & Responsabilidade fiscal \\
Participação em Comissões & 20\% & Trabalho técnico especializado \\
\hline
\end{tabular}
\end{table}

\subsubsection{Produtividade Legislativa}

Este critério avalia o avanço de proposições de autoria do senador através do processo legislativo, atribuindo pontuação crescente por estágio alcançado (Apresentado: 1 ponto, Em Comissão: 2, Aprovado em Comissão: 4, Aprovado em Plenário: 8, Transformado em Lei: 16). Os multiplicadores refletem a maior complexidade de Propostas de Emenda Constitucional (PEC: $\times$3.0) e Projetos de Lei Complementar (PLP: $\times$2.0). Senadores que atuam como relatores recebem pontuação adicional.

\subsubsection{Presença em Votações}

Calculada como a razão entre votações participadas e votações disponíveis:

\begin{equation}
\text{Presença} = \frac{\text{Votações Participadas}}{\text{Votações Disponíveis}} \times 100
\end{equation}

As votações disponíveis excluem períodos de licença oficial (médica ou para cargo executivo), garantindo que afastamentos justificados não penalizem o senador. Obstrução \textbf{não} conta como presença.

\subsubsection{Economia na Cota Parlamentar}

Calculada como a proporção não utilizada do teto da CEAPS:

\begin{equation}
\text{Economia} = \left(1 - \frac{\text{Gasto Senador}}{\text{Teto CEAPS}}\right) \times 100
\end{equation}

O teto da CEAPS varia por UF, composto por verba indenizat\'{o}ria fixa (R\$ 15.000) e verba de transporte a\'{e}reo vari\'{a}vel. Os valores de mar\c{c}o de 2025 v\~{a}o de R\$ 36.582 (DF/GO/TO) at\'{e} R\$ 52.798 (AM), com m\'{e}dia nacional de R\$ 46.402. Senadores com gasto acima de 120\% do teto recebem pontua\c{c}\~{a}o zero neste crit\'{e}rio.

\subsubsection{Participação em Comissões}

Pontuação atribuída conforme o nível de engajamento: membro titular (+2 pontos), suplente (+1), vice-presidente (+3), presidente (+5). Comissões estratégicas (CAE, CCJ) recebem multiplicador de $\times$1.5.

\subsubsection{Fórmula Final e Tratamento de Casos Especiais}

Cada métrica é normalizada de 0 a 100 antes da ponderação:

\begin{equation}
\textit{Score} = (P_L \times 0.35) + (P_V \times 0.25) + (E_C \times 0.20) + (P_C \times 0.20)
\end{equation}

Onde:
\begin{itemize}
    \item $P_L$: Produtividade Legislativa;
    \item $P_V$: Presença em Votações;
    \item $E_C$: Economia na Cota;
    \item $P_C$: Participação em Comissões.
\end{itemize}

Casos especiais são tratados da seguinte forma: senadores novos (suplentes) requerem período mínimo de 30 dias para entrar na classificação; licenças curtas mantêm o senador na lista com ajuste nos denominadores; licenças longas (>30 dias) congelam a pontuação; dados indisponíveis causam reponderação proporcional dos demais critérios.

\subsubsection{Limitações Metodológicas e Ciclo Eleitoral}
\label{sec:limitacoes}

A renovação do Senado Federal em terços alternados --- 1/3 em uma eleição, 2/3 na seguinte --- implica que senadores ativos na 57ª legislatura possuem mandatos iniciados em anos distintos (2019 ou 2023). O índice proposto não normaliza automaticamente por tempo de mandato ativo no Produto Mínimo Viável (MVP), optando por oferecer filtros temporais que permitem ao usuário realizar comparações contextualizadas.

Uma característica singular do Senado Federal é o seu sistema de renovação parcial: a cada quatro anos, alternam-se as eleições para um terço ou dois terços das cadeiras. Isso implica que, em um dado momento, convivem senadores em estágios distintos de seus mandatos de oito anos. Por exemplo, na 57ª Legislatura (2023-2027), coexistem parlamentares eleitos em 2018 (encerrando mandato) e parlamentares eleitos em 2022 (iniciando mandato).

Essa assincronia impõe desafios para a comparação direta de desempenho acumulado. Um senador em final de mandato naturalmente acumula mais proposições e discursos do que um recém-empossado. Para mitigar essa distorção, a interface de comparação (Figura \ref{fig:comparador}) permite filtrar o desempenho por ano.

\begin{figure}[H]
    \centering
    \includegraphics[width=0.95\textwidth]{imagens/comparador-v2.png}
    \caption{Interface de Comparação de Senadores}
    \label{fig:comparador}
    \par\small{Fonte: Autoria Própria}
\end{figure}

Para garantir a isonomia da classificação, adotaram-se duas decisões metodológicas:

\begin{enumerate}
    \item \textbf{Filtro Anual como Padrão}: A comparação principal é realizada dentro de janelas de tempo anuais (ex: desempenho em 2024), período em que todos os parlamentares ativos estiveram submetidos às mesmas condições de tempo;
    \item \textbf{Recorte da Legislatura Atual}: Para análises acumuladas, o sistema considera dados a partir de fevereiro de 2023 (início da 57ª Legislatura), normalizando o período de análise para todos os senadores atuais, independentemente do ano de eleição.
\end{enumerate}

Para trabalhos futuros, planeja-se a implementação de uma normalização por ``meses de mandato ativo'', permitindo comparações mais justas entre legislaturas completas.
