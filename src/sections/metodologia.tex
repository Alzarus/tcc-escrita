\section{Metodologia}

Esta seção descreve a metodologia adotada para o desenvolvimento do \textit{Tô De Olho}, detalhando a abordagem de desenvolvimento, as fontes de dados utilizadas, a arquitetura do sistema e a infraestrutura de implantação.

\subsection{Abordagem de Desenvolvimento}

O desenvolvimento do projeto seguiu uma abordagem iterativa e incremental, inspirada em práticas ágeis. O trabalho foi organizado em ciclos de desenvolvimento focados em entregas funcionais, permitindo validação contínua das funcionalidades implementadas.

A divisão do trabalho ocorreu em cinco fases principais:
\begin{enumerate}
    \item \textbf{Fundação}: Estruturação do projeto em Golang, implementação do cliente para a API Legislativa do Senado, criação das \textit{migrations} do banco de dados e configuração inicial do \textit{frontend} em Next.js;
    \item \textbf{Ingestão de Dados}: Implementação do cliente para a API Administrativa, configuração do \textit{scheduler} para tarefas agendadas, coleta de votações nominais e carga de dados históricos;
    \item \textbf{Ranking e API}: Desenvolvimento do serviço de cálculo de rankings, criação dos \textit{endpoints} REST para consumo pelo \textit{frontend}, configuração do cache Redis e implementação de testes automatizados;
    \item \textbf{Frontend}: Desenvolvimento do \textit{dashboard} principal, interface de ranking interativo e páginas de perfil dos senadores;
    \item \textbf{Emendas e Polimento}: Integração com o Portal da Transparência para dados de emendas parlamentares, visualizações de dados e preparação para \textit{deploy}.
\end{enumerate}

\subsection{Fontes de Dados}

O sistema integra três fontes de dados governamentais oficiais, cada uma com características e formatos distintos:

\subsubsection{API Legislativa do Senado}

Disponível em \texttt{legis.senado.leg.br/dadosabertos}, esta API RESTful fornece dados do processo legislativo: lista de senadores em exercício, histórico de mandatos, votações nominais em plenário e comissões, proposições de autoria parlamentar, discursos proferidos e composição das comissões. Os dados são retornados em formato JSON, com paginação para grandes volumes.

\subsubsection{API Administrativa do Senado}

Acessível em \texttt{adm.senado.gov.br/adm-dadosabertos}, esta interface disponibiliza dados administrativos: lançamentos da Cota para o Exercício da Atividade Parlamentar dos Senadores (CEAPS), informações sobre auxílio-moradia, escritórios de apoio, lista de servidores por lotação e remunerações mensais. Alguns dados sensíveis são disponibilizados apenas em arquivos CSV para \textit{download}.

\subsubsection{Portal da Transparência (CGU)}

A API do Portal da Transparência (\texttt{api.portaldatransparencia.gov.br}) fornece dados de emendas parlamentares e transferências da União. Para acesso, é necessária autenticação via chave de API. O filtro \texttt{tipoEmenda=Transferência Especial} permite identificar as chamadas ``emendas PIX'', modalidade de repasse que dispensa convênio e apresenta menor rastreabilidade.

\subsection{Estratégia de Ingestão}

A ingestão de dados segue uma estratégia híbrida que combina carga inicial massiva com atualização contínua:

\textbf{Backfill (Carga Histórica)}: Para a população inicial do banco de dados com dados históricos, podem ser utilizados arquivos em massa CSV quando disponíveis. A API Administrativa do Senado fornece endpoint REST para consulta de despesas CEAPS por ano, sendo a fonte principal para sincronização contínua.

\textbf{Sincronização Contínua}: Tarefas agendadas (\textit{cron jobs}) executam diariamente a coleta de atualizações via APIs. O serviço de ingestão verifica a data da última atualização de cada entidade e solicita apenas os registros novos ou modificados, otimizando o consumo de recursos e respeitando os limites de requisição das APIs.

\subsection{Arquitetura do Sistema}

O \textit{Tô De Olho} adota uma arquitetura de \textbf{monolito modular}, na qual os componentes são organizados internamente em módulos bem definidos:

\begin{itemize}
    \item \texttt{internal/senador}: Gerencia dados cadastrais dos parlamentares e cálculo de rankings;
    \item \texttt{internal/ceaps}: Processa e totaliza despesas da Cota Parlamentar;
    \item \texttt{internal/emenda}: Integra dados de emendas e transferências do Portal da Transparência;
    \item \texttt{internal/votacao}: Coleta e armazena votações nominais;
    \item \texttt{internal/ranking}: Orquestra o cálculo de scores e agregações.
\end{itemize}

Os módulos internos comunicam-se através de chamadas de função diretas, simplificando o fluxo de dados e facilitando a depuração. Para operações de ingestão, utilizam-se \textit{goroutines} para processamento concorrente.

\subsection{Stack Tecnológico}

A escolha das tecnologias foi orientada por critérios de desempenho, manutenibilidade e adequação ao domínio do problema. A seguir, justificamos cada escolha com base em benchmarks e literatura técnica.

\textbf{Backend --- Golang:}

A linguagem Go foi selecionada para o \textit{backend} após avaliação de alternativas como Node.js e Python. O fator decisivo foi o modelo de \textbf{concorrência nativo}: as \textit{goroutines} são threads leves gerenciadas pelo runtime Go, consumindo aproximadamente 2KB de memória inicial --- em contraste com threads do sistema operacional que utilizam cerca de 1MB cada \cite{nanz2015comparative}. Essa eficiência permite processar centenas de milhares de conexões simultâneas, característica essencial para a ingestão paralela de dados de três APIs distintas. Adicionalmente, a tipagem estática e a compilação para binário único simplificam o \textit{deploy} em contêineres e a robustez do sistema.

O framework \textbf{Gin} foi escolhido por seu roteamento HTTP baseado em \textit{radix tree}, reportando desempenho até 40 vezes superior a frameworks anteriores como Martini \cite{alfian2024gin}. \textbf{GORM} oferece mapeamento objeto-relacional com suporte a migrações, reduzindo a complexidade de gerenciamento do esquema de banco.

\textbf{Banco de Dados --- PostgreSQL e Redis:}

\textbf{PostgreSQL} foi selecionado como banco relacional por sua robustez em consultas analíticas e suporte a índices compostos, essenciais para agregações por senador, período e tipo de despesa. O sistema originou-se em 1986 na Universidade da Califórnia, Berkeley, liderado pelo Professor Michael Stonebraker como sucessor do projeto Ingres \cite{stonebraker1986postgres}. A conformidade ACID garante integridade nas operações de ingestão, enquanto sua adoção pela comunidade \textit{open source} é respaldada por décadas de evolução acadêmica e industrial.

\textbf{Redis} atua como camada de cache para rankings pré-computados e totalizadores de gastos. Por armazenar dados inteiramente em memória RAM, Redis alcança latências típicas entre 100 e 500 microssegundos --- aproximadamente 1.250 vezes mais rápido que SSDs \cite{redis2024}. Essa característica elimina a necessidade de recalcular métricas a cada requisição, atendendo ao requisito de latência inferior a 2 segundos (RNF01).

\textbf{Frontend --- Next.js 15:}

A escolha de \textbf{Next.js} sobre React ``puro'' justifica-se pela necessidade de \textbf{SEO} e performance inicial \cite{salim2024nextjs}. Aplicações React tradicionais utilizam renderização no cliente (CSR), onde motores de busca recebem uma página HTML vazia que depende de JavaScript para exibir conteúdo --- prejudicando a indexação. Next.js resolve esse problema com \textit{Server-Side Rendering} (SSR) e \textit{Static Site Generation} (SSG), entregando HTML pré-renderizado aos \textit{crawlers}. Estudos indicam melhorias significativas em métricas de Core Web Vitals como First Contentful Paint (FCP) e Largest Contentful Paint (LCP) \cite{nextjsseo2024}.

A biblioteca \textbf{Recharts} foi selecionada para visualização de dados por sua integração nativa com React e suporte a gráficos responsivos \cite{recharts2024}. \textbf{Tailwind CSS} permite estilização eficiente com classes utilitárias; a partir da versão 4, o modo \textit{Just-In-Time} (JIT) gera apenas o CSS utilizado em cada página, resultando em bundles frequentemente inferiores a 10KB \cite{tailwindcss2024}.

% Scraping removido: decisão arquitetural de usar exclusivamente APIs oficiais

\subsection{Infraestrutura e Implantação}

Todos os componentes são containerizados com Docker, utilizando \textit{multi-stage builds} para otimização das imagens. A implantação ocorre via Google Cloud Run, que oferece escala automática (inclusive a zero) e deploy simplificado a partir de um único Dockerfile.

O \textit{pipeline} de CI/CD, implementado com GitHub Actions, automatiza as etapas de \textit{build}, testes e \textit{deploy}:
\begin{enumerate}
    \item \textbf{Build}: Compilação dos binários Go e verificação de erros de sintaxe;
    \item \textbf{Test}: Execução de testes unitários e de integração, utilizando \textit{testcontainers} para instâncias efêmeras de PostgreSQL e Redis;
    \item \textbf{Publish}: Construção da imagem Docker e envio para o Google Container Registry;
    \item \textbf{Deploy}: Atualização automática do serviço no Google Cloud Run.
\end{enumerate}

\subsection{Algoritmo de Ranking}

O cálculo do \textit{score} de cada senador é inspirado no \textit{State Legislative Effectiveness Score} (SLES) de Volden e Wiseman \cite{volden2018legislative}, adaptado para o contexto brasileiro. Baseia-se em quatro critérios objetivos, ponderados conforme sua relevância para a fiscalização cidadã:

\begin{itemize}
    \item \textbf{Produtividade Legislativa (35\%)}: Capacidade de avançar proposições através do processo legislativo, considerando estágio de tramitação (apresentação, comissão, plenário, sanção) e tipo de proposição (PEC, PLP, PL), com multiplicadores para projetos de maior complexidade;
    \item \textbf{Presença em Votações (25\%)}: Percentual de participação nas votações nominais em plenário, excluindo períodos de licença oficial;
    \item \textbf{Economia na Cota (20\%)}: Relação entre o valor utilizado e o teto disponível da CEAPS;
    \item \textbf{Participação em Comissões (20\%)}: Envolvimento efetivo nas comissões permanentes e temporárias, com bonificação para cargos de presidência e vice-presidência.
\end{itemize}

Cada métrica é normalizada em uma escala de 0 a 100 antes do cálculo final, permitindo comparabilidade entre critérios de natureza distinta. A fórmula resultante é:

\[
Score = (\text{Produtividade} \times 0.35) + (\text{Presença} \times 0.25) + (\text{Economia} \times 0.20) + (\text{Comissões} \times 0.20)
\]

Os rankings são recalculados diariamente após a conclusão das tarefas de ingestão e armazenados em cache Redis, garantindo resposta imediata às consultas do \textit{frontend}.

