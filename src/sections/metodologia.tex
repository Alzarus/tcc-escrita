\section{Metodologia}
\label{sec:metodologia}

O desenvolvimento do \textit{Tô De Olho} enquadra-se no paradigma da \textit{Design Science Research} (DSR), abordagem metodológica adequada para pesquisas que visam a construção e avaliação de artefatos tecnológicos destinados a resolver problemas organizacionais identificados \cite{hevner2004design}. O problema abordado --- a fragmentação de dados públicos sobre a atuação de senadores federais em múltiplas APIs governamentais não integradas --- demanda a construção de um artefato de software capaz de consolidar, processar e apresentar essas informações de forma acessível ao cidadão.

Esta seção detalha a abordagem de desenvolvimento iterativo adotada, as fontes de dados governamentais integradas e a metodologia de avaliação de desempenho parlamentar inspirada no \textit{Legislative Effectiveness Score}.

\subsection{Abordagem de Desenvolvimento}

A natureza do projeto --- integração de múltiplas APIs governamentais com estruturas de dados heterogêneas e documentação variável --- demandou uma abordagem de desenvolvimento capaz de acomodar descobertas incrementais e ajustes frequentes de escopo. Metodologias tradicionais de desenvolvimento em cascata, que pressupõem requisitos estáveis e bem definidos desde o início, mostraram-se inadequadas para este contexto de exploração de APIs públicas com comportamentos nem sempre previsíveis.

Optou-se, portanto, por uma abordagem \textbf{iterativa e incremental}, na qual o trabalho foi organizado em ciclos de desenvolvimento focados em entregas funcionais. Cada ciclo produzia um incremento utilizável do sistema, permitindo validação contínua das funcionalidades implementadas e ajustes baseados nos aprendizados obtidos durante a integração com cada API.

O desenvolvimento contou com apoio de ferramentas de inteligência artificial generativa como assistentes de codificação, seguindo a tendência contemporânea de \textit{AI-assisted software development} \cite{peng2023impact}. Tais ferramentas, baseadas em modelos de linguagem de grande escala (LLMs), foram empregadas para aceleração de tarefas operacionais como geração de código \textit{boilerplate}, refatoração e \textit{debugging}. O uso de assistentes de IA em desenvolvimento de software representa uma evolução natural das ferramentas de produtividade, análoga à adoção de IDEs com \textit{autocomplete} e analisadores estáticos de código.

É fundamental distinguir entre \textbf{assistência operacional} e \textbf{autoria intelectual}. As decisões estruturantes do projeto --- arquitetura do sistema, design do algoritmo de ranking, seleção de fontes de dados, modelagem do domínio e interpretação de resultados --- foram integralmente concebidas, avaliadas e validadas pelo desenvolvedor. A ferramenta de IA atuou como acelerador de implementação, não como substituto do julgamento técnico.

A divisão do trabalho ocorreu em cinco fases principais:
\begin{enumerate}
    \item \textbf{Fundação}: Estruturação do projeto em Golang, implementação do cliente para a API Legislativa do Senado, criação das \textit{migrations} do banco de dados e configuração inicial do \textit{frontend} em Next.js utilizando Bun como \textit{runtime};
    \item \textbf{Ingestão de Dados}: Implementação do cliente para a API Administrativa, configuração do \textit{scheduler} para tarefas agendadas, coleta de votações nominais e carga de dados históricos;
    \item \textbf{Ranking e API}: Desenvolvimento do serviço de cálculo de rankings, criação dos \textit{endpoints} REST para consumo pelo \textit{frontend}, configuração do cache Redis e implementação de testes automatizados;
    \item \textbf{Frontend}: Desenvolvimento do \textit{dashboard} principal, interface de ranking interativo e páginas de perfil dos senadores;
    \item \textbf{Emendas e Polimento}: Integração com o Portal da Transparência para dados de emendas parlamentares, visualizações de dados e preparação para \textit{deploy}.
\end{enumerate}

\subsection{Fontes de Dados}

O sistema integra três fontes de dados oficiais, fundamentadas no arcabouço legal brasileiro de transparência pública. A Lei de Acesso à Informação (Lei n. 12.527/2011) estabelece como diretriz a ``disponibilização de informações em formatos abertos, estruturados e legíveis por máquina'' \cite{lai2011}, princípio que as APIs governamentais operacionalizam. Conforme demonstrado pela Operação Serenata de Amor, tecnologias desenvolvidas sobre esses dados abertos podem gerar valor público ao facilitar o controle social do gasto parlamentar \cite{albuquerque2018serenata}.

A seleção dos \textit{endpoints} seguiu três critérios: (i) \textbf{relevância} --- priorizando dados de gastos, votações e atuação legislativa; (ii) \textbf{confiabilidade} --- selecionando fontes com documentação oficial; e (iii) \textbf{viabilidade técnica} --- considerando formatos estruturados (JSON/XML) e limites de requisição adequados.

\subsubsection{API Legislativa do Senado}

Documentada em \url{legis.senado.leg.br/dadosabertos}, esta API RESTful fornece dados do processo legislativo. A URL base para requisições é \url{https://legis.senado.leg.br/dadosabertos}. Os dados são retornados em formato JSON, com suporte a paginação e limite de 10 requisições por segundo. Os principais recursos utilizados incluem: lista de senadores em exercício, detalhes biográficos, histórico de mandatos e licenças, votações nominais em plenário e comissões, orientação partidária, proposições de autoria, relatorias, participação em comissões e discursos.

\subsubsection{API Administrativa do Senado}

Documentada em \url{adm.senado.gov.br/adm-dadosabertos/swagger-ui}, esta interface disponibiliza dados financeiros e administrativos. Os recursos utilizados incluem: despesas da Cota Parlamentar (CEAPS) com detalhamento por fornecedor e tipo, opção por auxílio-moradia, escritórios de apoio, lista de servidores de gabinete, remunerações mensais e mapeamento de lotações.

\subsubsection{Portal da Transparência (CGU)}

A API do Portal da Transparência (\url{api.portaldatransparencia.gov.br}) fornece dados de emendas parlamentares e transferências federais. O acesso requer autenticação via chave de API no \textit{header} \texttt{chave-api-dados}. O filtro \texttt{tipoEmenda=Transferência Especial} permite identificar as ``emendas PIX'', modalidade de repasse que dispensa convênio.

\subsubsection{Limitações e Cobertura Temporal}

Cada fonte de dados apresenta limitações que impactam o escopo das análises:

\begin{itemize}
    \item \textbf{API Legislativa}: Votações nominais disponíveis apenas a partir de 2019 (legislatura 56). Limite de 10 requisições por segundo;
    \item \textbf{API Administrativa}: Despesas CEAPS disponíveis desde 2008 em CSV; API REST cobre apenas o ano corrente. Não há \textit{endpoint} para vincular servidores diretamente ao gabinete;
    \item \textbf{Portal da Transparência}: Emendas disponíveis a partir de 2015. Busca por autor utiliza nome textual, exigindo normalização de grafia. Limite de 300 requisições por minuto.
\end{itemize}

\subsection{Metodologia de Avaliação de Desempenho}

A avaliação objetiva do desempenho parlamentar é elemento central para a participação cidadã. Conforme a escada de participação de Arnstein \cite{arnstein1969ladder}, o acesso a informações claras é pré-requisito para que cidadãos avancem de níveis consultivos para controle social efetivo. A construção de rankings requer transparência metodológica \cite{hullman2011visualization}. Ressalta-se que o índice proposto não pretende mensurar ``qualidade moral'' ou ``mérito político'' subjetivo dos parlamentares, mas sim oferecer um instrumento comparativo fundamentado em critérios observáveis e quantificáveis de produtividade e gestão fiscal.

O cálculo do \textit{score} de cada senador é inspirado no \textit{Legislative Effectiveness Score} (LES), metodologia desenvolvida por Volden e Wiseman para avaliar a efetividade legislativa de parlamentares americanos \cite{volden2018legislative}. O LES define efetividade legislativa como ``a capacidade comprovada de avançar itens da agenda de um parlamentar através do processo legislativo até sua transformação em lei'' \cite{volden2014lawmakers}.

Para a adaptação brasileira, manteve-se a filosofia de valorizar o avanço de proposições através do processo legislativo, incorporando critérios adicionais relevantes ao contexto de fiscalização cidadã. O índice proposto compõe-se de quatro dimensões objetivamente mensuráveis:

\begin{table}[H]
\centering
\caption{Critérios do algoritmo de ranking}
\label{tab:criterios}
\begin{tabular}{|l|c|l|}
\hline
\textbf{Critério} & \textbf{Peso} & \textbf{Justificativa} \\
\hline
Produtividade Legislativa & 35\% & Core do LES: capacidade de aprovar proposições \\
Presença em Votações & 25\% & Compromisso efetivo com o mandato \\
Economia na Cota (CEAPS) & 20\% & Responsabilidade fiscal \\
Participação em Comissões & 20\% & Trabalho técnico especializado \\
\hline
\end{tabular}
\end{table}

\subsubsection{Produtividade Legislativa}

Este critério avalia o avanço de proposições de autoria do senador através do processo legislativo, atribuindo pontuação crescente por estágio alcançado (Apresentado: 1 ponto, Em Comissão: 2, Aprovado em Comissão: 4, Aprovado em Plenário: 8, Transformado em Lei: 16). Os multiplicadores refletem a maior complexidade de Propostas de Emenda Constitucional (PEC: $\times$3.0) e Projetos de Lei Complementar (PLP: $\times$2.0). Senadores que atuam como relatores recebem pontuação adicional.

\subsubsection{Presença em Votações}

Calculada como a razão entre votações participadas e votações disponíveis:

\begin{equation}
\text{Presença} = \frac{\text{Votações Participadas}}{\text{Votações Disponíveis}} \times 100
\end{equation}

As votações disponíveis excluem períodos de licença oficial (médica ou para cargo executivo), garantindo que afastamentos justificados não penalizem o senador. Obstrução \textbf{não} conta como presença.

\subsubsection{Economia na Cota Parlamentar}

Calculada como a proporção não utilizada do teto da CEAPS:

\begin{equation}
\text{Economia} = \left(1 - \frac{\text{Gasto Senador}}{\text{Teto CEAPS}}\right) \times 100
\end{equation}

O teto da CEAPS varia por UF (R\$ 26.000 para DF até R\$ 44.000+ para estados da região Norte). Senadores com gasto acima de 120\% do teto recebem pontuação zero neste critério.

\subsubsection{Participação em Comissões}

Pontuação atribuída conforme o nível de engajamento: membro titular (+2 pontos), suplente (+1), vice-presidente (+3), presidente (+5). Comissões estratégicas (CAE, CCJ) recebem multiplicador de $\times$1.5.

\subsubsection{Fórmula Final e Tratamento de Casos Especiais}

Cada métrica é normalizada de 0 a 100 antes da ponderação:

\begin{equation}
Score = (P_L \times 0.35) + (P_V \times 0.25) + (E_C \times 0.20) + (P_C \times 0.20)
\end{equation}

Onde:
\begin{itemize}
    \item $P_L$: Produtividade Legislativa;
    \item $P_V$: Presença em Votações;
    \item $E_C$: Economia na Cota;
    \item $P_C$: Participação em Comissões.
\end{itemize}

Casos especiais são tratados da seguinte forma: senadores novos (suplentes) requerem período mínimo de 30 dias para entrar no ranking; licenças curtas mantêm o senador no ranking com ajuste nos denominadores; licenças longas (>30 dias) congelam o score; dados indisponíveis causam reponderação proporcional dos demais critérios.
