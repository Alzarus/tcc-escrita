\section{Fundamentação Teórica}

Esta seção apresenta os conceitos fundamentais que embasam o desenvolvimento do \textit{Tô De Olho}, abrangendo transparência pública, democracia digital e arquitetura de sistemas distribuídos.

\subsection{Transparência Pública e Dados Abertos}

A transparência governamental constitui pilar fundamental do Estado Democrático de Direito. No Brasil, a Lei de Acesso à Informação (LAI – Lei nº 12.527/2011) estabelece que o acesso é a regra e o sigilo, a exceção, garantindo aos cidadãos o direito de solicitar e receber informações públicas sem necessidade de justificativa \cite{avelino2021democracia}.

A literatura distingue duas modalidades de transparência: a \textbf{transparência ativa}, na qual o Estado disponibiliza informações de forma proativa em portais e bases de dados; e a \textbf{transparência passiva}, que responde às solicitações dos cidadãos via canais específicos. O Portal de Dados Abertos do Senado Federal exemplifica a primeira modalidade, disponibilizando APIs e arquivos para consulta pública.

O conceito de \textit{Open Government Data} (Dados Governamentais Abertos) preconiza que as informações públicas devem ser disponibilizadas em formatos abertos, processáveis por máquina e livres de licenças restritivas. Tim Berners-Lee propôs uma escala de cinco estrelas para avaliar a qualidade dos dados abertos, sendo o nível máximo aquele em que os dados são linkados (\textit{Linked Open Data}), permitindo cruzamentos entre diferentes fontes \cite{avelino2021democracia}.

\subsection{Democracia Digital e Participação Cidadã}

O conceito de democracia digital refere-se ao emprego de tecnologias de informação e comunicação (TICs) para produzir ``mais democracia e melhores democracias'' \cite{Gomes2010}. Gomes identifica três fases históricas neste campo: a teledemocracia (anos 1970-90), marcada por experimentos com televisão interativa; a fase da internet (1995-2005), caracterizada pelo debate sobre potenciais e limites da rede; e a autonomização contemporânea, onde subtemas como governo aberto, \textit{smart cities} e parlamento digital desenvolvem-se de forma independente \cite{gomes2019}.

A participação cidadã mediada por tecnologia pode assumir diferentes níveis de profundidade. Sherry Arnstein, em seu trabalho seminal de 1969, propõe a ``Escada da Participação Cidadã'', uma tipologia de oito degraus que classifica o grau de poder real conferido aos cidadãos \cite{arnstein1969ladder}. Os degraus inferiores --- manipulação e terapia --- representam formas de \textbf{não-participação}, onde o objetivo é ``educar'' ou ``curar'' os participantes em vez de ouvi-los. Os degraus intermediários --- informação, consulta e pacificação --- constituem níveis de \textbf{participação simbólica}, nos quais cidadãos podem ouvir e ser ouvidos, mas sem garantia de que suas vozes influenciem decisões. Apenas nos degraus superiores --- parceria, delegação de poder e controle cidadão --- observa-se redistribuição efetiva de poder decisório.

Ferramentas de transparência como o \textit{Tô De Olho} situam-se primariamente no degrau da \textbf{informação}: proveem ao cidadão dados estruturados sobre a atuação parlamentar, condição necessária --- mas não suficiente --- para o exercício pleno da fiscalização. Como adverte Arnstein, ``participação sem redistribuição de poder é um processo vazio e frustrante para os desprovidos de poder'' \cite{arnstein1969ladder}. Reconhecendo essa limitação, o projeto busca ir além da mera disponibilização de dados, oferecendo \textit{rankings}, comparativos e visualizações que \textbf{empoderam} o cidadão para uma fiscalização mais qualificada.

No contexto brasileiro, Avelino et al. mapeiam iniciativas de governo aberto no âmbito federal, identificando avanços significativos na disponibilização de dados, porém alertando para desafios como a brecha digital e o analfabetismo funcional, que limitam o acesso efetivo da população às informações disponibilizadas \cite{avelino2021democracia}.

\subsection{Métricas de Efetividade Legislativa}

A avaliação quantitativa do desempenho parlamentar constitui tema relevante na ciência política contemporânea. Volden e Wiseman desenvolveram o \textit{Legislative Effectiveness Score} (LES), uma métrica que mensura a capacidade de parlamentares em conduzir suas proposições através do processo legislativo \cite{volden2018legislative}. O modelo considera múltiplas dimensões: quantidade de projetos apresentados, taxa de aprovação em comissões, progressão para votação em plenário e conversão em lei.

Os autores identificaram fatores correlacionados à maior efetividade: senioridade no mandato, posição em comissões estratégicas, pertencimento ao partido majoritário e experiência prévia em cargos legislativos estaduais. Embora desenvolvida para o contexto norte-americano, a metodologia oferece um \textit{framework} adaptável para avaliar parlamentares brasileiros.

No \textit{Tô De Olho}, o ``Score'' do senador inspira-se nesta abordagem, combinando indicadores objetivos como presença em votações, produtividade legislativa (proposições de autoria e relatorias), economia na utilização da cota parlamentar e participação em comissões. A transparência metodológica --- expor claramente os critérios e pesos utilizados --- é fundamental para que o \textit{ranking} seja percebido como ferramenta de informação, não de manipulação.

\subsection{Visualização de Dados e Retórica Visual}

A apresentação de dados ao cidadão não é neutra: escolhas de \textit{design} influenciam a interpretação das informações. Hullman e Diakopoulos investigaram os ``efeitos de enquadramento'' (\textit{framing effects}) em visualizações narrativas, demonstrando que técnicas retóricas como seleção, omissão, ênfase e sequenciamento podem direcionar a leitura do público \cite{hullman2011visualization}.

Os autores identificam quatro categorias de técnicas retóricas em visualizações: (1) \textbf{proveniência} --- identificar a origem e credibilidade dos dados; (2) \textbf{mapeamento visual} --- como elementos gráficos representam variáveis; (3) \textbf{anotações linguísticas} --- textos que guiam a interpretação; e (4) \textbf{interatividade} --- controles que permitem ao usuário explorar os dados por conta própria.

Para o \textit{Tô De Olho}, esses princípios orientam decisões de \textit{design}: exibir sempre a fonte oficial e data de atualização (\textbf{proveniência}); utilizar escalas consistentes em gráficos comparativos (\textbf{mapeamento}); contextualizar valores absolutos com médias e percentis (\textbf{anotação}); e permitir filtros por partido, estado e período (\textbf{interatividade}). O objetivo é maximizar a transparência metodológica, evitando que a plataforma seja percebida como veículo de viés político.

\subsection{Arquitetura de Software: Monolito Modular}

A arquitetura de \textbf{monolito modular} representa uma abordagem intermediária entre o monolito tradicional e os microsserviços, estruturando uma aplicação como módulos bem definidos dentro de um único artefato de deploy. Esta escolha é particularmente adequada para equipes pequenas e projetos acadêmicos, oferecendo benefícios de organização e manutenibilidade sem a complexidade operacional de sistemas distribuídos.

Os principais benefícios desta arquitetura incluem: simplicidade de deploy, com um único container em ambiente serverless (Cloud Run); baixa latência entre módulos, já que a comunicação ocorre via chamadas de função em memória; e facilidade de evolução, permitindo eventual migração para microsserviços se a escala justificar.

Para o \textit{Tô De Olho}, esta abordagem organiza a aplicação em módulos internos (ingestão, domínio, API) que podem ser desenvolvidos e testados de forma independente, mantendo a simplicidade operacional necessária para um projeto acadêmico com prazo definido.

\subsection{Engenharia de Dados: APIs e Processos ETL}

A estratégia de ingestão de dados do \textit{Tô De Olho} fundamenta-se no padrão ETL (\textit{Extract, Transform, Load}): extração dos dados brutos das fontes oficiais, transformação para normalização e enriquecimento, e carga no banco de dados da aplicação.

A abordagem adotada consome exclusivamente APIs RESTful oficiais: a API Legislativa do Senado (matérias, votações, comissões), a API Administrativa (CEAPS, remunerações de servidores) e a API do Portal da Transparência (emendas parlamentares). O processo combina \textit{backfill} para carga histórica e sincronização diária via tarefas agendadas.

\subsection{Emendas PIX e Desafios de Transparência}

As Transferências Especiais, popularmente conhecidas como ``emendas PIX'', constituem modalidade de repasse de recursos federais a estados e municípios sem necessidade de convênio, criadas pela Emenda Constitucional nº 105/2019. Alencar \cite{alencar2024emendaspix} demonstra sérias deficiências de transparência fiscal nesta modalidade: do total de R\$ 20,5 bilhões transferidos, apenas R\$ 933 milhões (menos de 5\%) tiveram prestação de contas adequada.

O autor identifica três vertentes de \textit{accountability} relevantes para a fiscalização parlamentar: a \textbf{accountability vertical}, exercida pelo eleitorado nas urnas; a \textbf{accountability horizontal}, desempenhada por instituições como tribunais de contas; e a \textbf{accountability social}, protagonizada por organizações da sociedade civil e imprensa. A fragmentação das fontes de dados dificulta todas essas formas de controle, evidenciando a necessidade de ferramentas que consolidem informações dispersas.
