\section{Fundamentação Teórica}

O desenvolvimento de uma plataforma de transparência parlamentar situa-se na interseção de múltiplos campos do conhecimento: direito à informação, ciência política, teoria democrática, design de interação e engenharia de software. Compreender essas bases teóricas é essencial não apenas para fundamentar as decisões de projeto, mas também para posicionar a ferramenta no debate mais amplo sobre controle social e qualidade da democracia.

Esta seção organiza-se em três eixos complementares. O primeiro eixo --- \textbf{normativo-institucional} --- examina os marcos legais e conceituais que sustentam a transparência governamental no Brasil, desde a Lei de Acesso à Informação até os compromissos internacionais de governo aberto. O segundo eixo --- \textbf{político-participativo} --- aborda as teorias de democracia digital e prestação de contas, investigando como tecnologias podem (ou não) fortalecer a participação cidadã e a responsabilização de agentes públicos. O terceiro eixo --- \textbf{técnico-metodológico} --- apresenta os fundamentos de mensuração de efetividade legislativa, visualização de dados e arquitetura de sistemas que orientam a implementação do \textit{Tô De Olho}.

%% ========================================
%% EIXO 1: NORMATIVO-INSTITUCIONAL
%% ========================================

\subsection{Eixo Normativo-Institucional}

Este eixo examina os fundamentos jurídicos e conceituais que regulamentam o acesso à informação pública no Brasil, bem como os desafios de transparência em modalidades orçamentárias específicas.

\subsubsection{Transparência Pública e Dados Abertos}

O sistema integra três fontes de dados oficiais, fundamentadas no arcabouço legal brasileiro de transparência pública. A Lei de Acesso à Informação (Lei n. 12.527/2011) estabelece como diretriz a ``disponibilização de informações em formatos abertos, estruturados e legíveis por máquina'' \cite{lai2011}, princípio que as APIs governamentais operacionalizam — como exemplificado pelo Portal de Dados Abertos do Senado \cite{dadosabertossenado2024}. Conforme demonstrado pela Operação Serenata de Amor, tecnologias desenvolvidas sobre esses dados abertos podem gerar valor público ao facilitar o controle social do gasto parlamentar \cite{albuquerque2018serenata}. A LAI impõe aos órgãos públicos o dever de divulgação proativa de informações de interesse coletivo, incluindo dados sobre despesas, contratos e remunerações de servidores.

A literatura distingue duas modalidades complementares de transparência. A \textbf{transparência ativa} ocorre quando o Estado disponibiliza informações de forma proativa em portais e bases de dados, independentemente de solicitação --- exemplificada pelos Portais de Dados Abertos do Senado Federal e da Câmara dos Deputados. A \textbf{transparência passiva}, por sua vez, responde às solicitações dos cidadãos via canais específicos como o Sistema Eletrônico do Serviço de Informação ao Cidadão (e-SIC). Enquanto a primeira amplia o acesso massivo a dados estruturados, a segunda garante o direito individual à informação específica \cite{avelino2021democracia}. Pesquisas de Michener \textit{et al.} (\citeyear{michener2018opacity}) demonstram que, cinco anos após a promulgação da LAI, a transparência permanece significativamente mais fraca nos níveis estadual e municipal, com índices de \textit{compliance} inferiores a 50\% em muitos entes federativos.

O conceito de \textit{Open Government Data} (Dados Governamentais Abertos) vai além da simples disponibilização: preconiza que as informações públicas devem ser liberadas em formatos abertos, processáveis por máquina e livres de licenças restritivas. Tim O'Reilly (\citeyear{oreilly2010government}), em ensaio seminal que cunhou o termo \textit{``Government as a Platform''}, argumenta que governos devem atuar como \textbf{plataformas} que habilitam a inovação cidadã, em vez de provedores diretos de serviços isolados. Nessa visão, APIs de dados abertos funcionam como infraestrutura sobre a qual a sociedade civil constrói ferramentas de fiscalização --- exatamente o que o \textit{Tô De Olho} realiza. Tim Berners-Lee (\citeyear{bernerslee2010linked}), inventor da \textit{World Wide Web}, complementa essa perspectiva ao propor uma escala de cinco estrelas para avaliar a qualidade dos dados abertos:

\begin{enumerate}
    \item \textbf{Uma estrela}: Dados disponíveis na web em qualquer formato, sob licença aberta (ex: PDF escaneado);
    \item \textbf{Duas estrelas}: Dados estruturados e legíveis por máquina (ex: planilha Excel);
    \item \textbf{Três estrelas}: Dados em formato aberto não-proprietário (ex: CSV ao invés de Excel);
    \item \textbf{Quatro estrelas}: Dados identificados por URIs, seguindo padrões W3C como RDF;
    \item \textbf{Cinco estrelas}: Dados linkados formando rede interoperável (\textit{Linked Open Data}).
\end{enumerate}

Os portais brasileiros de dados abertos situam-se predominantemente entre duas e três estrelas: oferecem arquivos CSV e JSON processáveis por máquina, mas raramente implementam identificadores únicos (URIs) ou linkagem semântica entre bases de dados de diferentes órgãos. A fragmentação identificada na Declaração do Problema --- três APIs de dois órgãos distintos --- exemplifica essa limitação: embora os dados sejam tecnicamente ``abertos'', a ausência de interoperabilidade impõe barreiras significativas à consolidação e análise integrada \cite{avelino2021democracia}.

O Brasil integra a \textit{Open Government Partnership} (OGP) desde 2011, tendo desenvolvido seis planos de ação nacionais com participação da sociedade civil. Essas iniciativas resultaram em 130 reformas voltadas à melhoria da governança e ao fortalecimento da Lei de Acesso à Informação \cite{ogp2024brazil}. Contudo, como demonstram as pesquisas sobre dados abertos, a mera disponibilização de informações não garante transparência efetiva: é necessário que os dados alcancem o público, que este tenha capacidade de processá-los, e que existam mecanismos institucionais para responsabilização dos agentes públicos.

\subsubsection{Emendas PIX e Desafios de Transparência Orçamentária}

As Transferências Especiais, popularmente conhecidas como ``emendas PIX'', constituem modalidade de repasse de recursos federais a estados e municípios criada pela Emenda Constitucional nº 105/2019. Diferentemente de convênios tradicionais, que exigem plano de trabalho prévio e prestação de contas detalhada, as emendas PIX transferem recursos diretamente às contas dos entes federativos com discricionariedade ampla sobre sua aplicação.

A pesquisa de Alencar (\citeyear{alencar2024emendaspix}) documenta deficiências graves de transparência fiscal nesta modalidade, com índices de prestação de contas inferiores a 5\% do volume total transferido (conforme dados apresentados na introdução deste trabalho).

A distribuição territorial revela disparidades extremas: no mesmo estado, alguns municípios receberam mais de R\$ 4.500 \textit{per capita} via emendas PIX, enquanto outros receberam menos de R\$ 1 --- sem qualquer justificativa pública dos parlamentares sobre os critérios de alocação. Esta opacidade compromete as três vertentes de prestação de contas discutidas posteriormente nesta seção: a \textbf{vertical} (eleitores não conseguem avaliar escolhas de seus representantes), a \textbf{horizontal} (tribunais de contas enfrentam dificuldades de fiscalização) e a \textbf{social} (jornalistas e pesquisadores encontram dados fragmentados e incompletos).

O \textit{Tô De Olho} aborda esta lacuna ao integrar dados de emendas do Portal da Transparência com informações legislativas do Senado, permitindo que o cidadão visualize, para cada senador: o total de recursos destinados via transferências especiais, os municípios beneficiados e a evolução temporal dos repasses. Ao consolidar informações dispersas em interface única, a plataforma contribui para reduzir a opacidade que caracteriza esta modalidade orçamentária.

%% ========================================
%% EIXO 2: POLÍTICO-PARTICIPATIVO
%% ========================================

\subsection{Eixo Político-Participativo}

Este eixo investiga as bases teóricas que fundamentam a relação entre tecnologia, participação cidadã e responsabilização de agentes públicos. Examina-se como ferramentas digitais podem fortalecer --- ou apenas simular --- o controle social.

\subsubsection{Democracia Digital e Participação Cidadã}

O conceito de democracia digital refere-se ao emprego de tecnologias de informação e comunicação (TICs) para produzir ``mais democracia e melhores democracias'' \cite{Gomes2010}. Wilson Gomes identifica três fases históricas neste campo: a \textbf{teledemocracia} (anos 1970-90), marcada por experimentos com televisão interativa e enquetes eletrônicas; a \textbf{fase da internet} (1995-2005), caracterizada pelo debate sobre potenciais e limites da rede para a participação política; e a \textbf{autonomização contemporânea}, onde subtemas como governo aberto, \textit{smart cities} e parlamento digital desenvolvem-se de forma independente, com metodologias e agendas próprias \cite{gomes2019}.

A participação cidadã mediada por tecnologia pode assumir diferentes níveis de profundidade e poder real. Sherry Arnstein, em seu trabalho seminal de 1969, propõe a ``Escada da Participação Cidadã'' (\textit{Ladder of Citizen Participation}), uma tipologia de oito degraus que classifica o grau de poder efetivamente conferido aos cidadãos \cite{arnstein1969ladder}:

\begin{itemize}
    \item \textbf{Não-participação} (degraus 1-2): \textit{Manipulação} e \textit{Terapia} --- formas em que o objetivo é ``educar'' ou ``curar'' os participantes, não ouvi-los. Comitês consultivos sem poder deliberativo exemplificam essa categoria.
    
    \item \textbf{Participação simbólica} (degraus 3-5): \textit{Informação}, \textit{Consulta} e \textit{Pacificação} --- cidadãos podem ouvir e ser ouvidos, mas sem garantia de que suas vozes influenciem decisões. Audiências públicas e pesquisas de opinião situam-se neste nível.
    
    \item \textbf{Poder cidadão} (degraus 6-8): \textit{Parceria}, \textit{Delegação de Poder} e \textit{Controle Cidadão} --- redistribuição efetiva de poder decisório. Orçamentos participativos vinculantes e conselhos com poder de veto exemplificam esses degraus superiores.
\end{itemize}

Ferramentas de transparência como o \textit{Tô De Olho} situam-se primariamente no degrau da \textbf{informação}: proveem ao cidadão dados estruturados sobre a atuação parlamentar, condição necessária --- mas não suficiente --- para o exercício pleno da fiscalização. Como adverte Arnstein (\citeyear{arnstein1969ladder}), ``participação sem redistribuição de poder é um processo vazio e frustrante para os desprovidos de poder''. Reconhecendo essa limitação estrutural, o projeto busca ir além da mera disponibilização de dados: ao oferecer \textit{rankings}, comparativos e visualizações contextualizadas, a plataforma \textbf{empodera} o cidadão para uma fiscalização mais qualificada, fornecendo-lhe ferramentas para exercer pressão informada sobre seus representantes.

No contexto brasileiro, a brecha digital representa obstáculo adicional à democracia digital. Autores como Saikali (\citeyear{Saikali2021}) e Lemos (\citeyear{lemos2013cidades}) alertam que iniciativas de ``Cidades Inteligentes'' e ferramentas digitais correm o risco de reproduzir desigualdades sociais se não forem desenhadas com foco explícito em inclusão. Farias (\citeyear{farias2013possibilidades}) argumenta que as possibilidades democráticas da internet dependem crucialmente de desenhos institucionais que facilitem a apropriação tecnológica por camadas populares. Avelino et al. (\citeyear{avelino2021democracia}) mapeiam iniciativas de governo aberto no âmbito federal, identificando avanços significativos na disponibilização de dados, porém reforçam que tecnologias precisam ser mediadas por visualizações claras e linguagem acessível para efetivação do controle social.

\subsubsection{Teoria Principal-Agente e Accountability}

A relação entre cidadãos e representantes eleitos pode ser analisada através da lente da \textbf{teoria econômica de agência}, originalmente desenvolvida para compreender relações contratuais em organizações. Neste modelo, os cidadãos (eleitores) atuam como \textbf{principais} que delegam autoridade a \textbf{agentes} (parlamentares e burocratas) para tomar decisões em seu nome. O problema fundamental surge da \textbf{assimetria informacional}: os agentes possuem mais informação sobre suas próprias ações, esforços e competências do que os principais que os monitoram \cite{moe1984new}.

Esta assimetria manifesta-se de duas formas principais. O \textbf{risco moral} (\textit{moral hazard}) ocorre quando os principais não conseguem observar plenamente as ações dos agentes após a delegação: um parlamentar pode ``enrolar'' em suas funções, ausentar-se de votações ou priorizar interesses particulares sem que o eleitorado perceba. A \textbf{seleção adversa} (\textit{adverse selection}) surge antes da delegação: candidatos podem exagerar suas qualificações ou ocultar incompetências durante campanhas eleitorais. Em ambos os casos, a divergência entre os interesses do principal e do agente gera ``custos de agência'' --- ineficiências, corrupção ou políticas distorcidas.

O conceito de \textbf{accountability} --- frequentemente traduzido como ``prestação de contas'' ou ``responsabilização'' --- emerge como mecanismo para mitigar esses problemas. A literatura distingue três vertentes complementares \cite{alencar2024emendaspix}:

\begin{itemize}
    \item \textbf{Accountability vertical}: Exercida pelos eleitores através do voto. Parlamentares que não atendem às expectativas podem ser punidos nas urnas. Contudo, eleições são instrumentos ``grosseiros'': ocorrem periodicamente (a cada 4-8 anos para senadores), envolvem múltiplas dimensões de avaliação simultâneas, e dependem de que o eleitor tenha informação suficiente para julgar o desempenho.
    
    \item \textbf{Accountability horizontal}: Exercida por instituições de controle como tribunais de contas, controladorias, Ministério Público e o próprio Poder Judiciário. Esses órgãos possuem competência técnica e acesso privilegiado a informações, mas enfrentam limitações de capacidade operacional diante do volume de atos a fiscalizar.
    
    \item \textbf{Accountability social}: Exercida por organizações da sociedade civil, imprensa investigativa, acadêmicos e cidadãos organizados. Esta modalidade complementa as anteriores ao ampliar a capacidade de monitoramento e pressionar por transparência.
\end{itemize}

A transparência é condição necessária --- mas não suficiente --- para a accountability. Pesquisas demonstram que para a divulgação de informações traduzir-se em responsabilização efetiva, três condições devem ser satisfeitas: (1) a informação deve efetivamente alcançar o público relevante; (2) o público deve ter capacidade de processá-la e reagir a ela; e (3) devem existir mecanismos institucionais que permitam consequências para os agentes. A simples disponibilização de dados brutos --- o que alguns autores chamam de ``governo nu'' (\textit{naked government}) --- pode até amplificar percepções negativas sem gerar accountability efetiva.

O \textit{Tô De Olho} posiciona-se como ferramenta de \textbf{accountability social}, reduzindo a assimetria informacional entre eleitores e senadores. Ao consolidar dados fragmentados, contextualizar valores com médias comparativas e oferecer \textit{rankings} metodologicamente fundamentados, a plataforma amplia a capacidade do cidadão de monitorar seus representantes --- condição prévia para o exercício tanto da accountability vertical (voto informado) quanto da pressão por accountability horizontal (denúncias fundamentadas a órgãos de controle). Esta abordagem alinha-se aos princípios de governo aberto defendidos pela Open Government Partnership, que destacam a importância de intermediários tecnológicos para traduzir dados brutos em inteligência cívica acionável \cite{ogp2022parliament, verhulst2016open}.

\subsubsection{Civic Tech e Sociedade Civil}

O termo \textit{civic tech} (tecnologia cívica) refere-se ao uso de tecnologias digitais para fortalecer a participação cidadã, a transparência governamental e a colaboração entre sociedade e Estado. Diferencia-se de \textit{GovTech} (tecnologia para eficiência governamental interna) por seu foco na interface com o cidadão e no empoderamento da sociedade civil. Lathrop e Ruma, em coletânea seminal sobre governo aberto, reúnem contribuições de pioneiros como Tim O'Reilly, Beth Noveck e Aaron Swartz, estabelecendo três pilares para a \textit{civic tech}: \textbf{transparência} (dados abertos), \textbf{participação} (engajamento cidadão) e \textbf{colaboração} (parcerias entre governo e sociedade) \cite{lathrop2010open}.

No Brasil, o ecossistema de \textit{civic tech} consolidou-se a partir de 2010, impulsionado pela Lei de Acesso à Informação (2011) e pelo crescimento de organizações especializadas. Entre as iniciativas mais relevantes:

\begin{itemize}
    \item \textbf{Transparência Brasil}: Fundada em 2000, é uma das principais organizações dedicadas à promoção da integridade e supervisão cívica no setor público. Desenvolve pesquisas, relatórios e ferramentas para monitoramento de políticas públicas \cite{transparenciabrasil2024}.
    
    \item \textbf{Open Knowledge Brasil}: Responsável pela Operação Serenata de Amor e pelo projeto ``Querido Diário'', que aplica técnicas de inteligência artificial para auditar diários oficiais municipais.
    
    \item \textbf{Fiquem Sabendo}: Organização sem fins lucrativos que utiliza \textit{civic tech} de código aberto para expor gastos governamentais não divulgados, tendo revelado mais de 500 bilhões de reais em despesas não reportadas ao longo de 27 anos.
    
    \item \textbf{Associação Brasileira de Jornalismo Investigativo (Abraji)}: Emprega jornalismo de dados e ferramentas tecnológicas para investigar corrupção e má gestão de recursos públicos.
\end{itemize}

O cenário internacional oferece referências consolidadas que inspiram o movimento brasileiro. No Reino Unido, a plataforma \textit{TheyWorkForYou} \cite{theyworkforyou2024} é pioneira em simplificar a linguagem parlamentar para o cidadão comum. No Brasil, além das iniciativas citadas, destacam-se a \textit{Base dos Dados} \cite{basedosdados2024}, que universaliza o acesso a dados públicos tratados, e o \textit{Colab} \cite{colab2024}, focado na gestão participativa municipal. O Portal Brasileiro de Dados Abertos \cite{dadosgov2024}, embora institucional, fornece a infraestrutura de dados brutos que alimenta muitas dessas aplicações.

Apesar dos avanços, o ecossistema enfrenta desafios significativos. A \textbf{sustentabilidade financeira} é precária: muitas iniciativas dependem de financiamento coletivo, bolsas internacionais ou trabalho voluntário, limitando sua capacidade de operação contínua. A \textbf{brecha digital} restringe o alcance a populações com menor acesso à internet ou habilidades tecnológicas limitadas. E a \textbf{fragmentação de esforços} --- múltiplas ferramentas com escopos parcialmente sobrepostos --- dispersa recursos e dificulta o engajamento do cidadão comum.

O \textit{Tô De Olho} posiciona-se neste ecossistema com um diferencial claro: o foco exclusivo no \textbf{Senado Federal}, câmara legislativa até então carente de ferramentas específicas de fiscalização cidadã. Enquanto a Câmara dos Deputados conta com ao menos três plataformas consolidadas (Serenata de Amor, De Olho no Congresso, De Olho em Você), os 81 senadores --- que exercem mandatos de oito anos e detêm competências exclusivas de alto impacto --- permanecem em relativa ``sombra'' digital.

%% ========================================
%% EIXO 3: TÉCNICO-METODOLÓGICO
%% ========================================

\subsection{Eixo Técnico-Metodológico}

Este eixo expõe fundamentos técnicos e metodológicos que orientam a implementação do \textit{Tô De Olho}. Aborda avaliação parlamentar, visualização de dados e arquitetura de software.

\subsubsection{Métricas de Efetividade Legislativa}

A avaliação quantitativa do desempenho parlamentar constitui tema relevante na ciência política contemporânea. Craig Volden e Alan E. Wiseman (\citeyear{volden2014lawmakers, volden2024legislative}), co-diretores do \textit{Center for Effective Lawmaking}, desenvolveram o \textit{Legislative Effectiveness \textit{Score}} (LES), uma métrica que mensura a capacidade de parlamentares em conduzir suas proposições através do processo legislativo.

A metodologia do LES fundamenta-se em cinco estágios do processo legislativo, cada qual representando um grau crescente de sucesso na agenda do parlamentar:

\begin{enumerate}
    \item \textbf{Introdução}: O projeto é formalmente apresentado;
    \item \textbf{Ação em comissão}: O projeto recebe parecer ou é debatido em comissão temática;
    \item \textbf{Votação em plenário (câmara de origem)}: O projeto é levado à votação na casa onde foi apresentado;
    \item \textbf{Aprovação na câmara de origem}: O projeto é aprovado e segue para a outra casa;
    \item \textbf{Conversão em lei}: O projeto completa a tramitação e é sancionado.
\end{enumerate}

Crucialmente, nem todos os projetos possuem igual peso na metodologia. Volden e Wiseman (\citeyear{thelawmakers2024}) categorizam as proposições em três níveis de significância: \textbf{projetos comemorativos} (como denominação de logradouros), que recebem peso mínimo; \textbf{projetos substantivos}, que alteram políticas públicas de forma moderada; e \textbf{projetos substantivos e significativos}, que promovem mudanças estruturais relevantes. Esta ponderação evita que parlamentares ``inflem'' suas estatísticas com proposições triviais.

Os autores identificaram fatores consistentemente correlacionados à maior efetividade legislativa: senioridade no mandato, ocupação de posições em comissões estratégicas (especialmente presidências e relatorias), pertencimento ao partido majoritário e experiência prévia em legislaturas estaduais. Interessantemente, pesquisas subsequentes demonstraram que, controlando demais variáveis, parlamentares mulheres tendem a ser mais efetivas que seus colegas homens \cite{volden2014lawmakers, volden2024legislative}.

Embora desenvolvida para o contexto norte-americano, a metodologia oferece um \textit{framework} adaptável para avaliar parlamentares brasileiros. No \textit{Tô De Olho}, o ``\textit{Score}'' do senador inspira-se nesta abordagem, combinando quatro dimensões com pesos públicos:

\begin{itemize}
    \item \textbf{Produtividade Legislativa (35\%)}: Avalia proposições de autoria e relatorias, com multiplicadores por tipo (PEC: 3x, PLP: 2x) e estágio de tramitação alcançado;
    \item \textbf{Presença em Votações (25\%)}: Mensura comparecimento efetivo às sessões deliberativas;
    \item \textbf{Economia na Cota Parlamentar (20\%)}: Compara uso individual da CEAPS com a mediana do Senado;
    \item \textbf{Participação em Comissões (20\%)}: Pondera engajamento em comissões, com bônus para cargos de liderança.
\end{itemize}

A transparência metodológica --- expor publicamente os critérios e pesos utilizados --- é fundamental para que o \textit{ranking} seja percebido como ferramenta de informação, não de manipulação política. Diferente de classificações opacas, o cidadão poderá compreender --- e questionar --- os fundamentos da avaliação.

\subsubsection{Visualização de Dados e Retórica Visual}

A apresentação de dados ao cidadão não é neutra: escolhas de \textit{design} influenciam profundamente a interpretação das informações. Jessica Hullman e Nicholas Diakopoulos investigaram os ``efeitos de enquadramento'' (\textit{framing effects}) em visualizações narrativas, demonstrando que técnicas retóricas como seleção, omissão, ênfase e sequenciamento podem direcionar a leitura do público de forma consciente ou inconsciente \cite{hullman2011visualization}.

Os autores identificam quatro categorias de técnicas retóricas em visualizações de dados:

\begin{enumerate}
    \item \textbf{Proveniência}: Identificação da origem e credibilidade dos dados. Visualizações que ocultam fontes ou datas de atualização comprometem a verificabilidade. No \textit{Tô De Olho}, cada gráfico exibe a fonte oficial (API do Senado ou Portal da Transparência) e a data da última sincronização.
    
    \item \textbf{Mapeamento visual}: Como elementos gráficos representam variáveis numéricas. Escalas inconsistentes, truncamento de eixos ou escolhas de cores podem distorcer percepções. A plataforma adota escalas consistentes em gráficos comparativos e paletas de cores acessíveis.
    
    \item \textbf{Anotações linguísticas}: Textos, títulos e legendas que guiam a interpretação. Valores absolutos sem contexto podem induzir conclusões equivocadas (ex: ``Senador X gastou R\$ 100 mil'' parece muito sem saber que a média é R\$ 150 mil). O \textit{Tô De Olho} contextualiza valores com médias e percentis.
    
    \item \textbf{Interatividade}: Controles que permitem ao usuário explorar dados por conta própria reduzem a dependência de narrativas pré-construídas. A plataforma oferece filtros por partido, estado e período, permitindo análises personalizadas.
\end{enumerate}

A \textbf{literacia em visualização de dados} (\textit{Data Visualization Literacy} --- DVL) refere-se à capacidade de interpretar corretamente representações visuais de informações. Pesquisas demonstram que mesmo populações com alta escolaridade frequentemente cometem erros de interpretação em gráficos aparentemente simples. No contexto brasileiro, onde 29\% da população é funcionalmente analfabeta \cite{inaf2024}, o desafio é ainda maior: visualizações complexas podem excluir justamente os cidadãos mais vulneráveis à falta de transparência.

Para o \textit{Tô De Olho}, esses princípios orientam decisões de \textit{design}: priorizar visualizações simples e intuitivas; oferecer múltiplas formas de apresentação (gráficos, tabelas, textos explicativos); e testar a compreensibilidade com usuários de diferentes perfis. O objetivo é maximizar a transparência metodológica, evitando que a plataforma seja percebida como veículo de viés político.

\subsubsection{Arquitetura de Software: Monolito Modular}

A escolha arquitetural de um sistema de software envolve \textit{trade-offs} fundamentais entre complexidade operacional, velocidade de desenvolvimento e capacidade de evolução. A literatura distingue três abordagens principais: o \textbf{monolito tradicional}, onde todos os componentes residem em uma única base de código sem separação clara; os \textbf{microsserviços}, que fragmentam a aplicação em serviços independentes comunicando-se via rede; e o \textbf{monolito modular}, que combina a simplicidade operacional do primeiro com a organização interna do segundo \cite{dragoni2017microservices, su2024modularmonolith}.

O monolito modular organiza o código em módulos discretos, fracamente acoplados e orientados a domínio, porém dentro de uma única unidade de \textit{deploy}. Diferente do monolito tradicional, que tende a se tornar uma ``bola de lama'' (\textit{big ball of mud}) com o crescimento, o monolito modular impõe fronteiras lógicas explícitas entre componentes, facilitando a manutenção e permitindo que equipes trabalhem de forma relativamente independente em diferentes partes do sistema \cite{su2024modularmonolith}.

A escolha por monolito modular no \textit{Tô De Olho} fundamenta-se em critérios técnicos e contextuais:

\begin{itemize}
    \item \textbf{Simplicidade de \textit{Deploy}}: Um único contêiner \textit{Docker} simplifica a infraestrutura e reduz custos operacionais. Plataformas \textit{serverless} como \textit{Google \textit{Cloud Run}} beneficiam-se particularmente desta característica, oferecendo escala automática (inclusive a zero) sem a complexidade de orquestração de múltiplos serviços;
    
    \item \textbf{Ausência de Latência de Rede}: A comunicação entre módulos ocorre via chamadas de função em memória, eliminando a latência e os riscos de falha associados a chamadas de rede. Em microsserviços, cada interação entre componentes adiciona \textit{overhead} de serialização, transmissão e desserialização;
    
    \item \textbf{Consistência Transacional}: Operações que envolvem múltiplos módulos podem compartilhar uma única transação de banco de dados, garantindo atomicidade. Em arquiteturas distribuídas, esse padrão requereria implementação de \textit{sagas} ou \textit{two-phase \textit{commit}}, aumentando significativamente a complexidade;
    
    \item \textbf{Adequação ao Contexto}: Para equipes pequenas e projetos acadêmicos, a complexidade operacional de microsserviços --- monitoramento distribuído, orquestração de contêineres, \textit{service discovery}, gerenciamento de configurações --- frequentemente supera os benefícios de escalabilidade granular.
\end{itemize}

A literatura recente sobre monolito modular destaca a importância de princípios do \textbf{Domain-Driven Design} (DDD) na definição de fronteiras entre módulos. Eric Evans, criador do DDD, propõe o conceito de \textit{Bounded Contexts} --- regiões do sistema onde um modelo de domínio específico se aplica, com interfaces bem definidas para comunicação com outros contextos \cite{evans2003ddd}. No \textit{Tô De Olho}, cada módulo interno (\texttt{senador}, \texttt{ceaps}, \texttt{votacao}, \texttt{emenda}, \texttt{\textit{ranking}}) representa um \textit{bounded context} com responsabilidades claras.

\textit{Frameworks} modernos têm formalizado o padrão de monolito modular. O \textit{Google Service Weaver} permite escrever aplicações como monolitos modulares que podem ser implantados como processos únicos ou microsserviços distribuídos, sem alteração de código \cite{serviceweaver2024}. O \textit{Spring Modulith} oferece suporte similar no ecossistema \textit{Java}, com verificação automatizada de fronteiras entre módulos \cite{springmodulith2024}. Estudos recentes indicam que esta abordagem tem ganhado tração na indústria como alternativa pragmática à complexidade prematura de microsserviços \cite{su2024modularmonolith}. Embora o \textit{Tô De Olho} não utilize esses \textit{frameworks} específicos (dado o \textit{backend} em \textit{Go}), a arquitetura interna segue os mesmos princípios: módulos com interfaces explícitas, dependências unidirecionais e comunicação via contratos bem definidos.

Esta escolha não descarta a possibilidade de evolução futura. Caso o sistema alcance escala que justifique escalabilidade granular de componentes específicos, a estrutura modular facilitará a extração de microsserviços independentes --- uma estratégia de ``decomposição incremental'' recomendada pela literatura como alternativa ao risco de migração completa \cite{dragoni2017microservices}.

\subsubsection{Engenharia de Dados: APIs e Processos ETL}
\label{sec:etl-fundamentacao}

O padrão ETL (\textit{Extract, Transform, Load}) constitui a espinha dorsal de sistemas de integração de dados desde os primórdios da computação analítica. Ralph Kimball (\citeyear{kimball2004etl}), pioneiro em \textit{data warehousing}, define ETL como o processo responsável por ``extrair dados de sistemas fonte, aplicar transformações de qualidade e conformidade, e entregar os dados em formato dimensional''. Vassiliadis (\citeyear{vassiliadis2009survey}), em revisão abrangente da literatura, estima que processos ETL consomem entre 60\% e 80\% do esforço de desenvolvimento de projetos de integração de dados.

A estratégia de ingestão do \textit{Tô De Olho} segue as três fases clássicas do ETL:

\begin{enumerate}
    \item \textbf{Extração (Extract)}: Coleta de dados brutos das fontes oficiais via APIs RESTful. O sistema consome três APIs distintas: a API Legislativa do Senado (matérias, votações nominais, comissões), a API Administrativa do Senado (CEAPS, remunerações de gabinete) e a API do Portal da Transparência (emendas parlamentares). Cada fonte possui características próprias de paginação, limites de requisição e formatos de resposta, exigindo adaptadores específicos;
    
    \item \textbf{Transformação (Transform)}: Normalização, limpeza e enriquecimento dos dados extraídos. Esta fase inclui: padronização de nomes de senadores (tratamento de variações como ``José da Silva'' vs ``JOSE DA SILVA''), conversão de formatos de data, cálculo de métricas derivadas (como percentuais de presença) e detecção de anomalias (valores negativos, datas futuras, registros duplicados);
    
    \item \textbf{Carga (Load)}: Persistência dos dados transformados no banco \textit{PostgreSQL}, escolhido por sua robustez e suporte nativo a tipos JSONB, facilitando o armazenamento de estruturas semi-estruturadas das APIs \cite{postgresql2024}. Utilizam-se estratégias de upsert para evitar duplicação e manutenção de histórico de alterações.
\end{enumerate}

O processo combina duas estratégias complementares de carga:

\begin{itemize}
    \item \textbf{\textit{Backfill} (Carga Histórica)}: Execução única para ingestão do histórico completo desde um ano configurável (padrão: 2019). Esta flexibilidade, implementada via variável de ambiente, permite ajustar o escopo temporal conforme necessidades específicas --- por exemplo, restringir a dados de 2023 em ambiente de desenvolvimento para acelerar testes, ou expandir para 2015 para análises de longo prazo;
    
    \item \textbf{Sincronização Incremental}: Tarefas agendadas (\textit{CronJobs}) executadas diariamente às 03:00 BRT, capturando apenas atualizações desde a última sincronização. Esta abordagem otimiza o consumo de recursos e respeita os limites de requisição das APIs oficiais.
\end{itemize}

A escolha por APIs oficiais --- em detrimento de técnicas de \textit{web scraping} --- fundamenta-se em critérios de confiabilidade e manutenibilidade. APIs possuem contratos documentados, formatos estruturados (JSON/XML) e versionamento explícito, enquanto o \textit{scraping} de páginas HTML é intrinsecamente frágil a mudanças de layout. Kimball adverte que ``a qualidade dos dados de entrada determina o teto de qualidade do sistema final'' \cite{kimball2013toolkit} --- premissa que reforça a preferência por fontes oficiais com garantias de integridade.

Adicionalmente, as APIs oficiais brasileiras oferecem vantagens práticas: o Portal da Transparência disponibiliza dados de emendas com granularidade diária; a API Legislativa do Senado permite consultas por período específico; e a API Administrativa provê arquivos CSV consolidados para carga em massa. Essas características distintas exigem que o módulo ETL do \textit{Tô De Olho} implemente adaptadores especializados para cada fonte, unificando os dados em um modelo dimensional coerente antes da carga.

