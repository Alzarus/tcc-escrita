\section{Implantação e Infraestrutura}
\label{sec:implantacao}

A arquitetura de implantação do \textit{Tô De Olho} foi projetada para garantir alta disponibilidade, escalabilidade elástica e baixo custo operacional, alinhando-se aos princípios da metodologia \textit{Twelve-Factor App} \cite{wiggins2017twelvefactor}. Balalaie \textit{et al.} argumentam que a adoção de arquiteturas orientadas a serviços e práticas de \textit{DevOps} é crucial para permitir o desenvolvimento ágil e a entrega contínua de valor em sistemas complexos \cite{balalaie2016microservices}. Seguindo essa premissa, a infraestrutura baseia-se integralmente em serviços gerenciados na nuvem (\textit{Cloud Native}), eliminando a necessidade de provisionamento e manutenção de servidores físicos ou virtuais tradicionais.



\subsection{Containerização}

A portabilidade da aplicação é assegurada pelo uso de contêineres \textit{Docker}. O processo de construção das imagens utiliza a técnica de \textit{Multi-Stage Builds}, recomendada pelas melhores práticas de segurança \cite{googlecontainers2024}. Esta abordagem divide o processo em dois estágios:

\begin{enumerate}
    \item \textbf{Estágio de Construção}: Utiliza uma imagem base completa (\texttt{golang:1.21-alpine}) contendo compiladores e ferramentas de construção necessárias para gerar o binário estático da aplicação.
    \item \textbf{Estágio de Execução}: Copia apenas o binário compilado para uma imagem minimalista (\texttt{gcr.io/distroless/static}), isenta de gerenciadores de pacotes ou \textit{shell}.
\end{enumerate}

O resultado são imagens finais extremamente leves (aproximadamente 25MB) e com superfície de ataque reduzida, visto que não contêm ferramentas que poderiam ser exploradas por atacantes.

\subsection{\textit{Pipeline} de Integração e Entrega Contínua (CI/CD)}

A automação do ciclo de vida do software é gerenciada via \textit{GitHub Actions}, configurada para executar \textit{pipelines} distintos baseados em eventos do repositório. A adoção de CI/CD permite reduzir o tempo de \textit{feedback} e mitigar erros humanos em processos manuais de \textit{deploy} \cite{kinsman2021software}. O fluxo principal (\textit{pipeline} de produção) é composto pelas seguintes etapas:

\begin{itemize}
    \item \textbf{Verificação (\textit{Linting} \& \textit{Vet})}: Análise estática do código para garantir conformidade com os padrões de estilo da linguagem \textit{Go} e detecção prévia de construções suspeitas.
    \item \textbf{Testes Automatizados}: Execução de todas as suítes de testes unitários e de integração. A falha em qualquer teste bloqueia imediatamente o processo de entrega.
    \item \textbf{Construção e Publicação (\textit{build} e \textit{push})}: Construção da imagem \textit{Docker} e publicação no \textit{Google Artifact Registry}, versionada com o \textit{hash} do \textit{commit} (\textit{SHA}).
    \item \textbf{Implantação (\textit{deploy})}: Atualização do serviço no \textit{Google Cloud Run}, utilizando a estratégia de atualização gradual (\textit{rolling update}) para garantir tempo de inatividade zero (\textit{zero downtime}) durante a transição de versões.
\end{itemize}

\subsection{Infraestrutura \textit{Serverless} no \textit{Google Cloud}}

O ambiente de execução principal é o \textbf{\textit{Google \textit{Cloud Run}}}, uma plataforma de computação \textit{serverless} que abstrai a complexidade de gerenciamento de infraestrutura \textit{Kubernetes} \cite{cloudrun2024}. Esta escolha arquitetural oferece vantagens estratégicas para o projeto:

\begin{itemize}
    \item \textbf{Escalabilidade Automática}: O serviço ajusta automaticamente o número de instâncias com base no tráfego de requisições, podendo escalar a zero (\textit{shutdown} total) em momentos de inatividade, otimizando custos.
    \item \textbf{Resiliência e Graceful Shutdown}: A aplicação implementa rotinas de desligamento gracioso para lidar com os sinais de interrupção (\texttt{SIGTERM}) da plataforma. Isso garante que requisições em andamento sejam finalizadas e conexões com o banco de dados sejam encerradas corretamente antes da destruição do contêiner \cite{nygard2018release}.
\end{itemize}

Para mitigar o problema de partida a frio (\textit{Cold Start}) --- latência inicial na subida de novas instâncias \cite{vahidinia2020coldstart}, característico de arquiteturas \textit{serverless} ---, utilizam-se otimizações como a inicialização tardia (\textit{lazy initialization}) de conexões pesadas e configurações de instâncias mínimas (\textit{min-instances}) em períodos críticos. Adicionalmente, estratégias de resiliência como Espera Exponencial (\textit{Exponential Backoff}) com Variação Aleatória (\textit{Jitter}) são empregadas na comunicação entre serviços para evitar o efeito de estouro de manada (\textit{thundering herd}) em caso de falhas transientes, conforme recomendado pela arquitetura de referência da \textit{AWS} \cite{aws2015backoff}. Para evoluções futuras, planeja-se a implementação do padrão Disjuntor (\textit{Circuit Breaker}) \cite{montesi2016circuit} e o aprofundamento da observabilidade distribuída baseada nos conceitos de \textit{Dapper} \cite{sigelman2010dapper} e nos três pilares de monitoramento modernos \cite{sridharan2018observability}.

\subsection{Estratégia \textit{Mobile}-First e Acessibilidade}

A decisão de priorizar a experiência em dispositivos móveis (\textit{Mobile-First}) fundamenta-se na realidade do acesso à internet no Brasil. Dados da PNAD Contínua TIC 2024 revelam que o celular é o dispositivo de acesso exclusivo para a grande maioria dos internautas brasileiros, atingindo 99\% de penetração nas classes D e E \cite{pnad2024}. Portanto, garantir uma interface leve, responsiva e performática em conexões móveis não é apenas uma escolha técnica de \textit{front-end}, mas um requisito de inclusão digital indispensável para uma ferramenta de controle social.

Para assegurar a acessibilidade, o projeto segue as diretrizes da WCAG 2.1 (Web Content Accessibility Guidelines) nível AA. As implementações técnicas incluem:

\begin{itemize}
    \item \textbf{Semântica HTML}: Uso rigoroso de tags semânticas (\texttt{<main>}, \texttt{<nav>}, \texttt{<article>}) e atributos ARIA onde necessário, facilitando a navegação por leitores de tela.
    \item \textbf{Tipografia Legível}: Adoção da fonte Inter (Google Fonts) com pesos e contrastes otimizados para leitura em telas pequenas.
    \item \textbf{Internacionalização}: Configuração explícita do atributo \texttt{lang="pt-BR"} em todas as páginas, garantindo a correta pronúncia por assistentes de voz.
    \item \textbf{Design Adaptativo}: Utilização do Tailwind CSS para criar layouts fluidos que se adaptam desde dispositivos móveis (320px) até monitores 4K, sem perda de funcionalidade.
\end{itemize}
