\documentclass[conference]{IEEEtran}
\IEEEoverridecommandlockouts
\usepackage{cite}
\usepackage{amsmath,amssymb,amsfonts}
\usepackage{algorithmic}
\usepackage{indentfirst}
\usepackage{graphicx}
\usepackage{textcomp}
\usepackage{xcolor}
\usepackage{url}
\bibliographystyle{IEEEtran}
\def\BibTeX{{\rm B\kern-.05em{\sc i\kern-.025em b}\kern-.08em
    T\kern-.1667em\lower.7ex\hbox{E}\kern-.125emX}}
\begin{document}

\title{Tô De Olho: Democratizando o Acesso à Transparência Política em Salvador}

\author{
\IEEEauthorblockN{Pedro Batista de Almeida Filho}
\IEEEauthorblockA{Instituto Federal da Bahia \\
Salvador, Bahia, Brasil \\
pedro.baf25@gmail.com} \\
\and
\IEEEauthorblockN{Pablo Vieira Florentino}
\IEEEauthorblockA{Instituto Federal da Bahia \\
Salvador, Bahia, Brasil \\
pablovf@ifba.edu.br}
}
% Desenvolver uma plataforma completa de transparência política que democratize o acesso aos dados da Câmara dos Deputados, promovendo maior engajamento democrático através de tecnologia, gamificação e participação social.
\maketitle

\textbf{\textit{O projeto "Tô De Olho" foi concebido para promover a transparência política e fortalecer a cidadania em Salvador, Bahia, utilizando uma aplicação web que organiza e apresenta dados relevantes de forma acessível. Além de dados sobre contratos públicos, frequência parlamentar, produtividade legislativa, proposições e gastos públicos, a plataforma conta com um fórum interativo. Esse espaço permite que cidadãos criem discussões, compartilhem ideias e debatam temas relacionados à gestão pública, ampliando o engajamento cívico e fomentando uma democracia digital mais participativa. A arquitetura do sistema inclui microsserviços que coordenam a coleta, processamento e exibição dos dados, garantindo confiabilidade e atualização constante. Combinando tecnologias modernas e foco na usabilidade, o "Tô De Olho" busca democratizar o acesso à informação pública e criar um espaço digital que incentive a cidadania ativa e colaborativa.}}\\

Palavras-chave: transparência política, democracia digital, cidadania ativa, fórum cívico, engajamento social, dados abertos.\\

\textbf{\textit{The "Tô De Olho" project was designed to promote political transparency and strengthen citizenship in Salvador, Brazil, through a web application that organizes and presents relevant data in an accessible way. In addition to providing information on public contracts, parliamentary attendance, legislative productivity, proposals, and public expenditures, the platform features an interactive forum. This space allows citizens to create discussions, share ideas, and debate topics related to public management, enhancing civic engagement and fostering a more participatory digital democracy. The system's architecture includes microservices that coordinate data collection, processing, and presentation, ensuring reliability and constant updates. By combining modern technologies and a focus on usability, "Tô De Olho" aims to democratize access to public information and create a digital space that encourages active and collaborative citizenship.}}\\

Keywords: political transparency, digital democracy, active citizenship, civic forum, social engagement, open data.\\


\section{Introdução}

O avanço das Tecnologias da Informação e Comunicação (TICs) tem transformado significativamente a interação entre governos e cidadãos, consolidando a democracia digital como uma ferramenta essencial para a gestão pública moderna. Esse conceito, descrito por Gomes \cite{Gomes2010} como o uso estratégico de ferramentas tecnológicas para ampliar a participação cidadã e a transparência governamental, fortalece o papel do cidadão como agente ativo no processo democrático. No contexto das "cidades inteligentes", onde tecnologia e sustentabilidade se combinam para otimizar recursos e melhorar a qualidade de vida, a transparência e o engajamento social tornam-se elementos indispensáveis \cite{Saikali2021}.

A democracia digital, segundo Costa \cite{duarte2019}, transcende a simples digitalização de processos administrativos; ela representa uma transformação nos mecanismos de governança, promovendo acesso à informação, colaboração e tomada de decisão coletiva. Ferramentas como \textit{e-government}, plataformas colaborativas e espaços interativos têm sido fundamentais para facilitar o acesso às informações públicas e incentivar uma participação cidadã mais ativa e inclusiva \cite{farias2013possibilidades}.

Nesse contexto, o projeto \textit{Tô De Olho} apresenta-se como uma solução abrangente para democratizar o acesso à informação pública em Salvador, Bahia. A plataforma combina tecnologia e usabilidade para oferecer um espaço digital onde cidadãos podem acessar dados sobre contratos públicos, frequência parlamentar, produtividade legislativa, proposições de leis e gastos com viagens. O fórum interativo integrado à plataforma atua como um espaço dinâmico de participação cidadã, permitindo que usuários compartilhem ideias, debatam questões relacionadas à gestão pública e colaborem na construção de soluções. Essa interação fortalece o engajamento social e fomenta a fiscalização cívica baseada em dados concretos, ampliando o impacto da transparência governamental \cite{ulbricht2020scraping}.


A arquitetura do sistema segue um modelo de microsserviços, onde cada componente desempenha uma função específica, garantindo a integração e confiabilidade das informações públicas. \textit{Crawlers} automatizam a extração de dados diretamente dos portais de transparência, enquanto o \textit{json-processor} organiza e valida as informações coletadas. A API centraliza o acesso a esses dados, facilitando a sua distribuição para diferentes consumidores, enquanto o \textit{frontend} se encarrega de apresentar as informações de maneira clara e acessível ao público. Esse ecossistema modular permite atualizações rápidas e reduz o risco de falhas sistêmicas, alinhando-se às práticas recomendadas em soluções baseadas em microsserviços \cite{munaf2019microservices}.

Ao explorar a interseção entre tecnologia, democracia digital e engajamento cívico, este trabalho destaca como o \textit{Tô De Olho} se posiciona como uma ferramenta inovadora para fortalecer a cidadania ativa e promover a transparência governamental. A combinação de tecnologias modernas e um foco no usuário reforçam o papel da plataforma como um instrumento de transformação social e política, alinhado aos ideais de gestão pública democrática e participativa \cite{duarte2019}.

Este estudo está estruturado da seguinte forma: a Seção 2 apresenta uma revisão da literatura sobre democracia digital, incluindo temas como transparência, \textit{web scraping}, microsserviços, \textit{Docker Compose}, \textit{APIs}, \textit{brokers}, usabilidade e engajamento cívico. A Seção 3 analisa trabalhos relacionados, destacando iniciativas semelhantes e como elas se conectam ao escopo do \textit{Tô De Olho}. A Seção 4 detalha a solução proposta, descrevendo os componentes e funcionalidades do sistema. A Seção 5 avalia os resultados e impactos do projeto, enquanto a Seção 6 conclui o trabalho com considerações finais e direções futuras para o aprimoramento da plataforma.


\section{Revisão da literatura}

Para compreender a importância do \textit{Tô De Olho} no contexto da democracia digital, esta seção apresenta uma revisão dos principais conceitos e tecnologias envolvidas na transparência governamental e participação cidadã. Inicialmente, são discutidos os fundamentos da democracia digital e seus impactos na governança pública. Em seguida, abordam-se as principais ferramentas tecnológicas aplicadas à coleta e processamento de dados abertos, incluindo \textit{web scraping}, microsserviços e \textit{Docker Compose}. Por fim, analisa-se a relação entre usabilidade e engajamento cívico, destacando como a acessibilidade e a interatividade das plataformas digitais influenciam a participação dos cidadãos.

\subsection{Democracia Digital}

A democracia digital, também chamada de ciberdemocracia ou e-democracia, surge como uma evolução natural das práticas democráticas em meio ao avanço acelerado das Tecnologias da Informação e Comunicação (TICs). Como destaca Gomes \cite{Gomes2010}, trata-se de um conceito que busca expandir e aprimorar as democracias, integrando ferramentas tecnológicas aos processos participativos para proporcionar maior transparência, acessibilidade e engajamento cidadão em decisões coletivas.

Sua trajetória reflete uma transformação gradual ao longo das últimas décadas, marcada por três grandes fases. No início, entre os anos 1970 e 1990, predominavam experimentos baseados em assembleias eletrônicas, onde cidadãos eram convidados a deliberar sobre questões públicas em um ambiente controlado. Com a expansão da internet nos anos 1990, surgiu o conceito de “ágora virtual”, prometendo uma participação mais ampla e descentralizada. A partir de 2005, a democratização digital ganha novas formas práticas, consolidando-se em iniciativas de governo aberto, dados abertos e plataformas de participação cidadã, alinhadas ao conceito de cidades inteligentes \cite{Saikali2021}.

Entretanto, um dos principais desafios enfrentados, sobretudo em países como o Brasil, está na exclusão digital. Dados da PNAD revelam que cerca de 15,2 milhões de domicílios ainda não possuem acesso à internet, o que limita o alcance dessas iniciativas e reforça as desigualdades sociais \cite{pnad2024}. Essa exclusão afeta especialmente populações rurais e comunidades em situação de vulnerabilidade, criando uma barreira significativa entre aqueles que podem se engajar nos processos digitais e aqueles que permanecem marginalizados \cite{Saikali2021}. O acesso à infraestrutura digital, contudo, não é a única barreira. Mesmo entre os que têm acesso à internet, a falta de letramento digital e a complexidade das plataformas muitas vezes reduzem a participação efetiva \cite{lemos2013cidades}. É necessário, portanto, que políticas públicas promovam não apenas inclusão digital, mas também educação tecnológica para garantir que a democracia digital alcance seu verdadeiro potencial.

No cenário brasileiro, algumas iniciativas se destacam positivamente, como o Portal da Transparência, que aumenta a \textit{accountability} ao permitir que cidadãos monitorem o uso de recursos públicos \cite{avelino2021democracia}. Contudo, práticas como a “pseudoparticipação”, em que a contribuição cidadã se limita a endossar decisões já tomadas, ainda comprometem o impacto dessas ferramentas \cite{farias2013possibilidades}. Como observa Pateman \cite{pateman1970}, a verdadeira democratização exige mais do que acesso à informação; ela requer a possibilidade de influenciar diretamente os processos decisórios. A falta de padronização e acessibilidade dos dados divulgados por órgãos públicos também representa um obstáculo, dificultando a análise crítica por parte da sociedade civil \cite{duarte2019}. Para que a transparência seja realmente efetiva, é indispensável oferecer mecanismos que tornem as informações compreensíveis e úteis para a tomada de decisão coletiva.

Legislações como a Lei de Acesso à Informação (LAI) e o Marco Civil da Internet representam marcos fundamentais para a transparência e participação digital no Brasil \cite{avelino2021democracia}. A LAI, instituída em 2011, estabelece diretrizes para que órgãos públicos disponibilizem informações de interesse coletivo de forma acessível e padronizada. No entanto, muitos desses dados ainda são divulgados de maneira fragmentada e pouco estruturada, o que dificulta sua utilização \cite{farias2013possibilidades}. Iniciativas como o orçamento participativo mostram que, onde bem implementadas, há um aumento na confiança nas instituições e no engajamento nas decisões públicas \cite{Saikali2021}. Ainda assim, a digitalização desses processos traz desafios consideráveis, exigindo infraestrutura tecnológica adequada e organização, algo que muitos municípios ainda não conseguem oferecer \cite{duarte2019}.

A democracia digital se apresenta como uma oportunidade única para modernizar e fortalecer os processos democráticos, mas para que suas promessas sejam cumpridas, é necessário enfrentar questões estruturais e investir em plataformas que promovam a interação genuína entre cidadãos e o Estado. Esse modelo de “gestão pública digital” \cite{duarte2019} depende de um alinhamento entre inclusão, participação e transparência, criando uma sociedade mais justa, engajada e conectada.

%TODO:APRIMORAR ESSA PARTE
\subsection{Web Scraping e Processamento de Dados}

O web scraping emergiu como uma técnica indispensável para coletar informações de fontes online, especialmente em contextos onde o acesso por meio de APIs não está disponível. Essa prática permite a extração automática de dados diretamente de páginas da web, facilitando a criação de bases de dados estruturadas para análises avançadas \cite{vela2019semi}. Após o escândalo da Cambridge Analytica, quando o acesso às APIs foi significativamente restringido, o \textit{web scraping} tornou-se uma alternativa crucial para pesquisadores que buscam estudar o impacto das redes sociais no comportamento político \cite{mancosu2020what}.

Apesar de seu potencial, a prática levanta importantes questões éticas relacionadas à privacidade e proteção de dados, especialmente no contexto do Regulamento Geral de Proteção de Dados (GDPR). Recomenda-se o uso de técnicas de anonimização e a coleta responsável de informações, garantindo que apenas dados públicos sejam processados \cite{mancosu2020what}. Em paralelo, alguns estudos defendem o \textit{web scraping} como uma ferramenta democrática, capaz de promover maior transparência nos processos políticos, ainda que existam limitações significativas devido a possíveis vieses interpretativos \cite{ulbricht2020scraping}.

No domínio da governança pública, métodos semiautomáticos de raspagem têm sido aplicados para coletar dados sobre sistemas de transporte público, demonstrando como a automação pode facilitar a criação de repositórios abertos e acessíveis de informações \cite{vela2019semi}. Esses exemplos destacam a importância de um processamento estruturado e consistente para garantir a qualidade das informações disponibilizadas.

%TODO: FAZENDO
\subsection{Arquitetura de Microsserviços}

A arquitetura de microsserviços configura-se como uma abordagem contemporânea e estratégica para o desenvolvimento de sistemas de software, especialmente aqueles que demandam alta escalabilidade, manutenibilidade e flexibilidade, como é o caso da plataforma "Tô De Olho". Diferentemente das arquiteturas monolíticas tradicionais, onde todas as funcionalidades são acopladas em um único bloco de código – o que pode levar a dificuldades de atualização, teste e evolução \cite{munaf2019microservices, taibi2017microservices}, os microsserviços propõem a segmentação do sistema em um conjunto de serviços menores, independentes e focados em responsabilidades de negócio específicas \cite{munaf2019microservices, maureal2024enhancing, taibi2017microservices}. Cada um desses serviços possui seu próprio ciclo de vida, podendo ser desenvolvido, implantado e escalado autonomamente, e se comunica com os demais através de interfaces bem definidas e protocolos leves, como APIs REST ou sistemas de mensageria assíncrona \cite{munaf2019microservices, maureal2024enhancing, john2025personalized}.

A decisão de adotar a arquitetura de microsserviços para o "Tô De Olho" fundamenta-se na busca por superar desafios inerentes a plataformas de dados públicos e transparência. Um dos principais benefícios esperados é a \textit{escalabilidade granular}. O "Tô De Olho" precisa lidar com volumes de dados públicos que podem variar significativamente (por exemplo, aumento de contratos durante certos períodos ou picos de acesso da cidadania em momentos de maior interesse político). Com microsserviços, é possível escalar horizontalmente apenas os componentes que realmente necessitam de mais recursos – como os \textit{crawlers} responsáveis pela coleta de dados ou o serviço de API que atende às consultas dos usuários – em vez de replicar toda a aplicação, otimizando custos e a alocação de infraestrutura \cite{munaf2019microservices, taibi2017microservices}. A \textit{manutenibilidade} do sistema também é um fator crucial. À medida que novas fontes de dados são adicionadas ou as existentes sofrem alterações, a capacidade de atualizar ou corrigir um \textit{crawler} específico sem impactar o restante da plataforma é vital. A modularidade dos microsserviços facilita essa gestão, permitindo que intervenções sejam realizadas de forma isolada e ágil \cite{munaf2019microservices, taibi2017microservices}. Milasus destaca precisamente como a divisão em microsserviços beneficia a escalabilidade e a manutenibilidade em sistemas de harmonização e análise de dados, um contexto análogo ao tratamento de dados públicos pelo "Tô De Olho" \cite{milasus2024design}.

Além disso, a \textit{independência tecnológica} proporcionada pelos microsserviços permite que o "Tô De Olho" utilize as ferramentas mais adequadas para cada tarefa. Por exemplo, os \textit{crawlers}, responsáveis pela extração de dados de portais governamentais muitas vezes complexos e dinâmicos, podem ser implementados com JavaScript e bibliotecas como Playwright ou Puppeteer, enquanto o \textit{backend} da API central, que exige processamento eficiente de requisições e acesso a bancos de dados, pode ser desenvolvido em Golang, conhecido por seu desempenho e capacidades de concorrência \cite{munaf2019microservices, taibi2017microservices}. Essa flexibilidade é essencial para um projeto que integra diversas tecnologias.

A relevância dos microsserviços para plataformas de \textit{e-Governance} e transparência é corroborada por diversas iniciativas. O sistema \textit{NTC-EDGE}, por exemplo, modernizou processos regulatórios nas Filipinas utilizando microsserviços, alcançando uma redução de 40\% no tempo de processamento e um aumento de 25\% na satisfação dos clientes, além de eliminar brechas de segurança \cite{maureal2024enhancing}. Embora o "Tô De Olho" não seja um sistema transacional governamental no mesmo sentido, os ganhos de eficiência e a capacidade de gerenciar dados de forma segura e modular são diretamente aplicáveis. Similarmente, a tese de Milasus demonstra a aplicação de microsserviços para a coleta flexível (via APIs ou arquivos) e o pré-processamento automatizado de dados, ecoando as necessidades do "Tô De Olho" de ingerir e tratar informações de fontes heterogêneas \cite{milasus2024design}. A proposta de John \& Ali para um sistema de \textit{e-voting}, embora com foco em \textit{blockchain}, também se apoia em microsserviços para prover escalabilidade, tolerância a falhas e responsividade em tempo real, qualidades desejáveis para uma plataforma de engajamento cívico como o "Tô De Olho" \cite{john2025personalized}.

No contexto específico do "Tô De Olho", a arquitetura de microsserviços permite uma clara separação de responsabilidades entre os componentes chave: os múltiplos \textit{crawlers}, o \textit{json-processor} (responsável pela limpeza, validação e enriquecimento dos dados), a API central (que expõe os dados tratados) e o \textit{frontend} (que os apresenta aos cidadãos). Essa estrutura não só simplifica o desenvolvimento e a manutenção de cada parte, como também aumenta a resiliência geral da plataforma: uma eventual falha em um \textit{crawler} específico não deve comprometer a coleta de dados de outras fontes ou a disponibilidade das informações já processadas para consulta. A execução agendada dos \textit{crawlers} via \textit{cron jobs} também se integra naturalmente a essa arquitetura de modo a ter-se agendamentos frequentes da execução deles para a captação de dados, podendo ser definida em prazos personalizáveis - no caso do projeto, diários.

Apesar dos inúmeros benefícios, a implementação de uma arquitetura de microsserviços para o "Tô De Olho" exige atenção a desafios inerentes a sistemas distribuídos, de modo que a comunicação entre serviços é um ponto central. Para compartilhar as informações entre os serviços independentes, o "Tô De Olho" precisou de um incremento de estratégias, como uso de repositórios compartilhados, comunicação assíncrona (como no exemplo das mensagerias) e protocolo \textit{REST} - práticas altamente sugeridas e defendidas por Munaf et al. \cite{munaf2019microservices}.

Para gerenciar a implantação, o escalonamento e o ciclo de vida dos contêineres que encapsulam os microsserviços do "Tô De Olho", ferramentas de \textit{orquestração de contêineres} como Docker Compose (adequado para ambientes de desenvolvimento e pequenas implantações) ou Kubernetes são praticamente indispensáveis. Milasus também descreve o uso de Docker e Kubernetes para sua plataforma de harmonização de dados \cite{milasus2024design}, o que motivou a implementação do Docker para o projeto proposto. Finalmente, a experiência documentada por Taibi et al. sobre os processos de migração para microsserviços ressalta que, embora as motivações principais sejam melhorias em manutenibilidade e escalabilidade, o esforço inicial de planejamento, decomposição de funcionalidades e configuração da infraestrutura (incluindo DevOps) é considerável e deve ser previsto \cite{taibi2017microservices}, fortalecendo a razão do esforço inicial do projeto em seguir práticas de \textit{CI/CD}, visando eficiência na qualidade das entregas de futuras features e menor tempo de implantação.

Em suma, a escolha da arquitetura de microsserviços para o "Tô De Olho" visa dotar a plataforma da robustez, escalabilidade, adaptabilidade e disponibilidade necessárias para cumprir sua missão de promover a transparência política e o engajamento cívico em Salvador, enfrentando os desafios de um sistema de informação pública dinâmico e em constante evolução, além de manter-se efetivamente atualizado com as tecnologias mais recomendadas para a função do projeto.



\section{Trabalhos relacionados}
A fazer.

\section{Metodologia de trabalho}
A fazer.

\section{Proposta}
A fazer.

\section{Resultados}
Aqui estão os resultados.

\section{Conclusão}
Aqui está a conclusão.

\section*{Agradecimentos}
A fazer.

\bibliography{referencias}

\end{document}
